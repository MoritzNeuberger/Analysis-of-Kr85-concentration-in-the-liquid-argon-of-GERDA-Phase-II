	%%%%%%%%%%%%%%%%%%%%%%%%%%%%%%%%%%%%%%%%%%%%%%%%%%%%%%%%%%%%%%%%%%%%%%%%%%%%%%%%
% ---------------------------------------------------  ADD HERE YOUR REPORT  ---
%%%%%%%%%%%%%%%%%%%%%%%%%%%%%%%%%%%%%%%%%%%%%%%%%%%%%%%%%%%%%%%%%%%%%%%%%%%%%%%%

%sources:
% 0: https://arxiv.org/pdf/1006.1718v2.pdf





\chapter{Introduction}
\label{sec:intro}

The GERDA experiment tries to find evidence for the existence of the neutrino less double beta decay in \nuc{Ge}{76}.
Normal double beta decay is already well observed and always occurs when a state relaxes into a lower state over an energetically raised state via two simultaneous beta decays.
The neutrino less double beta decay is a special kind of double beta decay that can only occur when neutrinos are majorana fermions.
Majorana fermions have the characteristic that they are particle and antiparticle at the same time, breaking the lepton number conservation.
In the case of such a neutrinoless double beta decay, both neutrinos emitted during the two regular beta decays would immediately annihilate and the excess energy would be split up between the two escaping electrons.
Their energy can then be measured in a detector where they create a sharp peak at the endpoint energy of the regular double beta decay.
\\

The investigation of the neutrinoless beta decay in \Ge is also of historical interest, since in the Heidelberg Moscow experiment in 2003 had claimed to have measured exactly this decay in \Ge.
There have been many follow-up experiments to the Heidelberg Moscow experiment that showed that its results were incorrect.
However, \Ge is still very suitable for further investigations of neutrino less double beta decay, among other things because germanium itself can already be used as a detector.
\\

Because it is expected that the neutrinoless double beta decay has a very long lifetime it is very important to filter out every source of background radiation as good as possible. 
Therefor it is necessary to use a low temperature coolant to freeze the \nuc{Ge}{67} detectors down, to shield the detectors from external radioactive sources and to use a scintillator around the detectors to filter out any radiation coming from the outside. 
Due to the \Ge detectors also being the source of the double beta decay at the same time, all signal coming from the outside are background event. 
\\

Liquid argon is a fitting material for both of requirements listed above due to its low freezing point and its ability to scintillate. 
Commercial argon is extracted from the air by air liquefaction and therefor has residual foreign isotopes. 
The majority of the impurity in the liquid argon can be removed by cryogenic distillation but even the remaining alien isotopes are still very active.\\

One of these residual radioactive isotopes is \Kr. 
Compared to \nuc{Ge}{67} it has a relatively low endpoint energy and therefor shouldn't create any fake neutrino less double beta decays. 
It also isn't the strongest radioactive background source. 
This title belongs to \nuc{Ar}{42}. 
What is interesting about this isotope is the fact that first approximations of its concentration in the liquid argon showed that it is at least 10\(^{-3}\) times smaller than values measured in other experiments using liquid argon.
It is therefor of interest to determine a concrete value of its activity. % !!!!! hier die Abschätzung hin oder sich eine andere Begründung für meine Bachelorarbeit ausdenken !!!!!
\\

The aim of this theses is to determine the concentration of the \Kr in the liquid argon coolant and scintillator of the GERDA experiment and by this determining its influence to the radioactive background. 
This is accomplished by two different approaches that are further elaborated on later. \\

!!!!! Das hier zwischen den !!!!! auskommentieren

The first method uses the relatively rare decay of \Kr into an excited state of \nuc{Rb}{85} with a probability of 0.43\%. 
This excited state relaxes to its ground state by emitting a photon with 514keV of which there should be a noticeable peak in the spectrum. 
The overall activity of the \Kr can than be determined with the peak in the spectrum. 
\\

The second approach takes a look at the overall intensity reduction over time and determines the activity of the \Kr from the amplitude of the exponential decrease. 
These two methods are supposed to complement each other showing the accuracy of the resulting concentration.\\

!!!!!

The first chapter of this theses concentrates on the physical background discussing the ideas behind the different neutrino fermion models and their consequences. 
It is also described what a double beta decay is and why it is hoped to find a neutrino less double beta decay. 
After that the usefulness of argon as a coolant will be discussed and lastly follows a characterization of the \Kr.\\

In the second chapter the GERDA experiment itself will be the main focus. 
The aim of GERDA as well as its experimental setup and background reduction will be described and the current results discussed.\\

The third chapter is the main part of this theses. 
It will focus on how the concentration of \Kr will be determined. 
First of all the theoretical ideas behind and the problems of the two methods will be discuss. !!!!! das sollte ich wohl am besten als letztes schreiben !!!!!

% what is the general aim of my Bachelor thesis
% general overview of how I'm going to do this
% ? what are my predictions ?
% What possible influence would the result of my work might have on the result of the GERDA experiment?

\chapter{Physical Background}
\label{sec:PhyBG}

\section{Weyl, Majorana and Dirac fermions}
\label{sec:WMDf}


Dieser Teil wird noch stark gekürzt bzw. ich werde mich hier wahrschenlich vor allem auf "Kerne und Teilchen", der KTA Script und ein paar einfürhende Paper stützen.

% in this chapter almost everything from source 0

\begin{itemize}
\item historic introduction of question whether neutrinos are dirac or majorana fermions
\begin{itemize}
\item from Dirac relativistic equation fermion fields
\item electrons: have mass and charge, Dirac-eq predicts antiparticles, requires 4-comp fields
\item Weyl calculates for massless fermions that only two-component fields are necessary
\item Pauli predicts neutrinos in letter, no charge, seems to have vanishing mass 
\item \(\rightarrow\) assumption: neutrinos  are Weyl fermions
\item Majorana: neutrinos are antiparticles of itself since they are uncharged, first not taken seriously, only after first indications that neutinos have mass
\item \(\rightarrow\) discussion whether neutrinos are Dirac or Majorana fermions
\end{itemize}
\item short introduction to dirac-eq
\begin{itemize}
\item \((i\hbar\gamma^\mu \partial_\mu  - mc)\psi = 0\) , maybe Hamilton/Lagragian, has plane waves as solution multiplied with spinor
\item Spinors: any column like function of energy and momentum which when multiplied by factor \(\exp(i\vec{p}\cdot\vec{x})\) or  \(\exp(\-i\vec{p}\cdot\vec{x})\) becomes a solution of the Dirac equation
\end{itemize}
\item we know that the Klein-Gordon-eq is real, how can we get a real solution from the dirac-eq
\begin{itemize}
\item depends on how we choose our \(\gamma^\mu\), if all non-zero elements of all the \(\gamma^\mu\) are purly imaginary, then Dirac-eq is real
\item Majorana matrices, with usage in Dirac-eq one can obtain real solutions that satisfy \(\tilde{\psi} = \tilde{\psi}^*\)
\item \(\rightarrow\) these solutions represent Majorana fermions
\item general solution to Majorana condition can be obtained by transformation with unitary matrix: \(\gamma^\mu = \mathrm{U}\tilde{\gamma^\mu}\mathrm{U}^\dagger\)
\item general Majorana condition: \(\psi = \mathrm{U}\mathrm{U}^\top\tilde{\psi}^*\), with \(\mathrm{U}\mathrm{U}^\top \equiv \gamma^0 \mathrm{C}\)
\item with compact notation \(\widehat{\psi} \equiv \gamma^0 \mathrm{C} \psi^*\)
\item general definition of a Majorana fermion fields through definition: \(\psi = \widehat{\psi}\), condition is Lorenz invariant
\end{itemize}
\item clunky repetition of what helicity and chirality is and what their problem with massive fermions is
\begin{itemize}
\item helicity defined as twice the value of the spin component of the particle along the direction of the momentum \(h_{\vec{p}} = \frac{2 \vec{J} \cdot \vec{p}}{p}\)
\item eigenstates of eigenvalue -1 called "left-helical", eigenstates of eigenvalue +1 called "right-helical"
\item is invariant under time/rotation, not invariant under boost
\item chirality meaning assigned to matrix \(\gamma_5 = i\gamma^0\gamma^1\gamma^2\gamma^3\)
\item projection matrices on fermion fields: \( \mathrm{L} = \frac{1}{2} \left( 1- \gamma_5\right ) \), \( \mathrm{R} = \frac{1}{2} \left( 1 + \gamma_5\right ) \)
\item projections of L, R are called lift/right-chiral
\item wavefunction can be written as \(\psi = \psi_L + \psi_R\) with \(\psi_L = \mathrm{L}\psi\) and \(\psi_R = \mathrm{R}\psi\)
\item \(\rightarrow\) helicity: conserved for free particles, not under Lorenz trafo
\item \(\rightarrow\) chirality: is Lorenz invariant, not conserved
\item \(\rightarrow\) both not appropriate for characterizing a fermion that has mass
\end{itemize}
\item how to define Weyl fermions
\begin{itemize}
\item problem with helicity/chirality disappears if the fermion is massless
\item general solution of Dirac-eq is not irreducible representation of Lorentz group (Lorenz group: group of Lorentz trafo in Mirkowski-space-time)
\item \(\rightarrow\) left/right-chiral fields are Lorentz invariant, representation with 2-component-field: \( \begin{pmatrix}\frac{1}{2} \\ 0\end{pmatrix}\) for left-chiral, \(\begin{pmatrix}0 \\ \frac{1}{2}\end{pmatrix}\) for right-chiral
\item when \(\chi\) is left chiral Weyl fermions, \( \widehat{\chi} \) is a right chiral Weyl fermions
\item a general fermion field can be described by two Weyl fields \(\rightarrow\) building blocks
\end{itemize}
\item how can we build Majorana/Dirac fermions from Weyl fermions
\begin{itemize}
\item Majorana and Dirac fermions both have mass and therefor must have both left/right chiral components
\item Dirac can be created by two independent left chiral Weyl fields \(\chi_1\), \(\chi_2\): \(\psi = \chi_1 + \widehat{\chi_2}\)
\item Dirac fermions are unconstrained solutions of the Dirac equation
\item unlike the Dirac fermions, the Majorana fermions must fulfill the reality condition \( \psi = \widehat{\psi} \)
\item Majorana fermions are represented by: \( \psi = \chi + \widehat{\chi} \)
\item where does mass come from?: mass term in Dirac-eq is of from \(\bar{\psi}\psi\), only \(\bar{\psi_L}\psi_R\) and \(\bar{\psi_R}\psi_L\) remains, \(\bar{\psi_L}\psi_L\) and \(\bar{\psi_R}\psi_R\) cancel out
\item \(\rightarrow\) Weyl fermions has special chirality therefore mass term must vanish, massive fermions must have left-chiral and right-chiral components
\end{itemize}
\item Dirac fields are completely unconstrained solutions of Dirac equation
\item Weyl/Majorana fields are simpler solutions with some kind of constrained imposed on solution, Weyl: chirality condition, Majorana: reality condition
\end{itemize}
% hier muss ich mich ein wenig einlesen ... mögliche Literatur:
% KTA-Expert script
% 0: https://arxiv.org/pdf/1006.1718v2.pdf

\subsection{Neutrinoless Double Beta Decay}
\label{sec:NDBD}
\begin{itemize}
\item Neutrino Oscillation have shown that neutrinos have finite mass, with NeOs only difference in mass measurable, lower limit on absolute mass with \(\mathrm{m}_{scale} = \sqrt{\Delta \mathrm{m}^2}\)
\begin{itemize}
\item SuperKamiokande showed mixing between \(\nu_\mu\) and \(\nu_\tau\) of atmospheric neutrinos
\item "solar neutrino puzzle" solved with mixing of \(\nu_e\) and mixture of \(\nu_\mu\) and \(\nu_\tau\)
\item NeOs can not determin absolute masses and also not separate between two different scenarios:
\begin{itemize}
\item hierarchical pattern ( \(m_i\)  ~= \(\sqrt{\Delta m^2}\))
\item degenerate pattern ( \(m_i >> \sqrt{\Delta m^2}\))
\end{itemize}
\end{itemize}
\item \(\beta\beta(0\nu)\) decay can only proceed when Neutrinos are massive Majorana particles
\begin{itemize}
\item standard electroweak model postulates that neutrinos are massless and total lepton number is conserved -> with \(\beta\beta(0\nu)\) physics beyond SM
\item double beta decay is rare transition between two nuclei with the same mass number A involving change of nuclear charge Z by two units
\begin{itemize}
\item can only proceed if initial nucleon is less bound than final and both are more bound than intermediate nucleon -> only fulfilled for even-even nucleons
\item \(\beta\beta(2\nu)\): \( (Z,A) \rightarrow (Z + 2, A) + e^-_1 + e^-_2 + \bar{\nu_{e1}} + \bar{\nu_{e2}} \), conserves lepton number
\item \(\beta\beta(0\nu)\): \( (Z,A) \rightarrow (Z + 2, A) + e^-_1 + e^-_2\), violates lepton number conservation
\item \(\beta\beta(0\nu, \chi)\): \((Z,A) \rightarrow (Z + 2, A) + e^-_1 + e^-_2 + \chi \)
\end{itemize}
\item easy to distinguish the three decay modes by shape of \(e^-\)-sum energy spectrum
\begin{itemize}
\item 2\(\nu\): broad maximum below half of endpoint
\item 0\(\nu\): \(e^-\) carry full available kinetic energy, single peak at endpoint
\item 0\(\nu,\chi\): again continuous, maximum shifted above halfway point
\end{itemize}
\end{itemize}
\item Majorana, Dirac neutrinos (again from above, maybe move stuff there)
\begin{itemize}
\item Majorana: particles that are identical with their own antiparticles, two component objects
\item Dirac: one can distinguish, four component objects
\item massive fermions usually described by Dirac eq with coupling of chiral eigenstates \(\psi_L,\psi_R\), \(\Psi = \begin{pmatrix} \psi_R \\ \psi_L\end{pmatrix}\)
\item Majorana suggested alternative description of massive fermions which do not have additive quantum numbers as two component states, chiral eigenstates connected via \(\psi_L = \epsilon \psi_R^*\)
\end{itemize}
\item Lorentz invariant mass in Dirac Lagrangian:
\begin{itemize}
\item Dirac mass: \(M_D [\bar{\nu_R}\nu_L + (\bar{\nu_L})^*\nu_R^*]\), requires both chirality eigenstates, conserves total lepton number
\item Majorana mass: \(M_L [(\bar{\nu_L})^*\nu_L + \bar{\nu_L}\nu_L^*]\), \(M_L [(\bar{\nu_R})^*\nu_R + \bar{\nu_R}\nu_R^*]\), violates total lepton number conservation, can be present even w/o existents of \(\nu_L/\nu_R\)
\item generally all three terms may coexist, when Lagrangian is diagonalized the resulting two general non-degenerate mass eigenvalues for each flavor (see-saw: light and heavy particle for each flavor) \(M = \begin{pmatrix} M_L & M_D \\ M_D & M_R\end{pmatrix}\) 
\item M diagonalized by unitary matrices \(\begin{pmatrix} U \\ V \end{pmatrix}\), U,V general mixing matrices, if non of the \(\nu_R\) states exist or \(M_R\) is so large only \(M_L\) is relevant and only U needed
\end{itemize}
\item Majorana mass
\begin{itemize}
\item transition amplitude for Majorana neutrino mass \(m_e\) is sum over product of \(m_j\) and combination of nuclear mixing element
\item in each vertice electron is emitted, therefor mixing amplitude \(\mathrm{U}_{ej}\) appears in each of them
\item \(\Rightarrow \beta\beta(0\nu)\) decay amplitude contains factor \(\mathrm{U}_{ej}^2\) not \(|\mathrm{U}_{ej}|^2 \Rightarrow <m_\nu> = |\sum_jm_j\mathrm{U}_{ej}^2|\) 
\end{itemize}
\item oscillation parameters
\begin{itemize}
\item \(<m_\nu>^2 = |\sum_jm_j\mathrm{U}_{ej}^2|^2 = |\sum_jm_j|\mathrm{U}_{ej}|^2e^{i\alpha}|^2\) with \(\alpha\) being the Majorana phase
\item possibility of cancellation of sum (Zee model \(<m_\nu> = 0\))
\item \(<m_\nu> = 0\) depends on unknown phase but upper/lower limit only depends on absolute values of mixing angles\(<m_\nu> = \sum_jm_j|\mathrm{U}_{ej}|^2\)
\item when \(<m_\nu>\) is known from tritium beta decay experiments one can determine phase \(\alpha_i\)
\end{itemize}
\end{itemize}

% also a bit about standard double beta decay
% differences between the standard and neutrinoless beta decay
 


\capter{\Kr isotop}
\label{sec:Kry85}

% where does it come from?
% what properties does it have?
% why is it important to calculate its influence on GERDA

\chapter{GERDA}
\label{sec:GERDA}

% general Information, e.g. 
% sizes 
% Gran Sasso, 
% what other  neutrino less beta decay experiments are there,

\section{Aim of \GERDA}
\label{sec:AimGERDA}

% not sure whether this is enough for its own section, maybe just connect it with general info


\section{Liquid Argon as Coolant, Shielding and Scintillator} 
\label{sec:LArcoolant}


\section{Experimental Setup}
\label{sec:ExSetup}

% well, just dump everything here
% also a bit about Tier1-4 storage of data 

\section{Background Reduction}
\label{sec:BGReduction}


% Mui, Mountain, Pulse shape disc.
% especially LAr-Veto 

\section{Recent Results}
\label{sec:ResultsofGERDA}

% what has happened so far

% short outline of approaches
 
\chapter{Line Count Rate Analysis}
\label{sec:SAfrom514}


As stated in the introduction, the main goal of this theses is to approximate the concentration of the isotope \Kr in the liquid argon coolant, shielding and scintillate of the GERDA experiment. 
This value is of special interest, as in other experiments this isotope in the liquid argon has produced a not negligible background. 
For example, a specific activity of  \((2.86\pm0.18) \frac{\unit{mBq}}{\unit{l}}\) was measured in the Darkside experiment \cite{PhysRevD.93.081101} and \((0.16\pm0.13)\frac{\unit{Bq}}{\unit{l}}\) in the WARP experiment \cite{Benetti:2006az}.
This corresponds to a concentration of about \((2.32\pm0.14)\times10^{-18}\frac{\unit{mol}}{\unit{l}}\) for the Darkside and  \((1.30\pm1.05)\times10^{-16}\frac{\unit{mol}}{\unit{l}}\) for the WARP experiment.
The concentration of \Kr in the GERDA experiment therefor of interest as it provides an estimation of its influence on the measured background.
\\

In the curse of this theses two different approaches are used to calculate the concentration of \Kr. 
The first and more precise method is the line count rate analysis of the \Kr decay into an excited state of \nuc{Rb}{85m}. 
It is also the topic of this chapter.
The second method that is discussed in the following chapter uses the reduction in intensity of all events over time. 
This method is less likely to create precise results due to it relying on a great approximation.
It is estimated that only \Kr's activity has a notable change with time whereas the activity of all other radioactive isotopes stay relatively constant. 
It is therefor intended to only be a crosscheck for the result of the first method.\\
\\

Now to how exactly the line count rate analysis is able to determine a specific activity.
As discussed in section \ref{sec:Kry85} has \Kr the property, that it has a small probability of 0.43\% to decay into an excited state of \nuc{Rb}{85m}. 
This state has a raised energy level of 514keV compared to its ground state and a half-life of 1.015\(\unit{\mu s}\). 
Theoretically it should be possible to measure a peak in the spectrum around the 514keV of by the photons emitted from the relaxation of \nuc{Rb}{85m}.
In the spectrum around this peak one can then apply an Gaussian peak fit to determine the number of events measured by the detectors in this area. 
\\

It is also necessary to calculate the efficiency of the detector to measure any photons with an energy of 514keV.
But this detector efficiency can't be calculated by just using the measured data.
It has to be determined by using a Monte Carlo simulation creating a huge amount of 514 keV photons in a constant volume and counting the measured events in the detector. 
The detector efficiency can then be calculated by dividing the simulated measured events by the total number of simulated decays.
This value can then be used in a conversation factor to determine the amount of decays actually necessary to create the measured peak.
\\

The final value needed to calculate the specific activity is the measuring time.
Because not every detector was measuring over the course of Phase II a mean measuring time for all detectors has to be calculated.
Now finally with the values of the amplitude, the conversion factor and the mean measuring time a mean specific activity can be determined. 
The line count rate analysis is expected to generate a relatively precise estimation of the specific activity.
This is because only the photons of the \Kr decays can be the cause of the investigated change in spectrum whereas in the second method every isotopes theoretically creates a change in intensity over time distorting.
\\ 

However, the biggest problem of this method to overcome lies with the proximity of the \Kr to the 511keV peak of the positron electron annihilation. 
Its peak is expected to partially dominate over the \Kr in the spectrum and does not allow for a direct measurement of the 514keV peak. 
This is not necessary a great setback because we can just adapt our fit function to a double Gaussian peak function.
But it is also of interest whether it is possible to completely suppers the annihilation peak without changing the 514keV photon line.
Due to the low energy of the escaping electron (47.65keV) in the \Kr decay into the excited \nuc{Rb}{85m}, this specific decay is very unlikely to create any scintillating light in the liquid argon. 
On the other hand one can expect the light of the positron electron annihilation to create a great signal in the photomultiplier that are positioned in the liquid argon tank.
It should therefor be possible to single out the 514keV photon events from the annihilation events.
If possible this could be a second method to determine the amplitude of the peak with just one Gaussian peak.
\\

The rest of the chapter will now concentrate on the concrete implementation of the method.
As it was stated above, to determine the amount of measured 514keV photons a fit has to be applied to the spectrum around the peak. 
But before the measured spectrum can be used for our analysis we have to apply some filters.
These filters should filter out any event of which we can say with a high probability they were not caused by a \Kr decay.
Two very appropriate filters are the Muon Veto and the detector anti-coincidence veto.
How exactly these two filters work will be discussed now.
\\

As it was elaborated in section \ref{sec:ExSetup} all GERDA data is stored in a multi-tier data structure. 
All of the data used in this theses are from either tier 3 or from tier 4.
In these tiers a lot of analysis has already been carried out on the individual events measured. 
Among other things, each event in tier 3 and 4 is given a flag whether or not there was a coincidence of the event with a signal in one of the photomultiplier in the water tank.
This flag is called the Muon Veto and it always triggers whenever there is a strong light signal measured.
Due to the high kinetic energy needed to create any Cherenkov light no isotope from inside the germanium source and liquid argon should trigger this Muon Veto.
Especially no \Kr decay. 
This veto can therefor be used to filter out high-energy particles from outside and the background they create.
\\

For a photon from the relaxation of \nuc{Rb}{85m}to create the distinct 514keV peak it is necessary to deposit all of its energy in only one of the detectors. 
This means that any event in which two different germanium detectors measured a non neglectable energy deposition is most likely not caused by a \Kr decay.
This kind of filtering is called a detector anti-coincidence filter.
Whether an event has more than one detector measure a signal can be determined by the multiplicity counter of each event stored in tier 3 and 4.
Using this counter even more background of none \Kr decays can be repressed.
\\

After applying the first few rather general filters on the measured events one has to make a distinction between the events measured in different kinds of detectors.
As it was elaborated in section \ref{sec:ExSetup}, two different types of detectors were used in the GERAD experiment - the BEGes (Broad Energy Germanium diods) and the COAX (coaxial diods). 
Due to their differences in design and weigh the two types have a different energy resolution and detector efficiencies. 
The BEGes are generally smaller and therefor have a higher energy resolution compared to the rather big COAX detectors.
For example, at an energy of 514keV the BEGes have a resolution of about 2.267keV while the COAX detectors are about 0.5keV higher at 2,72keV \cite{agostini_allardt_bakalyarov_barabanov_baudis_bauer_bellotti_belogurov_belyaev_benato_et al._2017}. 
On the other hand, due to the BEGes being smaller they also have a lower detection efficiency.
It is therefor necessary to evaluate their measured data separately.
\\

Now that we have listed the steps necessary to adjust our measurement results so that we can determine the number of decays from we can finally plot the spectra of the corresponding detectors (see figure \ref{fig:NoFilterBEGes} for the BEGes and \ref{fig:NoFilterCOAX} for the spectra of the COAX detectors). 
In these two spectra we can clearly see two distinct peaks - one at 511keV that corresponds to the positron electron annihilation events and one at 514keV that corresponds to the photons from the relaxation of \nuc{Rb}[85m].
From the fact that we can see a distinct peak at the 514keV line we can already claim the fact that there must be a non neglectible amount of \Kr in the liquid argon.
Otherwise no peak should have been able to be measured. 
The difference in resolution can be seen in the fact that in the BEGe diagram the two peaks have a smaller full width at half value..
Compared to the COAX detectors their peaks can easily be distinguished.  
\\

\begin{figure}[t!]
\centering
\begin{subfigure}{.5\textwidth}
  \centering
	\includegraphics[width=80mm]{./Bilder/500525NoFilterBEGes.pdf}
  \label{fig:NoFilterBEGes}
  \caption{BEGes}
\end{subfigure}%
\begin{subfigure}{.5\textwidth}
  \centering
	\includegraphics[width=80mm]{./Bilder/500525NoFilterCOAX.pdf}
  \caption{COAX}
  \label{fig:NoFilterCOAX}
\end{subfigure}
    \\
	\vspace{0.5cm}
	\caption{All events measured by the respective detectors in the range of 500keV to 525keV.}
\end{figure}

\begin{figure}[t!]
\centering
\begin{subfigure}{.5\textwidth}
  \centering
	\includegraphics[width=80mm]{./Bilder/500525LArVetoBEGesEventList.pdf}
    \caption{BEGes}
  \label{fig:LArBEGesEV}
\end{subfigure}%
\begin{subfigure}{.5\textwidth}
  \centering
	\includegraphics[width=80mm]{./Bilder/500525LArVetoCOAXEventList.pdf}
  \caption{COAX}
  \label{fig:LArCOAXEV}
\end{subfigure}
    \\
	\vspace{0.5cm}
    \caption{All events measured by the respective detectors with the LAr filter applied and the manually recovered precoincedence events in the range of 500keV to 525keV.}
\end{figure}

Now that we have the two spectra of the corresponding detectors, we have two possible ways of how we can determine the number of measured events around the 514keV peak.
The easier way is to just use the spectra as they are and only change the used fit function so that it also takes the second peak into account when fitting.
Another possible approach tries to suppress the annihilation peak and then fit the spectrum with only one Gaussian peak.
This can hopefully be done by using the liquid argon veto of GERDA and the precoincidence of the electron scintillation in the photomultipliers. 
\\ 

In this theses we will try to go both ways separately and later compare their resulting values.
We will probably end up using the unsuppressed spectra to determine the amount of events measured and only use the other method as a crosscheck.
As we will see later this is because the used filter also filters out events from the 514keV line peak.
This results in a lower amount of events measured at this energy and therefore, in the end, gives us a lower activity than there is in actuality. 
But to understand why this filter also filters out these false positives we have to discuss how this filter works first.
This will be the topic of the following chapter.
\\


\section{Annihilation Peak Suppression}
\label{sec:APS}

As mentioned above, it should be theoretically possible to filter out the majority of the positron electron annihilation events by using the scintillation property of the liquid argon.
In the case of the \Kr decay into the excited \nuc{Rb}{85m} the emitted electron has a very low mean kinetic energy of E\(_{mean}=47.65\)keV.
This means that in the majority of these decays no light should be seen in the detectors, because the beta electron is unlikely to trigger any of them.
In contrast to that you can expect a very strong light signal every time a positron electron annihilation occurs. 
One should therefor be able to filter out almost all of the annihilation events while keeping the majority of the \Kr decay by only using events where this flag is not triggered.
This chapter tries to implement a filter that uses this mechanism and discusses in the end how successful the repression really was.
\\

Among other things, each event in tier 4 was given a flag called "isLArVetoed".
This flag is always triggered whenever an event in the Germanium detectors coincides with a scintillation signal of about 0.5phe in one of the photomultipliers positioned in the liquid argon tank \cite{agostini_allardt_bakalyarov_barabanov_baudis_bauer_bellotti_belogurov_belyaev_benato_et al._2017}.
If one plots all events in which this flag has not been set one gets figure \ref{fig:LArBEGes} for the BEGes and figure \ref{fig:LArCOAX} for the COAX detectors.
You can see that the positron electron annihilation peak can not be identified any longer while the 514keV peak seems almost unchanged.
\\

\begin{figure}[t!]
\centering
\begin{subfigure}{.5\textwidth}
  \centering
	\includegraphics[width=80mm]{./Bilder/500525LArVetoBEGes.pdf}
    \caption{BEGes}
  \label{fig:LArBEGes}
\end{subfigure}%
\begin{subfigure}{.5\textwidth}
  \centering
	\includegraphics[width=80mm]{./Bilder/500525LArVetoCOAX.pdf}
  \caption{COAX}
  \label{fig:LArCOAX}
\end{subfigure}
    \\
	\vspace{0.5cm}
    \caption{All events measured by the respective detectors with the LAr filter applied in the range of 500keV to 525keV.}
\end{figure}

In comparison, when you take only those events in which the liquid argon veto was triggered you get figure \ref{} and \ref{}.
From those you can see that the majority of filtered events have an energy around the 511keV. 


At this point one could say that we have done enough filtering and use the liquid argon filtered spectra to determine the number of decays in the peak.
But due to the non vanishing possibility of the beta electron to trigger the liquid argon veto it is of interest to investigate whether the filter has accidentally filtered out any real rare \Kr decay and whether it is possible to recover them.
The number of events filtered out by the liquid argon veto in the energy range of 509 to 519 keV is 1728.
This number is way too big to look at every individual case.
\\

Luckily we can use the fact that the excited \nuc{Rb}{85} state has a half life of 1.015 \(\mathrm{\mu}\)s.
This means that theoretically we should be able to measure a pre-coincedence event of the beta creating scintillation light which can then be measured before even a signal in the germanium detectors could be seen.
This way we  can hopefully identify the majority of events created by \Kr decays from the rest of the filtered signals.
To do this, we have to limit the events used for our investigation to those events that have a negative time difference between these two events respective to the Germanium detector signal.
The time difference for each individual liquid argon events is already analyzed and stored in the vector "triggerLAr"  of the Tier 3 data set.
This vector has the same number of dimensions as there are photomultipliers and every entry is indexed with the corresponding input channel of this photomultiplier.
The entries of this vector are again vectors themselves storing the time difference for every signal that triggered the liquid argon veto in the corresponding channel.
Since the events are listed in ascending order and we are interested in the first trigger event of the photomultiplier, we use only the first entry of this inner vector here for simplicity reasons.
From now on we only use events in which at least one photomultiplier measured a negative time difference.
\\

In addition, we also know that the energy of the released beta electron is relatively low.
This means that we can expect that only a small number of photomultipliers should measure a signal in a real rare \Kr decay.
In this case we will only use those events that have a maximum of four different triggered photomultipliers.
Allowing events with maximum of four different triggered photomultipliers is already a lot for a \Kr decay. 
But we wanted to do a more detailed manually investigation later anyways which is why the filter can be a little coarser here.
\\

\begin{figure}[t!]
	\centering
	\ifmakefigures%
	\includegraphics[width=100mm]{./Bilder/TriggerTimeOnly4.pdf}
	\fi%
	\label{fig:Trigger4}
	\caption{
		All liquid argon filtered events with a negative time difference between the event in the Germanium detector and a signal in the photomultipliers in the liquid argon tank.
	}
\end{figure}

If you apply these two restrictions to the liquid argon filtered events and only take those photomuliplier signals with a signal strength of at least 0.5 phe you get a distributions as seen in figure \ref{fig:Trigger4}.
We are only interested in the signals that measured at least 0.5 phe because, as mentioned above, that is the necessary intensity to trigger the LAr veto making them the signals of interest. 
The x axis is the time difference of the events in the photomultipliers from the signal measured in one of the Germanium detectors.
Theoretically it should now be possible to see a exponential increase from the negative scale towards a vanishing time difference.
Because of the small number of events however, it is almost impossible to make any statements about the course of these events.
\\


!!!!! diesen Teil noch Überarbeiten wenn nicht weglassen !!!!!

Interestingly, the majority of all signals with a negative time difference were measured by the PMTs. 
This could perhaps be due to the better temporal resolution of the PMTs. 
The signals of all PMTs are recorded with a resolution of 10ns and the SiPM with a resolution of 80ns but over a larger range than the PMTs\cite{nature}. 
However, since the expected time difference is in the order of \(\mu\)s, this is no reason for the reduced number of measured events. 
Another difference between the photomultiplier types is that the PMTs have a faster reaction time than the SiPMs. 
As we will see, practically all deflections in the measured intensity of the PMT signal caused by procoincedents have already subsided before the signal was measured in the germanium detectors. 
Thus a clear maximum should be recognizable before the germanium detector events.
In contrast, the measured signal of the SiPMs increases exponentially with a characteristic time in the order of 10\(\mu\)s. 
The trigger positions are then determined by software. 
However, this should still be able to determine the trigger position of the signals for procoincedence events with sufficient accuracy.
!!!!! Hier dir noch was genaueres ausdenken, wenn überhaupt !!!!!
\\

!!!!! bis hier hin !!!!!


Nevertheless we were able to lower the number of potential 514keV photons from \Kr decay events from 1728 down to only 55.  
These remaining events can now be manually examined with a with software called GerLa written by GERDA employees.
This tool allows one to search for a specific event and see all recorded signals of the Germanium detectors and the photomultipliers around this event.
\\

With this program one can now perform a manual filter by looking at the remaining 55 events individually and perform a specific procedure to determine whether they are signals from a \Kr decay or not.
This procedure consists of filtering out every event that has a combined light intensity of over 5 phe and every event that only has a negative time difference in a signal weaker than 0.5 phe.
The upper limit of the light intensity comes from the fact that the mean kinetic energy of the electron released from the decay is 47.65keV. 
With the luminosity of liquid argon of about 50 photons/MeV one can expect about 2.5 photons at these energies. 
It can then be assumed that the predominant part of all betas released in the investigated decay only emit 5 photons at best.
\\

\begin{figure}[t!]
	\centering
	\ifmakefigures%
	\includegraphics[width=100mm]{./Bilder/BeispielSignal.pdf}
	\fi%
	\label{fig:Trigger4}
	\caption{
    The recorded signal of the photomultiplier tube P4 from the event with event number 1614036. !!!!!Hier noch was zu den Achsen!!!!!. 
    The blue bar represents the moment in time in which a signal in the Germanium detector was measured.
	}
\end{figure}

At this point we realized that this procedure does not really work as well as we hoped it would.
From the remaining 55 only a few events really were unambiguous enough that one could clearly claim that they are the photons we expected.
After this much filtering we can assume that probably all of these events are in fact caused by a \Kr decays.
Signals like the one shown in figure \ref{fig:Trigger4} are a rare example of an almost model signal that we wanted to see.
The great majority of other events were either background events in the SiPM that randomly triggered the LAr veto or their measured signal was too strong.  
\\

With this it was now determined that the majority of events with a negative time difference are in fact not caused by a \Kr decay.
This means that the remaining \Kr decay events must have a positive time difference.
But it is practically impossible to recover them from there due to the great amount of individual events that have to be looked at.
Our recovery attempt has therefor failed.
\\

!!!!! DAS HIER IST AUCH NOCH EINE PROBLEMSTELLE !!!!!

Nevertheless we are still able to make some qualitative estimations about why approach did not work. 
When trying to make a more realistic estimation one can see qualitatively that the actual portion of recovered events must be smaller.
An important factor limiting this recovery rate originates from an estimation applied in this approach.
It was approximated that the electrons created scintillation light instantaneously after being released.
In reality, the moment an electron emits scintillation light is determined by many complex mechanisms. 
However, its behavior can be approximated by a relaxation approach. 
This gives you a new characteristic time with which you can describe how many electrons have already created scintillation light after what time. 
But this not spontaneous behavior means, that the measured signal depends on both the relaxation time of the \nuc{Rb}{85} and the scintillation light of the electron.
It is therefor quite possible, that a number of \Kr decay events can have a positive effective time difference.
Those are then lost in our analysis due to us filtering out every event that has a positive time difference.
This results in a lower recovery rate.
\\

Now that we have seen that we were not able to recover any of the falsely filtered \Kr caused events it is justified to also question whether the majority of the rare \Kr decays can even pass through the liquid argon filter itself.
But we only get a satisfactory answer with a quantitative analysis which will be performed in the following chapter.
\\

\section{Fitting}
\label{sec:Fitting}

Our strategy to calculate the activity of \Kr through the 514keV peak involves determining the absolute amount of measured \Kr decays events over all of \PII.
Due to the still enormous amount of events from other sources that can not be suppressed, for example the \nuc{Ge}{67} decays, it is necessary to fit the 514keV peak with a suitable fit function.
The amount of measured \Kr decay events can then be calculated by integrating over the found fit function and dividing by the probability of the specific decay into \nuc{Rb}{85m}.
\\

The not liquid argon filtered spectrum \ref{fig:NoFilterBEGes} and \ref{fig:NoFilterCOAX} show two peaks.
This requires the fit function to include two Gaussian functions from which only the parameters of the second peak will then be analyzed.
Additionally to the two Gaussian peaks a constant and an exponential background function will be added.
Theoretically you can expect the distribution of the germanium events to behave like its phase-space function and therefor change with energy.
In this case however is the inspected energy interval small compared to the complete phase-space diagram of \nuc{Ge}{67}.
This allows the approximation of the background to change like an exponential .
The resulting fit function is shown in equation \ref{equ:FitNoFilters}.
\\

\begin{equation}
\mathrm{f}(x) = \mathrm{A}\frac{1}{\sqrt{2\pi}\mathrm{C}}\exp\left(-\frac{(x-\mathrm{B})^2}{2\mathrm{C}^2}\right) + \mathrm{D}\frac{1}{\sqrt{2\pi}\mathrm{F}}\exp\left(-\frac{(x-\mathrm{E})^2}{2\mathrm{F}^2}\right) + \mathrm{G}\exp\left(\mathrm{H}x\right) + \mathrm{I}
\label{equ:FitNoFilters}
\end{equation}
\\


In the course of the fitting process it has to be mentioned that some fitness parameters have only been left free for rather small interval.
Most notably the values C and F, being the variances of two Gaussian peaks, were each left free only on a very small range around the values derived from the resolutions of the detectors at the specific energies (see appendix section \ref{sec:ResDetermination}).
When you apply the fit function to the two different spectra of the two kinds of detectors you get the graphs displayed in figure \ref{fig:FitNoFilterBEGes} and \ref{fig:FitNoFilterCOAX}.
The resulting fit parameters are listed in table \ref{tab:FitParNoFilter}. 
\\


\begin{figure}[t!]
	\centering
	\begin{subfigure}{.5\textwidth}
		\centering
		\includegraphics[width=80mm]{./Bilder/500525FitNoFilterBEGes.pdf}
		\caption{BEGes}
		\label{fig:FitNoFilterBEGes}
	\end{subfigure}%
	\begin{subfigure}{.5\textwidth}
		\centering
		\includegraphics[width=80mm]{./Bilder/500525FitNoFilterCOAX.pdf}
		\caption{COAX}
		\label{fig:FitNoFilterCOAX}
	\end{subfigure}
    \\
	\vspace{0.5cm}
	\caption{All events measured by the respective detectors in the range of 500keV to 525keV fitted with a fit function in the form seen in equation \ref{equ:FitNoFilters}. The green function represent the two Gaussian peaks independently using the determined fit parameters.}
\vspace{0.5cm}
\end{figure}
\\

\begin{figure}[t!]
\centering
\begin{tabular}{|c|c|c|}
\hline
Name	& Value [BEGes] & Valuse [COAX]\\ 
\hline
A  &	(16.15 \(\pm\)	3.658)&	(29.842 \(\pm\)	3.968)	\\	
\hline
B  &	(510.693 \(\pm\)	0.564)&	(510.750 \(\pm\)	0.265)\\	
\hline
C  &	(1.022 \(\pm\)	1.433)	&	(1.564 \(\pm\)	0.236	)	\\
\hline
D  &	(27.239 \(\pm\)	3.870)	&	(20.463 \(\pm\)	3.279	)	\\
\hline
E  &	(513.940 \(\pm\)	0.701)	&	(513.858 \(\pm\)	0.213)	\\
\hline
F  &	(0.909 \(\pm\)	1.400)	&	(1.017 \(\pm\)	0.173	)	\\
\hline
G  &	(14.175 \(\pm\)	1.050)	&	(13.123 \(\pm\)	0.289	)	\\
\hline
H  &	(24.487 \(\pm\)	1.833)	&	(-384995.062 \(\pm\)	2.592)	\\
\hline
I  &	(-40.573 \(\pm\) 1.000)	&	(-40.573 \(\pm\)	1.414)\\
\hline

\end{tabular}
\label{tab:FitParNoFilter}
\captionof{table}[]{Fit parameters of fit function \ref{equ:FitNoFilters} applied on the spectra of the respective detectors.}
\end{figure}
\\

The only values of these fit parameters that is of real interest for the determination of the activity is variable D.
Its value is the amplitude of the second Gaussian peak.
Because the Gaussian peak was chosen in the normalized form this value also represents the amount of \Kr decays measured per binning of the histogram.
But we only want the amount of counts in the peak independently of the binning of the histogram.
We therefor have to multiply these values with $}\frac{1}{0.2}\unit{keV}$.
Finally, we can conclude that in the BEGes an amount of !!!!! events was measured and in the COAX detectors a number !!!!! of events.
\\

Now to the liquid argon filtered spectrum.
As mentioned above we can assumed that the positron electron annihilation peak is fully suppressed.
Additionally in the ideal case, that the majority of all \Kr decay events got through the liquid argon filter, we expect the amplitude of the Gaussian peak to be similar to the amplitude in the not liquid argon filtered case.
Therefor only one Gaussian peak has to be fitted together with the background function.
The fit results in the function displayed in equation \ref{equ:FitFilters}.
\\ 

\begin{equation}
\mathrm{f}(x) = \mathrm{A}\frac{1}{\sqrt{2\pi\mathrm{C}}}\exp\left(-\frac{(x-\mathrm{B})^2}{2\mathrm{C}^2}\right) + \mathrm{D}\exp\left(\mathrm{E}x\right) + \mathrm{F}
\label{equ:FitFilters}
\end{equation}
\\

After applying the fit to the spectra seen in figure \ref to \ref, you get the fit parameters displayed in table \ref.
As above, the amplitude A of the Gaussian peak correspond to the number of events measured in the area of the peak per binning.
This results in an amount of !!!! for the BEGe and !!!!! for the COAX spectrum.
As mentioned above these values are only planned to be a crosscheck for the not argon filtered values. 
Compared to those values, it can be seen that the number of events in the BEGe and in the COAX detectors has dropped considerably.
This means that some of the \Kr decay events must have been filtered out and a recovery of these falsely filtered \Kr decays is necessary.
But due to that not being possible we can not use those results in our analysis nor for a real crosscheck..
\\

\begin{figure}[t!]
	\centering
	\begin{subfigure}{.5\textwidth}
		\centering
		\includegraphics[width=80mm]{./Bilder/500525FitLArVetoBEGes.pdf}
		\caption{BEGes}
		\label{fig:FitLArVetoBEGes}
	\end{subfigure}%
	\begin{subfigure}{.5\textwidth}
		\centering
		\includegraphics[width=80mm]{./Bilder/500525FitLArVetoCOAX.pdf}
		\caption{COAX}
		\label{fig:FitLArVetoCOAX}
	\end{subfigure}
	\cpation{All events measured by the respective detectors  with the liquid argon veto applied in the range of 500keV to 525keV fitted with a fit function in the form seen in equation \ref{equ:FitFilters}.}
\end{figure}

\begin{figure}[t!]
	\centering
	\begin{tabular}{|l|r|r|r|r|}
		\hline
		Name	& Value [BEGes] \\ 
		\hline
		A  &	(17.844639 \(\pm\)	3.179471)&	(17.914476 \(\pm\)	2.683260)	&	(19.847330\(\pm\)	2.688415)&	(18.851511 \(\pm\)	2.696000)\\	
		\hline
		B  &	(513.838989 \(\pm\)	0.665171)&	(513.748535 \(\pm\)	0.179167)&	(513.849854 \(\pm\)	0.152052)&	(513.737183	\(\pm\) 0.167941)\\	
		\hline
		C  &	(0.828315 \(\pm\)	1.377836)	&	(0.940856 \(\pm\)	0.160788	)	&	(0.865958\(\pm\) 0.108436)&	(0.923679 \(\pm\)	0.149867)\\
		\hline
		D  &	(9.265341 \(\pm\)	1.032088)	&	(9.073170 \(\pm\)	0.233725	)	&	(9.307535\(\pm\)	0.236841)&	(9.076473 \(\pm\)	0.233940)\\
		\hline
		E  &	(18.142975 \(\pm\)	1.833031)	&	(-401772.843750 \(\pm\)	3.666062)	&	(-394189.968750\(\pm\)	45.660984)&	(-796827.062500 \(\pm\)	64.574379)\\
		\hline	
		F  &	(-26.647284 \(\pm\)	1.000000)	&	(-43.877174 \(\pm\)	1.414214	)	&	(-55.546192\(\pm\)	1.414214)&	(-55.546192 \(\pm\)	1.414214)\\
		\hline
	\end{tabular}
	\label{tab:FitParNoFilter}
	\captionof{table}[]{Fit parameters of fit function \ref{equ:FitNoFilters} applied on the spectra of the respective detectors.}
\end{figure}


Now that we have a number of the special \Kr events measured in the respective detectors.
But what we need for the determination of the activity is not the amount of measured events but actually the amount of \Kr decays in the whole liquid argon tank.
Luckily we can calculate the amount of actual events in the liquid argon tank by determining the detector efficiency of the two detector types.
This can be done by running a Monte Carlo simulation simulating a great amount of 514keV photon emissions in a cylindrical volume with a detector therein with the same design as used in GERDA.
From the amount of measured events in these simulated detectors and the absolute amount of simulated events one can then calculate the efficiency with which the detectors measure any \Kr events. 
How exactly this was implemented is the topic of the next chapter.
\\

\section{Monte Carlo Simulation}
\label{sec:MonteCarlo514}

As described above, we want to determine a conversion factor between the measured 514keV events and the decay density of \Kr necessary to create this signal.
Such a conversion factor can be determined with the help of a Monte Carlo simulation.
The tool used to perform this simulation is \mage (MAjorana-GErda), a GEANT4-based physics simulation software developed jointly by MAJORANA and GERDA \cite{boswell_melissa_chan_yuen-dat_detwiler_a._finnerty_padraic_henning_gehman_et}.
Both experiments aim to measure the neutrinoless double beta decay using enriched Germanium detectors.
Because it is expected that the halflife of this decay is in the order of at least \(10^{25}\) years, a lot of effort is put into finding out how big the contribution of different isotopes to the background is.
\mage is therefor specialized to simulate radioactive decays and their corresponding measured events in Germanium detectors.
\\

The simulation used here consists of a cylindrical of 2.5m of height and 3m in diameter, in which the detector structure of the GERDA experiment is placed in the middle.
The resulting volume of the liquid argon is $V_{sim} = 17.65 \mathrm{m}^3$ .
Compared with the volume of 64 m\(^3\) of liquid argon used in the GERDA experiment this volume is much smaller.
But as we will see later, this volume is by far big enough for our purpose.
A total of N = 50.000.000 rare \Kr decays are now simulated in the volume of this cylinder that is not occupied by the detectors. 
As can be seen in figure \ref{fig:CrossSecAb}, the density of the decays over the entire volume of the cylinder is relatively constant.
\\

\begin{figure}[t!]
	\centering
	\begin{subfigure}{.5\textwidth}
		\centering
		\includegraphics[height=80mm]{./Bilder/MC-Querschnitt-BEGes.pdf}
		\caption{Cross section from above}
		\label{fig:CrossSecAb}
	\end{subfigure}%
	\begin{subfigure}{.5\textwidth}
		\centering
		\includegraphics[height=80mm]{./Bilder/MC-Radius-BEGes.png}
		\caption{Radial cross section}
		\label{fig:CrossSecRa}
	\end{subfigure}
    \\
	\vspace{0.5cm}
    \caption{Cross section from above and a radial cross section showing the density of simulated decays in the cylindrical volume. The red points indicate all events that were measured by the BEGe detectors.}
\vspace{0.5cm}
\end{figure}
\\

From these 50 million simulated decays only about 90 thousand were actually detected by any of the detectors, only 30.465 of them crated a signal in one of the BEGe and 24902 in the COAX under the condition that one already uses an anti-coincedence filter.
The spatial distribution of all measured events in the respective detectors can be seen as red dots in Figure \ref{fig:CrossSecAb} and \ref{fig:CrossSecRa} for the BEGe detectors and Figure blah and blah for the COAX detectors, depending on the detector type.
From their position one can see that in both detector types the overwhelming majority of the detected events were positioned close to the detectors themselves.
Already at a distance of about 60cm from the detectors the majority of all decays were no longer measured.
This means that the amount of measured events should not change when one enlarges the volume of the tank, provided that the density of decays stays constant.
But that would also mean that the detector efficiency $\epsilon$ would have a reciprocal proportionality to the simulated volume.
Therefore, we expect the ratio $\frac{1}{\epsilon V_{sim}}$ to be invariant with change of volume, provided that the volume is large enough.   
This is the reason why we are able to use a smaller volume of liquid argon in the Monte Carlo simulation than was actually used in the experiment.
At the same time, this ratio is also a conversion factor between the amount of measured 514keV photons and the density of \Kr necessary to create the measured peak, which is exactly what we want to determine with this simulation.
\\

The spectrum of all the events detected in BEGe detectors is shown in figure \ref{fig:PhasenraumMC514}.
From it one can be seen that only a small amount of all measured events are direct signals from a rare \Kr decay. 
The great majority of photons measured were scattered before they arrived in the detector and therefor have a lower energy.
Among other things, the Compton edge of the photons at about 343 keV can be seen.
But to calculate the detector efficiency, only the measured events at the 514keV peak have to be used.
In the case of the BEGe detectors this peak contains a total of \(\Delta\mathrm{N} = 4511\pm67\) events while the COAX peak contains  \(\Delta\mathrm{N} = 3706\pm60\) .
With a total of 50 million initial decays, this results in a efficiency of 
\begin{equation*}
\epsilon_{\gamma\mathrm{,BEGe}} = \frac{\Delta\mathrm{N_{BEGe}}}{\mathrm{N}} = (9.02\pm0.13) \times 10^{-5}  \frac{\mathrm{event}}{\mathrm{decay(rare)}}
\end{equation*}
\begin{equation*}
\epsilon_{\gamma\mathrm{,COAX}} = \frac{\Delta\mathrm{N_{COAX}}}{\mathrm{N}} = (7.412\pm0.12) \times 10^{-5}  \frac{\mathrm{event}}{\mathrm{decay(rare)}}
\end{equation*}
for the volume of the simulated cylinder.
\\

This means that if a 514keV photon is emitted at any location in the liquid argon container, it has a probability \(\epsilon_{\gamma,\mathrm{,BEGe}}\) of being measured by one of the BEGe detectors.
On the other hand, it can also be said, that for every measured 514keV photon in one of the BEGe detectors an amount of about $\frac{1}{\epsilon_\gamma} = 10515 \frac{\mathrm{decay(rare)}}{\mathrm{event}}$ \nuc{Rb}[85m] relaxations must occur.
In other words, the value $\frac{1}{\epsilon_\gamma}$ is a factor from the measured entries to the actual amount of \nuc{Rb}{85m} relaxations.
If you divide this factor by the simulated volume you get the conversion factor you want to determine with this simulation.

\begin{equation*}
\frac{1}{\epsilon_{\gamma\mathrm{,BEGe}} V_{sim}} = (5.955\pm0.086) \times 10^{-1} \frac{\mathrm{decay(rare)}}{\mathrm{event} \times l}
\end{equation*}
\begin{equation*}
\frac{1}{\epsilon_{\gamma\mathrm{,COAX}} V_{sim}} = (5.955\pm0.086) \times 10^{-1} \frac{\mathrm{decay(rare)}}{\mathrm{event} \times l}
\end{equation*}

As mentioned above these values are invariant with change of volume which is why we can also apply it to our real liquid argon tank volume. 
\\

\begin{figure}[t!]
	\centering
	\ifmakefigures%
	\includegraphics[width=100mm]{./Bilder/MC-514-Phasenraum.pdf}
	\fi%
	\caption{
    Spectrum of the measured rare \Kr events in all of the BEGe detectors of the Monte Carlo simulation.
	}
	\label{fig:PhasenraumMC514}
\end{figure}

Now that the conversion factors of the two types of detectors are determined we can calculate the density of all \Kr decays by applying formula \ref{equ:density}

\begin{equation}
\rho_{dec} = \frac{\mathrm{N}}{p}\times\frac{1}{\epsilon_\gamma V_{sim}}
\label{equ:density}
\end{equation}

where p = 0.434$\%$ is the probability of a \Kr to decay into the 514keV energetically raised \nuc{Rb}{85m}.
For an amount of 27.24\(\pm\)3.87 events to be measured in the BEGe detectors, it can be concluded that a density for the \Kr decay of, whereas for the COAX detectors we get a value of . 
\\

!!!!! hier könnte man vielleicht noch einen Absatz zur Güte der MC simulation mit Hilfe der relativen Zählrate beim 514keV peak in der beiden Detektoren hinschreiben !!!!!
\\

\section{Calculating the Activity}
\label{sec:CalcActiv}

All we need now to calculate the mean activity of \Kr in \PII is the measuring time of all detectors.
But this is not as easy as it seems.
Firstly we have the problem that not all detectors were measuring over the curse of \PII and that there were time intervals in which no measurement was recorded at all.
This can easily be solved by looking at the effective measuring time of each detector individually.
The effective measuring time of a detector can easily be determined by looking at how many test pule signals have been recorded by it. 
Since the test pulse signals have been set to a frequency of 0.05Hz over the entire \PII, an effective measurement time can be achieved by determining the number of detected test pulse signals and then multiply it by 20 seconds.
The individual measuring times are therefor given by
\begin{equation*}
    t_\mathrm{i} = \mathrm{N}_{TP}(\mathrm{i}) \times 20\mathrm{s}
\end{equation*}
where i is the index of the respective input channel of each BEGe detector.
\\

The second problem arises from the fact that we calculated the decay density with the assumption that we could merge all detectors of the same kind into a single detector.
But now that we want to look at the measurement time of each individual detector, this assumption does not hold true.
There are now two different workarounds we could take.
\\

In the first method we would rerun our Monte Carlo simulation, this time considering the amount of time each detector was actually measuring.
This would be very inefficient.
\\

The second method works around the merging problem by calculating an average measurement time for all detectors.
With this you could then calculate with the detector block as if all detectors of one kind were such a single detector.
This is very elegant solution because no new computation of a simulation is necessary.
\\

For this average measuring time one has to consider the fact that we also have to apply a weigh on every individual measuring time.
This arises from the fact that every BEGe detector has an individual mass.
As we know, the detector efficiency of a single detector is directly dependent on its mass.
If we want to combine all single detectors into one large detector, we have to consider that heavier detectors contribute more to the measured decay rate than lighter detectors.
It is therefore necessary to weight the measuring time of each detector with its individual mass.
\\

Coincidentally, the multiplication of the individual measuring time of a detector with its mass is also the exposure this  single detector.
This means that to calculate the mean measuring time, we just have to divide the combined exposure of all detectors of the same kind by their combined mass.
\\

No matter what approach you take, in the end you get a mean measuring time of !!!!! for the BEGes and !!!!! for the COAX detectors.

\begin{equation*}
    \bar{t} = \frac{\sum_\mathrm{i} t_\mathrm{i} \times m_\mathrm{i}}{\sum_\mathrm{i} m_\mathrm{i}} = (1.592\pm0,001)
    \mathrm{y}
\end{equation*}
\\

With these mean measuring times we can now finally calculate the mean specific activity $\bar{a}$ of \Kr over the curse of all of \PII by applying
\begin{equation*}
    \bar{a} = \frac{\mathrm{N}}{p}\times\frac{1}{\epsilon_\gamma V_{sim}}\times\frac{1}{\bar{t}}
     \frac{\mathmr{mBq}}{\mathrm{l}}
\end{equation*}



Finally, we can conclude that with the line count rate analysis we were able to determine an activity of 

!!!!! sich noch überlegen, was man genau mit den COAX detektoren macht !!!!!

% calculate Amplitude of Gauss peak at 514keV and use factor from Monte Carlo Simulation to estimate

% look at phase diagram at range of 500 to 525 keV, use different filters and fit remaining data with Gaussian function
% -> get amplitude
% make a Monte Carlo simulation to estimate actual Kr85 activity in LAr from measured activity in detectors
% -> with amplitude and factors from MC-Simulation one can calculate the specific activity

\section{Activity from the Decrease in Rate}
\label{sec:SAfromDecrease}


As described in the beginning of section \ref{sec:AotKr}, the line count analysis method of the rare \Kr decay was expected to be relatively precise.
By following this procedure we were able to determine the activity down to the order of 10$^{-5} \frac{\mathrm{Bq}}{\mathrm{l}}$ with reasonable uncertainty.
As it was also stated above, the second method using the change in intensity with time is expected to be relatively imprecise.
This estimation came from the fact that this method relies on a great approximation.
It is necessary to expect that in the time interval of all of \PII only the intensity of the \Kr isotope has a non neglectable change in its intensity.
The idea behind this is that \Kr has the lowest half life \(T = 10.739\\unit{y}\) of all the other residual radioactive isotopes in the liquid argon.
The dominant background sources are in descending order \nuc{Ar}{42} with a half-life of 32.9 y, \nuc{Ar}{39} with 269 y and \nuc{U}{232} with 68.9 y. 
Compared to those should \Kr be the only isotope to change its activity notably in the timespan of Phase II.
It is therefor approximated that all change in intensity should originate from the \Kr decay.
\\

Now that we have made this approximation, we have to discuss how one can determine the specific activity using the change in intensity. 
To measure the change of intensity over time every measured event will be drawn over time and an exponential fit applied.
The amplitude of the resulting exponential function represents an event rate with which events in the investigated energy range occur.
To calculate the specific activity of \Kr with this value one again has to use a Monte Carlo simulation to calculate a fitting conversion factor. 
We cannot use the same simulation from the first method because in this case we actually have to simulate actual \Kr events, not only the emission of 514keV photons as it was done in the prior chapter.
That is why another Monte Carlo simulation has to be used, just now with regular \Kr decays.
With it another converstion factor can be determined.
Finally the specific activity can be calculated from these two values.
It is important to note that in this approach determines the specific activity of \Kr at the start of \PII iwhile the first method calculates the mean specific activity over all of \PII.
To compare those two results, the mean specific activity of the second method has to be determined.
\\

The only reason why we can use the second method is that any other radioactive isotope present in liquid argon has a half-life much higher than that of \Kr \(T = 10.739\\unit{y}\).
The dominant background sources are in descending order \nuc{Ar}{42} with a half-life of 32.9 y, \nuc{Ar}{39} with 269 y and \nuc{U}{232} with 68.9 y. 
Therefor it can be expected that no other isotope has a notable change in time over the course of \PII.
\\

This procedure will be carried out in the next section.
\\

\section{Event Activity}
\label{sec:EventAct}

\begin{figure}[t!]
	\centering
	\begin{subfigure}{.5\textwidth}
		\centering
		\includegraphics[height=80mm]{./Bilder/ZeitverlaufALLE.pdf}
		\caption{every event}
		\label{fig:ZeitAll}
	\end{subfigure}%
	\begin{subfigure}{.5\textwidth}
		\centering
		\includegraphics[height=80mm]{./Bilder/ZeitverlaufLimits.pdf}
		\caption{only energies between 200 and 400keV}
		\label{fig:ZeitLimits}
	\end{subfigure}
    \\
	\vspace{0.5cm}
    \caption{Change of Intensity with time. In both figures are the amount of events measured in one week plotted over the whole time of \PII. In figure (a) no filters were imposed onto the displayed events while in (b) only those events are shown that have an energy between 200 and 400 keV. One can see that further precautions must be taken before an exponential decrease can be determined. }
\vspace{0.5cm}
\end{figure}
\\


To determine the event activity of the all events measured in the detectors one has to plot the amount of measured events in specific intervals over time.
If one takes every event measured over all of Phase II one can see two key characteristic discrepancies from the expected continuous course (see figure \ref{fig:ZeitAll}).
\\

Firstly, you see a jump in the intensity at roughly \(52\times10^6\unit{s}\).
This jump results from lowering of the lower detection energy limit. !!!!! Hier noch genauer nachfragen !!!!!
Because of that more events were actually stored then previously resulting in a higher event intensity.
This however can easily be solved by only using events with a minimum energy of about 200keV.
We can also use an upper energy limit to suppress events caused by isotopes with a higher end point energy than \Kr.
Even though they should only create a constant background, but their presence make a change in intensity harder to identify.
We therefor chose the upper limit to be at 400keV.
Theoretically the endpoint energy of \Kr is at 687keV, but the count rate towards the endpoint is that low that we would basically only include more constant background by setting the upper limit higher.
A new plot including only events between the energy limits of 200 and 400keV can be seen in figure \ref{fig:ZeitLimits}.
\\

Secondly, you have here the same problem as before.
Not all detectors were on all of the time and there were also time intervals in which nothing at all was measured.
In section \ref{sec:CalcActiv} we used the test pulse signal to weigh each individual detector signals with which one we were able to calculate a weighted mean measuring time.
However here we are not interested in a mean measuring time over all of \PII.
In this case solve this issue by only using the events measured by all those detectors that were on over all of \PII and 


would get a lot of constant background from other radioactive isotopes with a much higher half life than \Kr.
This constant background can be partially suppressed by limiting the events examined.


An appropriate limitation is to only use those events with energies in the range of 200 to 400keV.
With this choice a lot of isotopes with a much higher mean beta energy will be filtered out while a great portion of the \Kr events lie in this interval.
Really the only two other isotopes that have their mean beta energy in the same area are \nuc{Ar}{39} and \nuc{Ar}{42}.
\nuc{Ar}{39} has a half life of about 269y and can therefor be seen as a constant background source.
\nuc{Ar}{42} however has only a half life of 32.9y 
%The only other major background source that also has its maximum energy in this range is \nuc{Ar}{39} but its half life is of about 269y which means that its change over time should is irrelevant.
Theoretically the ideal energy range for \Kr would already start at 50keV due to its maximum energy of about 100keV.
But this is problematic due to  !!!!! hier noch hin warum sich das Spektrum bei e<150 so seltsam verhält !!!!!.
\\



If you now plot the measured events over time, a new problem becomes apparent (see figure ).
In figure \ref{} one can see that the intensity changes very abruptly over time not following any kind of decrease.
But for our intensity diagram to have a continuous



This can easily be explained by what was already mentioned in \ref{sec:CalcActiv}.
We saw there that not every detector was active all the time and that in \PII there were time intervals at which nothing at all was measured due to maintenance work.
Luckily we already found a solution to this problem by using the test pule signal.
With it we were able to calculate an effective measurement time for every detector.
\\

In the case however we are not interested in an overall effective measurement time.
Rather, we are interested in the measurement time of each detector in each of the individual binning intervals of the histogram.
Those intervals were chosen to be of roughly one week.
Out strategy therefor is to determine the amount N$_{det,t}(i)$ of events with energies from 200 to 400keV of each individual detector in these week long intervals as well as the amount N$_{TP,t}(i)$ of measured test pulse signals.
Variable i is the input channel of the detectors whereas t is the index for the specific time interval.
From those one can calculate a ratio r_{i,t} between the number of measured events and the test pulse signals for each detector and each time interval.
\begin{equation*}
r_{i,t} = \frac{\mathrm{N}_{det,t}(i)}{\mathrm{N}_{TP,t}(i)}
\end{equation*}  
These ratios represent a relative frequency in which the investigated events with 200 to 400 keV occurred compared to the test pulse signal.



\begin{figure}[t!]
	\centering
	\ifmakefigures%
	\includegraphics[width=100mm]{./Bilder/MC-514-Phasenraum.pdf}
	\fi%
	\caption{
    Spectrum of the measured rare \Kr events in all of the BEGe detectors of the Monte Carlo simulation.
	}
	\label{fig:PhasenraumMC514}
\end{figure}



% use fit to calculate the decrease in rate of the signals in range from 200 to 500keV
% from fit and assumption that only Kr85's activity is decreasing one can calculate the specific activity from the amplitude of the fit

% use Volume of LAr-Tank to determine the number of Kr85 and from this the concentration in the argon

\section{Conclusion and Discussion}
\label{sec:calcOfTheCon}

With all the work done in this paper it could now finally be shown that the activity of \nuc{Kr}{85} in the liquid argon of the GERDA experiment has a value of 0.074 $\frac{\unit{mBq}}{\unit{l}]}$. 



% what went wrong?
% what does it mean?
% possible future enhancements to determine the Kr85 concentration

































% %%%%%%%%%%%%%%%%%%%%%%%%%%%%%%%%%%%%%%%%%%%%%%%%%%%%%%%%%%%%%%%%%%%%%%%%%%%%%%%%
% \section{Another Section}
% \label{sec:secname}


% Fig.~\ref{fig:pmt} shows a PMT.

% %------------------------------------------------------------------- figure ----
% \begin{figure}[hb]
% \centering
% \ifmakefigures%
%    \includegraphics[width=45mm]{kapselung-small.jpg}
% \fi%
%   \caption{\label{fig:pmt}
% The encapsulation of the Cerenkov PMT.
% }
% \end{figure}
% %-------------------------------------------------------------------------------

% %%%%%%%%%%%%%%%%%%%%%%%%%%%%%%%%%%%%%%%%%%%%%%%%%%%%%%%%%%%%%%%%%%%%%%%%%%%%%%%%
% \section{Results and Analysis}
% \label{sec:results}


% %----------------------------------------------------------------- equation ----
% {\centering
% \begin{equation}\label{eq:sensit}
%     T_{1/2}(0^{+} \rightarrow g.s.~with~single~\gamma)~ \geq ~\ln2 \cdot
%     \varepsilon \cdot a \cdot \frac{M \cdot N_A}{A} \cdot 
%     \sqrt{\frac{\Delta T}{b\cdot\Delta E}},
% \end{equation}
% } % end centering
% %-------------------------------------------------------------------------------

% Table~\ref{tab:param} compiles everything.

% %-------------------------------------------------------------------- table ----
% \begin{table}[t]
% \centering
% \caption{\label{tab:param}
% Experimental parameters and values.
% }
% \vspace*{2mm}
% \begin{tabular}{L{4cm}|C{6cm}}
%   Column1 & Column2 \\  \hline 
%   Row1 & $100\pm10$ \\
%   Row2 & $100\pm10$ \\ \hline
%   Row3 & $100\pm10$ \\
%   Row4 & $100\pm10$ \\ 
% \end{tabular}
% \end{table}
% %-------------------------------------------------------------------------------

% %%%%%%%%%%%%%%%%%%%%%%%%%%%%%%%%%%%%%%%%%%%%%%%%%%%%%%%%%%%%%%%%%%%%%%%%%%%%%%%%
% \section{Conclusions}
% \label{sec:conclusions}
