	%%%%%%%%%%%%%%%%%%%%%%%%%%%%%%%%%%%%%%%%%%%%%%%%%%%%%%%%%%%%%%%%%%%%%%%%%%%%%%%%
% ---------------------------------------------------  ADD HERE YOUR REPORT  ---
%%%%%%%%%%%%%%%%%%%%%%%%%%%%%%%%%%%%%%%%%%%%%%%%%%%%%%%%%%%%%%%%%%%%%%%%%%%%%%%%

%sources:
% 0: https://arxiv.org/pdf/1006.1718v2.pdf





\section{Introduction}
\label{sec:intro}

The GERDA experiment tries to find evidence for the existence of the neutrino less double beta decay in \nuc{Ge}{67} as it was reported to be found in the Heidelberg-Moskau experiment. 
Because it is expected that this decay has a very long lifetime it is very important to filter out every source of background radiation as good as possible. 
Therefor it is necessary to use a low temperature coolant to freeze the \nuc{Ge}{67} detectors down and to use a scintillator around the detectors to filter out any radiation coming from the outside. 
A setup like this is chosen because in this experiment the Germanium detectors made out of the \nuc{Ge}{67} which is at the same time the wanted double beta decay source. 
This makes any signal coming from the outside a background event. 
\\

Liquid argon is a fitting material for both of requirements listed above due to its low freezing point and its ability to scintillate. 
Commercial argon is extracted from the air by air liquefaction and therefor has residual foreign isotopes. 
The majority of the impurity in the liquid argon can be removed by cryogenic distillation but even the remaining alien isotopes are still very active.\\

One of these residual radioactive isotopes is \nuc{Kr}{85}. 
Compared to \nuc{Ge}{67} it has a relatively low endpoint energy and therefor shouldn't create any fake neutrino less double beta decays. 
It also isn't the strongest radioactive background source. 
This title belongs to \nuc{Ar}{42}. 
What is interesting about this isotope is the fact that first approximations of its concentration in the liquid argon showed that it is at least 10\(^{-3}\) times smaller than values measured in other experiments using liquid argon. !!!!! hier die Abschätzung hin oder sich eine andere Begründung für meine Bachelorarbeit ausdenken !!!!!
\\

The aim of this theses is to determine the concentration of the \nuc{Kr}{85} in the liquid argon coolant and scintillator of the GERDA experiment and by this determining its influence to the radioactive background. 
This is accomplished by two different approaches. 
\\

The first method uses the relatively rare decay of \nuc{Kr}{85} into an excited state of \nuc{Rb}{85} with a probability of 0.43\%. 
This excited state relaxes to its ground state by emitting a photon with 514keV of which there should be a noticeable peak in the phase diagram. 
The overall activity of the \nuc{Kr}{85} can than be determined with the peak in the phase diagram. 
\\

The second approach takes a look at the overall intensity reduction over time and determines the activity of the \nuc{Kr}{85} from the amplitude of the exponential decrease. 
These two methods are supposed to complement each other showing the accuracy of the resulting concentration.\\

The first part of this theses concentrates on the physical background discussing the ideas behind the different neutrino fermion models and their consequences. 
It is also described what a double beta decay is and why it is hoped to find a neutrino less double beta decay. 
After that the usefulness of argon as a coolant will be discussed and lastly follows a characterization of the \nuc{Kr}{85}.\\

The GERDA experiment itself will be the focus of the second section. 
The aim of GERDA as well as its experimental setup and background reduction will be described and the current results discussed.\\

The third part is the main part of this theses. 
It will focus on how the concentration of \nuc{Kr}{85} will be determined. 
It will start with the theoretical idea behind and the problems of the two methods and then concentration on the two approaches individually !!!!! das sollte ich wohl am besten als letztes schreiben !!!!!

% what is the general aim of my Bachelor thesis
% general overview of how I'm going to do this
% ? what are my predictions ?
% What possible influence would the result of my work might have on the result of the GERDA experiment?

\section{Physical Background}
\label{sec:PhyBG}

\subsection{Weyl, Majorana and Dirac fermions}
\label{sec:WMDf}


Dieser Teil wird noch stark gekürzt bzw. ich werde mich hier wahrschenlich vor allem auf "Kerne und Teilchen", der KTA Script und ein paar einfürhende Paper stützen.

% in this chapter almost everything from source 0

\begin{itemize}
\item historic introduction of question whether neutrinos are dirac or majorana fermions
\begin{itemize}
\item from Dirac relativistic equation fermion fields
\item electrons: have mass and charge, Dirac-eq predicts antiparticles, requires 4-comp fields
\item Weyl calculates for massless fermions that only two-component fields are necessary
\item Pauli predicts neutrinos in letter, no charge, seems to have vanishing mass 
\item \(\rightarrow\) assumption: neutrinos  are Weyl fermions
\item Majorana: neutrinos are antiparticles of itself since they are uncharged, first not taken seriously, only after first indications that neutinos have mass
\item \(\rightarrow\) discussion whether neutrinos are Dirac or Majorana fermions
\end{itemize}
\item short introduction to dirac-eq
\begin{itemize}
\item \((i\hbar\gamma^\mu \partial_\mu  - mc)\psi = 0\) , maybe Hamilton/Lagragian, has plane waves as solution multiplied with spinor
\item Spinors: any column like function of energy and momentum which when multiplied by factor \(\exp(i\vec{p}\cdot\vec{x})\) or  \(\exp(\-i\vec{p}\cdot\vec{x})\) becomes a solution of the Dirac equation
\end{itemize}
\item we know that the Klein-Gordon-eq is real, how can we get a real solution from the dirac-eq
\begin{itemize}
\item depends on how we choose our \(\gamma^\mu\), if all non-zero elements of all the \(\gamma^\mu\) are purly imaginary, then Dirac-eq is real
\item Majorana matrices, with usage in Dirac-eq one can obtain real solutions that satisfy \(\tilde{\psi} = \tilde{\psi}^*\)
\item \(\rightarrow\) these solutions represent Majorana fermions
\item general solution to Majorana condition can be obtained by transformation with unitary matrix: \(\gamma^\mu = \mathrm{U}\tilde{\gamma^\mu}\mathrm{U}^\dagger\)
\item general Majorana condition: \(\psi = \mathrm{U}\mathrm{U}^\top\tilde{\psi}^*\), with \(\mathrm{U}\mathrm{U}^\top \equiv \gamma^0 \mathrm{C}\)
\item with compact notation \(\widehat{\psi} \equiv \gamma^0 \mathrm{C} \psi^*\)
\item general definition of a Majorana fermion fields through definition: \(\psi = \widehat{\psi}\), condition is Lorenz invariant
\end{itemize}
\item clunky repetition of what helicity and chirality is and what their problem with massive fermions is
\begin{itemize}
\item helicity defined as twice the value of the spin component of the particle along the direction of the momentum \(h_{\vec{p}} = \frac{2 \vec{J} \cdot \vec{p}}{p}\)
\item eigenstates of eigenvalue -1 called "left-helical", eigenstates of eigenvalue +1 called "right-helical"
\item is invariant under time/rotation, not invariant under boost
\item chirality meaning assigned to matrix \(\gamma_5 = i\gamma^0\gamma^1\gamma^2\gamma^3\)
\item projection matrices on fermion fields: \( \mathrm{L} = \frac{1}{2} \left( 1- \gamma_5\right ) \), \( \mathrm{R} = \frac{1}{2} \left( 1 + \gamma_5\right ) \)
\item projections of L, R are called lift/right-chiral
\item wavefunction can be written as \(\psi = \psi_L + \psi_R\) with \(\psi_L = \mathrm{L}\psi\) and \(\psi_R = \mathrm{R}\psi\)
\item \(\rightarrow\) helicity: conserved for free particles, not under Lorenz trafo
\item \(\rightarrow\) chirality: is Lorenz invariant, not conserved
\item \(\rightarrow\) both not appropriate for characterizing a fermion that has mass
\end{itemize}
\item how to define Weyl fermions
\begin{itemize}
\item problem with helicity/chirality disappears if the fermion is massless
\item general solution of Dirac-eq is not irreducible representation of Lorentz group (Lorenz group: group of Lorentz trafo in Mirkowski-space-time)
\item \(\rightarrow\) left/right-chiral fields are Lorentz invariant, representation with 2-component-field: \( \begin{pmatrix}\frac{1}{2} \\ 0\end{pmatrix}\) for left-chiral, \(\begin{pmatrix}0 \\ \frac{1}{2}\end{pmatrix}\) for right-chiral
\item when \(\chi\) is left chiral Weyl fermions, \( \widehat{\chi} \) is a right chiral Weyl fermions
\item a general fermion field can be described by two Weyl fields \(\rightarrow\) building blocks
\end{itemize}
\item how can we build Majorana/Dirac fermions from Weyl fermions
\begin{itemize}
\item Majorana and Dirac fermions both have mass and therefor must have both left/right chiral components
\item Dirac can be created by two independent left chiral Weyl fields \(\chi_1\), \(\chi_2\): \(\psi = \chi_1 + \widehat{\chi_2}\)
\item Dirac fermions are unconstrained solutions of the Dirac equation
\item unlike the Dirac fermions, the Majorana fermions must fulfill the reality condition \( \psi = \widehat{\psi} \)
\item Majorana fermions are represented by: \( \psi = \chi + \widehat{\chi} \)
\item where does mass come from?: mass term in Dirac-eq is of from \(\bar{\psi}\psi\), only \(\bar{\psi_L}\psi_R\) and \(\bar{\psi_R}\psi_L\) remains, \(\bar{\psi_L}\psi_L\) and \(\bar{\psi_R}\psi_R\) cancel out
\item \(\rightarrow\) Weyl fermions has special chirality therefore mass term must vanish, massive fermions must have left-chiral and right-chiral components
\end{itemize}
\item Dirac fields are completely unconstrained solutions of Dirac equation
\item Weyl/Majorana fields are simpler solutions with some kind of constrained imposed on solution, Weyl: chirality condition, Majorana: reality condition
\end{itemize}
% hier muss ich mich ein wenig einlesen ... mögliche Literatur:
% KTA-Expert script
% 0: https://arxiv.org/pdf/1006.1718v2.pdf

\subsection{Neutrinoless Double Beta Decay}
\label{sec:NDBD}
\begin{itemize}
\item Neutrino Oscillation have shown that neutrinos have finite mass, with NeOs only difference in mass measurable, lower limit on absolute mass with \(\mathrm{m}_{scale} = \sqrt{\Delta \mathrm{m}^2}\)
\begin{itemize}
\item SuperKamiokande showed mixing between \(\nu_\mu\) and \(\nu_\tau\) of atmospheric neutrinos
\item "solar neutrino puzzle" solved with mixing of \(\nu_e\) and mixture of \(\nu_\mu\) and \(\nu_\tau\)
\item NeOs can not determin absolute masses and also not separate between two different scenarios:
\begin{itemize}
\item hierarchical pattern ( \(m_i\)  ~= \(\sqrt{\Delta m^2}\))
\item degenerate pattern ( \(m_i >> \sqrt{\Delta m^2}\))
\end{itemize}
\end{itemize}
\item \(\beta\beta(0\nu)\) decay can only proceed when Neutrinos are massive Majorana particles
\begin{itemize}
\item standard electroweak model postulates that neutrinos are massless and total lepton number is conserved -> with \(\beta\beta(0\nu)\) physics beyond SM
\item double beta decay is rare transition between two nuclei with the same mass number A involving change of nuclear charge Z by two units
\begin{itemize}
\item can only proceed if initial nucleon is less bound than final and both are more bound than intermediate nucleon -> only fulfilled for even-even nucleons
\item \(\beta\beta(2\nu)\): \( (Z,A) \rightarrow (Z + 2, A) + e^-_1 + e^-_2 + \bar{\nu_{e1}} + \bar{\nu_{e2}} \), conserves lepton number
\item \(\beta\beta(0\nu)\): \( (Z,A) \rightarrow (Z + 2, A) + e^-_1 + e^-_2\), violates lepton number conservation
\item \(\beta\beta(0\nu, \chi)\): \((Z,A) \rightarrow (Z + 2, A) + e^-_1 + e^-_2 + \chi \)
\end{itemize}
\item easy to distinguish the three decay modes by shape of \(e^-\)-sum energy spectrum
\begin{itemize}
\item 2\(\nu\): broad maximum below half of endpoint
\item 0\(\nu\): \(e^-\) carry full available kinetic energy, single peak at endpoint
\item 0\(\nu,\chi\): again continuous, maximum shifted above halfway point
\end{itemize}
\end{itemize}
\item Majorana, Dirac neutrinos (again from above, maybe move stuff there)
\begin{itemize}
\item Majorana: particles that are identical with their own antiparticles, two component objects
\item Dirac: one can distinguish, four component objects
\item massive fermions usually described by Dirac eq with coupling of chiral eigenstates \(\psi_L,\psi_R\), \(\Psi = \begin{pmatrix} \psi_R \\ \psi_L\end{pmatrix}\)
\item Majorana suggested alternative description of massive fermions which do not have additive quantum numbers as two component states, chiral eigenstates connected via \(\psi_L = \epsilon \psi_R^*\)
\end{itemize}
\item Lorentz invariant mass in Dirac Lagrangian:
\begin{itemize}
\item Dirac mass: \(M_D [\bar{\nu_R}\nu_L + (\bar{\nu_L})^*\nu_R^*]\), requires both chirality eigenstates, conserves total lepton number
\item Majorana mass: \(M_L [(\bar{\nu_L})^*\nu_L + \bar{\nu_L}\nu_L^*]\), \(M_L [(\bar{\nu_R})^*\nu_R + \bar{\nu_R}\nu_R^*]\), violates total lepton number conservation, can be present even w/o existents of \(\nu_L/\nu_R\)
\item generally all three terms may coexist, when Lagrangian is diagonalized the resulting two general non-degenerate mass eigenvalues for each flavor (see-saw: light and heavy particle for each flavor) \(M = \begin{pmatrix} M_L & M_D \\ M_D & M_R\end{pmatrix}\) 
\item M diagonalized by unitary matrices \(\begin{pmatrix} U \\ V \end{pmatrix}\), U,V general mixing matrices, if non of the \(\nu_R\) states exist or \(M_R\) is so large only \(M_L\) is relevant and only U needed
\end{itemize}
\item Majorana mass
\begin{itemize}
\item transition amplitude for Majorana neutrino mass \(m_e\) is sum over product of \(m_j\) and combination of nuclear mixing element
\item in each vertice electron is emitted, therefor mixing amplitude \(\mathrm{U}_{ej}\) appears in each of them
\item \(\Rightarrow \beta\beta(0\nu)\) decay amplitude contains factor \(\mathrm{U}_{ej}^2\) not \(|\mathrm{U}_{ej}|^2 \Rightarrow <m_\nu> = |\sum_jm_j\mathrm{U}_{ej}^2|\) 
\end{itemize}
\item oscillation parameters
\begin{itemize}
\item \(<m_\nu>^2 = |\sum_jm_j\mathrm{U}_{ej}^2|^2 = |\sum_jm_j|\mathrm{U}_{ej}|^2e^{i\alpha}|^2\) with \(\alpha\) being the Majorana phase
\item possibility of cancellation of sum (Zee model \(<m_\nu> = 0\))
\item \(<m_\nu> = 0\) depends on unknown phase but upper/lower limit only depends on absolute values of mixing angles\(<m_\nu> = \sum_jm_j|\mathrm{U}_{ej}|^2\)
\item when \(<m_\nu>\) is known from tritium beta decay experiments one can determine phase \(\alpha_i\)
\end{itemize}
\end{itemize}

% also a bit about standard double beta decay
% differences between the standard and neutrinoless beta decay
 
\subsection{LAr as coolant} 
\label{sec:LArcoolant}


\subsection{\nuc{Kr}{85} isotop}
\label{sec:Kry85}

% where does it come from?
% what properties does it have?
% why is it important to calculate its influence on GERDA

\section{GERDA}
\label{sec:GERDA}

% general Information, e.g. 
% sizes 
% Gran Sasso, 
% what other  neutrino less beta decay experiments are there,

\subsection{Aim of \GERDA}
\label{sec:AimGERDA}

% not sure whether this is enough for its own section, maybe just connect it with general info

\subsection{Experimental Setup}
\label{sec:ExSetup}

% well, just dump everything here
% also a bit about Tier1-4 storage of data 

\subsection{Background Reduction}
\label{sec:BGReduction}

% Mui, Mountain, Pulse shape disc.
% especially LAr-Veto 

\subsection{Results}
\label{sec:ResultsofGERDA}

% what has happened so far

\section{Analysis of the \nuc{Kr}{85} concentration}
\label{sec:AotKr}

The main goal of this theses is to approximate the concentration of the isotope \nuc{Kr}{85} in the liquid argon coolant and scintillate of the GERDA experiment. 
This value is of special interest, as in other experiments this isotope in the liquid argon has produced a not negligible background. 
For example, an activity of \((2.05\pm0.13) \mathrm{\frac{mBq}{kg}}\) was measured in the Darkside experiment \cite{PhysRevD.93.081101} and an activity of \((0.16\pm0.13)\frac{Bq}{l}\) in the WARP experiment \cite{Benetti:2006az}. 
As it is calculated later in section \ref{sec:calcOfTheCon} their concentrations are at HIER BERECHNETE WERTE HIN and show a distinct signal in the measured phase diagram by creating background events at the lower end of the spectrum. 
The concentration of \nuc{Kr}{85} in the GERDA experiment therefor of interest as it provides an estimation of its influence on the measured background.\\

In the curse of this theses two different approaches are used to calculate the concentration of \nuc{Kr}{85}. 
The first and more precise method uses the rare \nuc{Kr}{85} decay into an excited state while the other method uses the reduction in the overall event rate over time.\\

Section \ref{sec:Appro} concentrates on explaining the ideas and problems of theses two methods and compares the expected precision of them. 

The following two chapters describe the two analytical approaches in greater detail and provides the results found when using them on the data measured by GERDA.

The last section is used to finally calculate the concentration of \nuc{Kr}{85} at the starting point of GERDA Phase II using the two activities determined in the prior sections. 
Finally, the result is compared with the activities from Darkside and WARP. 


\subsection{Different Approaches to calculate the activity}
\label{sec:Appro}

The aim of this section is to give the reader an overview of the two methods used in this theses to determine the concentration of \nuc{Kr}{85}. 
The later sections go into further details about the concrete procedures.
\\

The first method uses \nuc{Kr}{85}'s property to decay into an excited state of \nuc{Rb}{85} with a probability of 0.43\%. 
This state has a raised energy level of 514keV compared to its ground state and has a half-life of 1.015\(\mu\)s. 
Theoretically it should now be possible to measure a peak in the phase diagram around the 514keV mark. 
One can than use a fit to determine the number of decays measured by the detectors. 
Due to the different energy resolution in the BEGes and the COAX detectors, the decay count in their signals will be done separately. 
It also of interest to know the overall detector efficiency of the two kinds. 
This can be approximated using a Monte-Carlo simulation creating a huge amount of 514keV decay events and counting the measured events in the detector. 
It should now be possible to determine the activity of \nuc{Kr}{85} by using the measured events, the detector efficiency and the volume from the Monte-Carlo simulation, the mass of the detectors and the exposure of the detectors.
\\

The biggest problem of this method to overcome lies with the proximity of the \nuc{Kr}{85} to the 511keV peak of the positron electron annihilation. 
Its peak is expected to partially dominate over the \nuc{Kr}{85} in the phase diagram and does not allow for a direct measurement of the 514keV peak. 
But due to the low energy of the escaping electron (47.65keV) in the rare decay is very unlikely to create any scintillating light in the liquid argon. 
Therefor it should be possible to single out the 514keV events from the annihilation events that generally create a strong light signal by using the liquid argon veto.
\\

The second method uses the reduction of the event rate over the range of 200 to 400 keV. 
This is only possible because every other isotope we know that is present in the liquid argon with a non negligible proportion has a half-life much higher than \nuc{Kr}{85} with its \(10.739\mathrm{y}\). 
The dominant background sources are in descending order \nuc{Ar}{42} with a half-life of 32.9 y, \n uc{Ar}{39} with 269 y and \nuc{U}{232} with 68.9 y. 
Since the liquid argon has not been replaced since the beginning of GERDA Phase I in 2013, it can be estimated that the Kr85 rate has decreased by a factor of \(\sqrt{2}\) while the rate of the other isotopes should have hardly changed.
\\

Due to routine check-ups and improvements to the experimental setup there were several time intervals in which the detectors were deactivated. 
This means that one has to find a way to weigh the measured rates with a signal that corresponds to the times the detector was actually active. 
Luckily the test pulser signal can be used as such a reference. 
!!!!! Hier noch eine kurze Erklärung zum TP !!!!!
%This test pulse emits a strong signal of TESTPULSERENERGIEHIER every twenty seconds and every measured event of it is separately marked as a test pulse signal. 
These signals can now be used as weigh to correct the previous histogram.
\\
% By laying a fit function of an exponential decrease plus a constant background over it one should be able to deduct the activity of the \nuc{Kr}{85} from the amplitude of the exponential decrease. With this one can then calculate the concentration by integrating over the activity and dividing over the liquid argon volume.\\

Because the second method is based on a very broad approximation it is expected to only give a rough estimation of the actual concentration. 
In contrast, the first method is expected to wield a much better appraisal because it does not heavily rely on any approximation.
\\

% short outline of approaches
 
\subsection{Activity from the 514keV peak}
\label{sec:SAfrom514}

As described in section \ref{sec:AotKr} this part of the theses concentrates on the first approach of finding the activity of \nuc{Kr}{85} by using relaxation of \nuc{Rb}{85} in a rare decay. 
The photon emitted carries an energy of almost exactly 514keV. 
It should therefor be possible to measure it in one of the detectors. 
When one plots all events measured by GERDA Phase II in a phase diagram from 500 to 525keV (as seen in figure \ref{fig:ungefiltertes500525}), two larger peaks are visible. 
The first peak at about 511keV is easily identifiable as the expected positron electron annihilation peak where as the second peak at about 514keV corresponds to the rare \nuc{Kr}{85} decay into the excited \nuc{Rb}{85} state. 
This already means that, like it was expected, there must be a non neglectable portion of \nuc{Kr}{85} in the liquid argon.
\\

\begin{figure}[ht]
	\centering
	\ifmakefigures%
	\includegraphics[width=130mm]{./Bilder/GraphNoFiltersAtAllAll.pdf}
	\fi%
	\caption{\label{fig:ungefiltertes500525}
		All events measured by all detectors in the range of 500keV to 525keV with no filter applied showing two overlapping peaks at 511keV and 514keV.
	}
\end{figure}



As it was elaborated in section \ref{sec:ExSetup} all GERDA data is stored in a multi-tier data structure. 
All of the data used in this theses are from either tier 3 or from tier 4.
In these tiers a lot of analysis has already been carried out on the individual events measured. 
Among other things, each event in tier 3 and 4 is given a flag whether or not there was a coincidence of the event with a signal in one of the photomultiplier in the water tank around the liquid argon.
This flag is called the "Muon veto" and it always triggers whenever there is a strong Cherenkov signal measured.
Due to the high energy needed to create any Cherenkov signal no isotope from inside the germanium source and liquid argon and certainly no \nuc{Kr}{85} should trigger the MuiVeto.
This can be used to filter out high-energy particles from outside and the background they create.
\\

For a photon of a rare \nuc{Kr}{85} decay to create the distinct 514keV peak it is necessary that it has to deposit all of its energy in only one of the detectors. 
This means that any energy measured in one Germanium detectors that has a coincidence with a non neglectable signal measured in another Germanium detector is most likely not a real rare \nuc{Kr}{85} decay.
Whether an event has more than one detector measure a signal can be determined by the multiplicity counter of each event stored in tier 3 and 4.
Using this counter even more background of none \nuc{Kr}{85} can be repressed by only using events where only one detector was triggered.
\\

\begin{figure}[ht]
	\centering
	\ifmakefigures%
	\includegraphics[width=130mm]{./Bilder/500525NoFilterAllDetectors.pdf}
	\fi%
	\caption{\label{fig:leichtGefiltertes500525}
		All events measured by all detectors in the range of 500keV to 525keV with the Mui Veto and only one detector triggered.
	}
\end{figure}

With these two filters you get a new graph with less background seen in figure \ref{fig:leichtGefiltertes500525}.
One can see that the annihilation events could be partially suppressed by these two filters making the two peaks more distinguishable.
From now on, only those events that came through these two filters will be used in further analysis.
\\

After applying the first few rather general filters on the measured events one has to make a distinction between the detectors.
As it was elaborated in section \ref{sec:ExSetup}, two different types of detectors were used in the GERAD experiment - the BEGes (Broad Energy Germanium diods) and the COAX (coaxial diods). 
Due to their differences in design and weigh the two types have a different energy resolution. 
The BEGes are generally smaller and therefor have a higher energy resolution compared to the rather big COAX detectors.
Because of this their measured data must and will be evaluated separately. 
If you only take the events that are recorded in the respective detectors, you get figure \ref{} for all the events measured in the BEGe detectors and figure\ref{} for all the events in the COAX detectors.
\\

\begin{figure}[ht]
\centering
\begin{minipage}{.5\textwidth}
  \centering
	\includegraphics[width=80mm]{./Bilder/500525NoFilterBEGes.pdf}
  \caption{figure}{All events measured by only BEGe detectors in the range of 500keV to 525keV.}
  \label{fig:test1}
\end{minipage}%
\begin{minipage}{.5\textwidth}
  \centering
	\includegraphics[width=80mm]{./Bilder/500525NoFilterCOAX.pdf}
  \caption{figure}{All events measured by only COAX detectors in the range of 500keV to 525keV.}
  \label{fig:test2}
\end{minipage}
\end{figure}

At this point it is clear that we have two options on how we can handle the problem of the positron electron annihilation peak. 
Either we try to find a way to filter out the majority of the annihilation events while keeping almost all of the \nuc{Kr}{85} events. 
This can hopefully be done by using the liquid argon veto of GERDA and the precoincidence of the electron scintillation in the photomultipliers. 
Or we try to leave it as it is and adjust our fit function to a fit over both of the peaks using a double Gaussian peak. 
In this theses we will try to go both ways separately and later compare their informative value.
\\

\subsubsection{Annihilation peak suppression}
\label{sec:APS}

First, we will try to get a better shape of the 514keV peak by trying to suppress the annihilation peak using the liquid argon veto and will later discuss its effectiveness.
\\

Among other things, each event in tier 4 was given a flag called "isLArVetoed".
This flag is always triggered whether there was a coincidence of the specific event with a strong scintillation signal in one of the photomultiplier in the liquid argon. 
Due to the low energy of the beta electron (E\(_{mean}=47.65\)keV) released in the beta decay it can be expected that almost none of the electrons of the rare \nuc{Kr}{85} decay create any light in the scintillator. 
In contrast to that you can expect a very strong light signal every time a positron electron annihilation occurs. 
One should therefor be able to filter out almost all of the annihilation events while keeping the majority of the \nuc{Kr}{85} decay by only using events where this flag is not triggered.
\\

If one plots all events that got through this filter one gets figure \ref{} for the BEGes and figure \ref{} for the COAX detectors.
One can see that the positron electron annihilation peak can not be identified anymore while the 514keV peak is almost unchanged.
\\

\begin{figure}[ht]
\centering
\begin{minipage}{.5\textwidth}
  \centering
	\includegraphics[width=80mm]{./Bilder/500525LArVetoBEGes.pdf}
  \caption{figure}{All events measured by only BEGe detectors in the range of 500keV to 525keV.}
  \label{fig:test1}
\end{minipage}%
\begin{minipage}{.5\textwidth}
  \centering
	\includegraphics[width=80mm]{./Bilder/500525LArVetoCOAX.pdf}
  \caption{figure}{All events measured by only COAX detectors in the range of 500keV to 525keV.}
  \label{fig:test2}
\end{minipage}
\end{figure}

At this point one could say that we have done enough filtering and use a fit functions to determine the number of decays in the peak.
But due to the non vanishing possibility of the beta electron to trigger the liquid argon veto it is of interest to investigate whether the filter has accidentally filtered out an actual rare \nuc{Kr}{85} decay.
The number of events filtered out by the liquid argon veto in the energy range of 509 to 519 keV is 1728.
This number of evens is way to big to look at every individual case.
\\

Luckily we can use the fact that the excited \nuc{Rb}{85} state has a half life of 1.015 \(\mathrm{\mu}\)s.
This means that we can identify a real rare \nuc{Kr}{85} decay by only looking at the events where the signal in the liquid argon happened before the event was measured in one of the Germanium detectors.
We also know that the energy of the released beta electron is relatively low.
This means that we can expect that only a small number of photomultipliers should measure a signal in a real rare \nuc{Kr}{85} decay.
\\
\begin{figure}[ht]
	\centering
	\ifmakefigures%
	\includegraphics[width=130mm]{./Bilder/TriggerTimeOnly2.pdf}
	\fi%
	\caption{\label{fig:ungefiltertes500525}
		All liquid argon filtered events with a negative time difference between the event in the Germanium detector and a signal in the photomultipliers in the liquid argon tank.
	}
\end{figure}

If you apply these two restrictions to the liquid argon filtered events you get a distributions as seen in figure \ref{}.
In this case, only events with a maximum of two triggered photomultipliers are shown in this graphic.
The x axis is the time difference of the events in the photomultipliers from the first signal measured in one of the Germanium detectors.
This means that negative times difference mean that the event in the photomultiplier was measured before there was even an event in the Germanium detectors.
Theoretically it should now be possible to see a exponential increase.
\\

But because of the small number of events left over, it is almost impossible to make any statements about the course of these events.
Nevertheless we were able to lower the number of potential \nuc{Kr}{85} events from 1728 down to only 48.  
\\





% calculate Amplitude of Gauss peak at 514keV and use factor from Monte Carlo Simulation to estimate

% look at phase diagram at range of 500 to 525 keV, use different filters and fit remaining data with Gaussian function
% -> get amplitude
% make a Monte Carlo simulation to estimate actual Kr85 activity in LAr from measured activity in detectors
% -> with amplitude and factors from MC-Simulation one can calculate the specific activity

\subsection{Activity from the decrease in rate}
\label{sec:SAfromDecrease}

% use fit to calculate the decrease in rate of the signals in range from 200 to 500keV
% from fit and assumption that only Kr85's activity is decreasing one can calculate the specific activity from the amplitude of the fit

% use Volume of LAr-Tank to determine the number of Kr85 and from this the concentration in the argon

\subsection{Calculation of the concentration}
\label{sec:calcOfTheCon}


\section{Conclusion}

% what went wrong?
% what does it mean?
% possible future enhancements to determine the Kr85 concentration

































% %%%%%%%%%%%%%%%%%%%%%%%%%%%%%%%%%%%%%%%%%%%%%%%%%%%%%%%%%%%%%%%%%%%%%%%%%%%%%%%%
% \section{Another Section}
% \label{sec:secname}


% Fig.~\ref{fig:pmt} shows a PMT.

% %------------------------------------------------------------------- figure ----
% \begin{figure}[hb]
% \centering
% \ifmakefigures%
%    \includegraphics[width=45mm]{kapselung-small.jpg}
% \fi%
%   \caption{\label{fig:pmt}
% The encapsulation of the Cerenkov PMT.
% }
% \end{figure}
% %-------------------------------------------------------------------------------

% %%%%%%%%%%%%%%%%%%%%%%%%%%%%%%%%%%%%%%%%%%%%%%%%%%%%%%%%%%%%%%%%%%%%%%%%%%%%%%%%
% \section{Results and Analysis}
% \label{sec:results}


% %----------------------------------------------------------------- equation ----
% {\centering
% \begin{equation}\label{eq:sensit}
%     T_{1/2}(0^{+} \rightarrow g.s.~with~single~\gamma)~ \geq ~\ln2 \cdot
%     \varepsilon \cdot a \cdot \frac{M \cdot N_A}{A} \cdot 
%     \sqrt{\frac{\Delta T}{b\cdot\Delta E}},
% \end{equation}
% } % end centering
% %-------------------------------------------------------------------------------

% Table~\ref{tab:param} compiles everything.

% %-------------------------------------------------------------------- table ----
% \begin{table}[t]
% \centering
% \caption{\label{tab:param}
% Experimental parameters and values.
% }
% \vspace*{2mm}
% \begin{tabular}{L{4cm}|C{6cm}}
%   Column1 & Column2 \\  \hline 
%   Row1 & $100\pm10$ \\
%   Row2 & $100\pm10$ \\ \hline
%   Row3 & $100\pm10$ \\
%   Row4 & $100\pm10$ \\ 
% \end{tabular}
% \end{table}
% %-------------------------------------------------------------------------------

% %%%%%%%%%%%%%%%%%%%%%%%%%%%%%%%%%%%%%%%%%%%%%%%%%%%%%%%%%%%%%%%%%%%%%%%%%%%%%%%%
% \section{Conclusions}
% \label{sec:conclusions}
