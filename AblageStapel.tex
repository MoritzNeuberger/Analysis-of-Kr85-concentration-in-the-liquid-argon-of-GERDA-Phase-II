
\section{Weyl, Majorana and Dirac fermions}
\label{sec:WMDf}


Dieser Teil wird noch stark gekürzt bzw. ich werde mich hier wahrschenlich vor allem auf "Kerne und Teilchen", der KTA Script und ein paar einfürhende Paper stützen.

% in this chapter almost everything from source 0

\begin{itemize}
\item historic introduction of question whether neutrinos are dirac or majorana fermions
\begin{itemize}
\item from Dirac relativistic equation fermion fields
\item electrons: have mass and charge, Dirac-eq predicts antiparticles, requires 4-comp fields
\item Weyl calculates for massless fermions that only two-component fields are necessary
\item Pauli predicts neutrinos in letter, no charge, seems to have vanishing mass 
\item \(\rightarrow\) assumption: neutrinos  are Weyl fermions
\item Majorana: neutrinos are antiparticles of itself since they are uncharged, first not taken seriously, only after first indications that neutinos have mass
\item \(\rightarrow\) discussion whether neutrinos are Dirac or Majorana fermions
\end{itemize}
\item short introduction to dirac-eq
\begin{itemize}
\item \((i\hbar\gamma^\mu \partial_\mu  - mc)\psi = 0\) , maybe Hamilton/Lagragian, has plane waves as solution multiplied with spinor
\item Spinors: any column like function of energy and momentum which when multiplied by factor \(\exp(i\vec{p}\cdot\vec{x})\) or  \(\exp(\-i\vec{p}\cdot\vec{x})\) becomes a solution of the Dirac equation
\end{itemize}
\item we know that the Klein-Gordon-eq is real, how can we get a real solution from the dirac-eq
\begin{itemize}
\item depends on how we choose our \(\gamma^\mu\), if all non-zero elements of all the \(\gamma^\mu\) are purly imaginary, then Dirac-eq is real
\item Majorana matrices, with usage in Dirac-eq one can obtain real solutions that satisfy \(\tilde{\psi} = \tilde{\psi}^*\)
\item \(\rightarrow\) these solutions represent Majorana fermions
\item general solution to Majorana condition can be obtained by transformation with unitary matrix: \(\gamma^\mu = \mathrm{U}\tilde{\gamma^\mu}\mathrm{U}^\dagger\)
\item general Majorana condition: \(\psi = \mathrm{U}\mathrm{U}^\top\tilde{\psi}^*\), with \(\mathrm{U}\mathrm{U}^\top \equiv \gamma^0 \mathrm{C}\)
\item with compact notation \(\widehat{\psi} \equiv \gamma^0 \mathrm{C} \psi^*\)
\item general definition of a Majorana fermion fields through definition: \(\psi = \widehat{\psi}\), condition is Lorenz invariant
\end{itemize}
\item clunky repetition of what helicity and chirality is and what their problem with massive fermions is
\begin{itemize}
\item helicity defined as twice the value of the spin component of the particle along the direction of the momentum \(h_{\vec{p}} = \frac{2 \vec{J} \cdot \vec{p}}{p}\)
\item eigenstates of eigenvalue -1 called "left-helical", eigenstates of eigenvalue +1 called "right-helical"
\item is invariant under time/rotation, not invariant under boost
\item chirality meaning assigned to matrix \(\gamma_5 = i\gamma^0\gamma^1\gamma^2\gamma^3\)
\item projection matrices on fermion fields: \( \mathrm{L} = \frac{1}{2} \left( 1- \gamma_5\right ) \), \( \mathrm{R} = \frac{1}{2} \left( 1 + \gamma_5\right ) \)
\item projections of L, R are called lift/right-chiral
\item wavefunction can be written as \(\psi = \psi_L + \psi_R\) with \(\psi_L = \mathrm{L}\psi\) and \(\psi_R = \mathrm{R}\psi\)
\item \(\rightarrow\) helicity: conserved for free particles, not under Lorenz trafo
\item \(\rightarrow\) chirality: is Lorenz invariant, not conserved
\item \(\rightarrow\) both not appropriate for characterizing a fermion that has mass
\end{itemize}
\item how to define Weyl fermions
\begin{itemize}
\item problem with helicity/chirality disappears if the fermion is massless
\item general solution of Dirac-eq is not irreducible representation of Lorentz group (Lorenz group: group of Lorentz trafo in Mirkowski-space-time)
\item \(\rightarrow\) left/right-chiral fields are Lorentz invariant, representation with 2-component-field: \( \begin{pmatrix}\frac{1}{2} \\ 0\end{pmatrix}\) for left-chiral, \(\begin{pmatrix}0 \\ \frac{1}{2}\end{pmatrix}\) for right-chiral
\item when \(\chi\) is left chiral Weyl fermions, \( \widehat{\chi} \) is a right chiral Weyl fermions
\item a general fermion field can be described by two Weyl fields \(\rightarrow\) building blocks
\end{itemize}
\item how can we build Majorana/Dirac fermions from Weyl fermions
\begin{itemize}
\item Majorana and Dirac fermions both have mass and therefor must have both left/right chiral components
\item Dirac can be created by two independent left chiral Weyl fields \(\chi_1\), \(\chi_2\): \(\psi = \chi_1 + \widehat{\chi_2}\)
\item Dirac fermions are unconstrained solutions of the Dirac equation
\item unlike the Dirac fermions, the Majorana fermions must fulfill the reality condition \( \psi = \widehat{\psi} \)
\item Majorana fermions are represented by: \( \psi = \chi + \widehat{\chi} \)
\item where does mass come from?: mass term in Dirac-eq is of from \(\bar{\psi}\psi\), only \(\bar{\psi_L}\psi_R\) and \(\bar{\psi_R}\psi_L\) remains, \(\bar{\psi_L}\psi_L\) and \(\bar{\psi_R}\psi_R\) cancel out
\item \(\rightarrow\) Weyl fermions has special chirality therefore mass term must vanish, massive fermions must have left-chiral and right-chiral components
\end{itemize}
\item Dirac fields are completely unconstrained solutions of Dirac equation
\item Weyl/Majorana fields are simpler solutions with some kind of constrained imposed on solution, Weyl: chirality condition, Majorana: reality condition
\end{itemize}
% hier muss ich mich ein wenig einlesen ... mögliche Literatur:
% KTA-Expert script
% 0: https://arxiv.org/pdf/1006.1718v2.pdf