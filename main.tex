\documentclass[encoding=utf8,british]{tumphthesis}
% \documentclass[pstricks,siunitx,addfonts,theorem,font=palatino,british]{tumphthesis}

% Das folgende Paket dient lediglich dazu, den Blindtext "Lorem ipsum ..."
% auszugeben und kann in einer echten Abschlussarbeit natürlich weggelassen 
% (oder auskommentiert) werden.
\usepackage{lipsum}
\usepackage{graphicx}
\usepackage{caption}
\usepackage{subcaption}
\usepackage{units}
\usepackage{capt-of}
\usepackage{subfig}
\usepackage{csquotes}
%\usepackage{natbib}
\usepackage{physics}
\usepackage[compat=1.1.0]{tikz-feynman}
\tikzset{/tikzfeynman/warn luatex=false}

\include{gerda-abbreviations}


\makeatletter
\renewcommand{\@chapapp}{}% Not necessary...
\newenvironment{chapquote}[2][2em]
  {\raggedright
   \setlength{\@tempdima}{#1}%
   \def\chapquote@author{#2}%
   \parshape 1 \@tempdima \dimexpr\textwidth-2\@tempdima\relax%
   \itshape
   }
  {\par\normalfont\hfill--\ \chapquote@author\hspace*{\@tempdima}\par\bigskip}
\makeatother


%\usepackage{draftwatermark}
%\SetWatermarkScale{3}

% Die Metadaten der Abschlussarbeit werden auf dem Deckblatt gedruckt und
% in dem PDF eingetragen.
\subject{Abschlussarbeit im Bachelorstudiengang Physik}
%\title{Analysis of Kr85 concentration in the liquid argon of \gerda\ Phase II}
\title{Determination of the specific \Kr\ activity in the liquid argon of \gerda\ Phase II}
%\subtitle{\foreignlanguage{british}{Title in English}}
\author{Moritz Neuberger}
\date{7.~August 2018}
%\cooperators{Max-Planck-Institut für Physik}

% Auf der Rückseite des Deckblatts können Themensteller, Zweitgutachter 
% und Tag der mündlichen Prüfung vermerkt werden.
\lowertitleback{Erstgutachter (Themensteller): Prof.\ S.~Schönert\\
Zweitgutachter: Prof.\ S.~Preuss}


%-------------------------------------------------------------------------------
%\include{header}              % packages/settings
%\include{\gerda\-abbreviations} % for definitions see:  abbreviations.tex
%-------------------------------------------------------------------------------

%%%%%%%%%%%%%%%%%%%%%%%%%%%%%%%%%%%%%%%%%%%%%%%%%%%%%%%%%%%%%%%%%%%%%%%%%%%%%%%%
% -----------------------------------------------------  DEFINE YOUR REPORT  ---
%%%%%%%%%%%%%%%%%%%%%%%%%%%%%%%%%%%%%%%%%%%%%%%%%%%%%%%%%%%%%%%%%%%%%%%%%%%%%%%%
%\newcommand{\repnumber}  {GSTR-18-0xx}    % REPNUMBER (yy->year, 0xx->running number)
%\newcommand{\repdate}    {August 8, 2018} % DATE
%\newcommand{\titleheader}{Analysis of \Kr concentration in the liquid argon of \gerda\ Phase II} % TITLE
%%%%%%%%%%%%%%%%%%%%%%%%%%%%%%%%%%%%%%%%%%%%%%%%%%%%%%%%%%%%%%%%%%%%%%%%%%%%%%%%
%%%%%%%%%%%%%%%%%%%%%%%%%%%%%%%%%%%%%%%%%%%%%%%%%%%%%%%%%%%%%%%%%%%%%%%%%%%%%%%%


% --------------------------------------------------------  begin document   ---
\begin{document}
\selectlanguage{british}
\frontmatter
\maketitle


% --------------------------------------------------------  begin title page  --
%\begin{titlepage}
% --------------------------------------------------------- logo /header -------
%\vspace*{20mm}
%\begin{center}
%{\Large\textbf{\titleheader}}
%\vspace*{10mm}

%%%%%%%%%%%%%%%%%%%%%%%%%%%%%%%%%%%%%%%%%%%%%%%%%%%%%%%%%%%%%%%%%%%%%%%%%%%%%%%%
% ----------------------------------------------------  DEFINE YOUR AUTHORS  ---
%%%%%%%%%%%%%%%%%%%%%%%%%%%%%%%%%%%%%%%%%%%%%%%%%%%%%%%%%%%%%%%%%%%%%%%%%%%%%%%%
%Moritz Neuberger, Christoph Wiesinger$^o$), Steffan Schönert$^o$)
%\vspace*{5mm}

% Address:  (select the appropriate ones and change letters if needed) 
%$^a$)  INFN Laboratori Nazionali del Gran Sasso  and Gran Sasso Science Institute, Assergi,  Italy\\[1mm]
%$^b$)  INFN Laboratori Nazionali del Sud, Catania, Italy\\[1mm]
%$^c$)  Institute of Physics, Jagiellonian University, Cracow, Poland\\[1mm]
%$^d$)  Institut f{\"u}r Kern- und Teilchenphysik Technische Universit{\"a}t Dresden, Dresden, Germany\\[1mm]
%$^e$)  Joint Institute for Nuclear Research, Dubna, Russia\\[1mm]
%$^f$)  European Commission, JRC-Geel, Geel, Belgium\\[1mm]
%$^g$)  Max-Planck-Institut f{\"u}r Kernphysik, Heidelberg, Germany\\[1mm]
%$^h$)  Universit{\`a} di Milano Bicocca, Milan, Italy\\[1mm]  
%$^i$)  INFN Milano Bicocca, Milan, Italy\\[1mm]                                     
%$^j$)  Universit{\`a} degli Studi di Milano e INFN Milano, Milan, Italy\\[1mm]                                 
%$^k$)  Institute for Nuclear Research of the Russian Academy of Sciences, Moscow, Russia\\[1mm]                 
%$^l$)  Institute for Theoretical and Experimental Physics, NRC ``Kurchatov Institute'', Moscow, Russia\\[1mm]                                
%$^m$)  National Research Centre ``Kurchatov Institute'', Moscow, Russia\\[1mm]                                       
%$^n$)  Max-Planck-Institut f{\"ur} Physik, Munich, Germany\\[1mm]                                          
%$^o$)  Physik Department E15 and Excellence Cluster Universe, Technische  Universit{\"a}t M{\"u}nchen, Munich, Germany\\[1mm]                           
%$^p$)  Dipartimento di Fisica e Astronomia dell{`}Universit{\`a} di Padova, Padova, Italy\\[1mm]         
%$^q$)  INFN  Padova, Padova, Italy\\[1mm]
%$^r$)  Physikalisches Institut, Eberhard Karls Universit{\"a}t T{\"u}bingen, T{\"u}bingen, Germany\\[1mm]       
%$^s$)  Physik Institut der Universit{\"a}t Z{\"u}rich, Z{\"u}rich, Switzerland\\[1mm] 
%\vspace*{20mm}
%%%%%%%%%%%%%%%%%%%%%%%%%%%%%%%%%%%%%%%%%%%%%%%%%%%%%%%%%%%%%%%%%%%%%%%%%%%%%%%%
%%%%%%%%%%%%%%%%%%%%%%%%%%%%%%%%%%%%%%%%%%%%%%%%%%%%%%%%%%%%%%%%%%%%%%%%%%%%%%%%

%%%%%%%%%%%%%%%%%%%%%%%%%%%%%%%%%%%%%%%%%%%%%%%%%%%%%%%%%%%%%%%%%%%%%%%%%%%%%%%%
% --------------------------------------------------  WRITE A SHORT ASTRACT  ---
%%%%%%%%%%%%%%%%%%%%%%%%%%%%%%%%%%%%%%%%%%%%%%%%%%%%%%%%%%%%%%%%%%%%%%%%%%%%%%%%
%\begin{abstract}
%Seite beabsichtigt leer gelassen.
%\end{abstract}
%%%%%%%%%%%%%%%%%%%%%%%%%%%%%%%%%%%%%%%%%%%%%%%%%%%%%%%%%%%%%%%%%%%%%%%%%%%%%%%%
%%%%%%%%%%%%%%%%%%%%%%%%%%%%%%%%%%%%%%%%%%%%%%%%%%%%%%%%%%%%%%%%%%%%%%%%%%%%%%%%
%\end{center}
%\vfill
%\end{titlepage}
%\vfill

% ---------------------------------------------------------------- settings  ---
%\pagenumbering{arabic}
%\setcounter{page}{1}
%\pagestyle{myheadings}
%\markboth{~-~~\repnumber\hfill\titleheader}{\titleheader\hfill\repnumber~~-~}
%\newpage
%----------------------------------------------------------------- body --------

% Ist die Arbeit auf Englisch verfasst, hier die Sprache umschalten. 
% Die Sprache muss als Klassenoption angegeben sein.

%\chapter{Titel des ersten Kapitels}
%\section{Erster Abschnitt}
%\lipsum[2-5]\cite{schwabl-qqi2002}

%Und noch etwas \emph{betontes}.

%\section{Zweiter Abschnitt}
%\lipsum[6] Test\cite{Setare:2013dra}
%\begin{figure}
%	\centering
%	\includegraphics[width=\textwidth]{tumlogo}
%	\caption{\label{fig:test}Test}
%\end{figure}
%\subsection{Unterabschnitt}
%\lipsum[7]\cite{schwabl-qqi2002,schwabl-qffi2002}
%\subsubsection{Unterunterabschnitt}
%\paragraph{Absatz} \lipsum[8]


	%%%%%%%%%%%%%%%%%%%%%%%%%%%%%%%%%%%%%%%%%%%%%%%%%%%%%%%%%%%%%%%%%%%%%%%%%%%%%%%%
% ---------------------------------------------------  ADD HERE YOUR REPORT  ---
%%%%%%%%%%%%%%%%%%%%%%%%%%%%%%%%%%%%%%%%%%%%%%%%%%%%%%%%%%%%%%%%%%%%%%%%%%%%%%%%

%sources:
% 0: https://arxiv.org/pdf/1006.1718v2.pdf


\chapter*{Abstract}
The GERmanium Detector Array (\gerda) experiment tries to find evidence for the neutrino less double beta decay in \nuc{Ge}{76}.
Enriched germanium detector are used simultaneously as source and detector.
The liquid argon in which the detectors are located acts as coolant, passive shielding against radiation from the outside and as an active veto due to its scintillation capability.
Due to the ultra low background condition \gerda\ explores half life at about 10$^{26}$ yr.
A residual radioactive isotope in the liquid argon is \nuc{Kr}{85}.
It does not contribute to the background for \gerda\'s neutrinoless double beta decay search, but might influence analyses carried out at lower energies.
The aim of this thesis is to determine its specific activity.
%here: a bit more (and more positive) about the analysis
In about 0.434$\%$ the \Kr\ decays are followed by the emission of a gamma with 514 keV.
By using the line count of its peak in \gerda's energy spectrum, the detector efficiency of detecting these gammas and the mean measuring time a specific \Kr\ activity can be calculated.
Using this method a specific activity of $0.508\frac{\unit{mBq}}{\unit{l}}$ was determined.
A cross-check analysis facilitating the decay lifetime was carried out.
However, the presence of \nuc{Kr}{42} with comparable lifetime did not allow a verification of the obtained result.


\begin{otherlanguage}{ngerman}
\chapter*{Zusammenfassung}
Das GERmanium Detector Array (\gerda) Experiment versucht Beweise für den neutrinolosen doppelten Beta-Zerfall in \nuc{Ge}{76} zu finden.
Da bekannt ist, dass die Halbwertszeit eines solchen neutrinolosen doppelten Beta-Zerfall größer als 10$^{26}$yr sein muss, wird viel Aufwand dafür verwendent jeglichen Hintergrund zu unterdrücken.
Das flüssige Argon, in dem sich die Detektoren befinden, wirkt als Kühlmittel, als Abschirmung gegen äußere Strahlung und als aktives Veto aufgrund seiner Szintillationsfähigkeit.
Ein zurückgebliebenes radioaktives Isotop im flüssigen Argon, das einen messbaren Hintergrund in den Detektoren erzeugt, ist \nuc{Kr}{85}.
Das Ziel dieser Arbeit ist die Bestimmung der spezifischen Aktivität dieses Isotops.
Zur Messung dieses Wertes werden zwei unterschiedliche und unabhängige Ansätze angewendet.
Die zweite Methode, die die Änderung der Zählrate über die Zeit nutzt und nur als Gegenprobe gedacht war, scheiterte an falsch getroffenen Annahmen.
Die erste Methode jedoch, bei der die Anzahl an Messungen in einem bestimmen Interval verwendet wurde, bestimmte die spezifische Aktivität von \nuc{Kr}{85} auf $0,508\frac{mBq} {\unit{l}}$.
\end{otherlanguage}

\tableofcontents

\mainmatter

\chapter{Introduction}
\label{sec:intro}
%paragraph about sec. neutrino phys

The double beta (\twonu) decay has been observed in several isotopes where single beta decay is forbidden.
The neutrinoless double beta (\onbb) decay will only occur when neutrinos are Majorana fermions.
Majorana particles have the characteristic that they are particle and antiparticle at the same time, breaking the lepton number conservation.
The consequences of this will be discussed in  section \ref{sec:PhyBG}.
\\

The following section \ref{sec:gerda} will focus on the \gerda\ experiment.
The GERmanium Detector Array (\gerda) experiment tries to find evidence for the neutrino less double beta (\onbb) decay in \nuc{Ge}{76}.
\onbb\ decay is known to have a very long half life.
It is therefore important to minimize background as good as possible. 
A passive shield around the detectors for suppression of external radiation as well as a scintillator around the detectors to actively reject radiation coming from the outside are therefore needed.
Liquid argon (LAr) is a fitting material for these requirements due to its low boiling point, its good shielding characteristic and its ability to scintillate. 
Commercial argon is extracted from the atmosphere by air liquefaction. 
Impurities can be removed by cryogenic distillation but traces of radioactive components can be left.
One residual radioactive isotopes is \Kr\ and  will be discussed in section \ref{sec:Kry85}. 
\\

% what is the general aim of my Bachelor thesis
% general overview of how I'm going to do this
% ? what are my predictions ?
% What possible influence would the result of my work might have on the result of the \gerda\ experiment?

\section{Neutrino physics}
\label{sec:PhyBG}

\subsection{Neutrinos and lepton number conservation}

Neutrinos are electrically neutral elementary particles.
They are leptons and, apart from gravity, they only interact with other particles via the weak force and gravity.
They occur in three different kinds of flavors: the electron neutrino $\nu_e$, the muon neutrino $\nu_{\mu}$ and the tau neutrino $\nu_{\tau}$.
According to the standard model neutrinos would be massless.
Neutrino oscillation, however, has shown that neutrinos are massive.
But the absolute values of their mass and whether they are Dirac or Marjoram particles are still unknown.
\\

The major difference between these two kinds of particles is that, in the case of Dirac particles, one can clearly identify particles and anti-particles while Majorana particles only have one kind.
Majorana particles are particle and anti-particle at the same time.
This would allow lepton number violation and point towards physics beyond the standard model.
Until now, only one possible transition is known that could identify neutrinos as Majorana particles.
This transition is referred to as a neutrinoless double beta (\onbb) decay \cite{noauthor_phys._nodate-1}.
 
%what are Majorana neutrinos



\subsection{\onbb\ Decay}
\label{sec:0nubetabeta}

The double beta (\twonu) decay describes the transition of two neutrons to two protons in the same nuclei.
The standard model of particle physics needs two electrons and two anti electron neutrinos in the final state of this process.
\begin{equation}
(A,Z)\rightarrow (A,Z+2) + 2e^- + 2\bar{\nu_e}
\end{equation} 
The corresponding diagram is drawn in figure \ref{fig:Feyn2nbb}.
It is only observable if a single beta decay is forbidden, while a \twonu\ decay is allowed.
In most cases, this is due to the fact that the initial state of the nuclei $(A,Z)$ is stronger bound than the transition state $(A,Z+1)$ and less strong than the final state $(A,Z+2)$.
Their half lives are in the order of $10^{19}$ to $10^{21} \unit{yr}$
\\

\WarningsOff
\begin{figure}[t!]
\centering
\begin{subfigure}{.475\textwidth}
\centering
\begin{tikzpicture}
\begin{feynman}
\vertex 					(b1) 	{\(u\)};
\vertex[right =5cm of b1] 	(b2) 	{\(u\)};

\vertex[below=1em of b1] 	(b3) 	{\(d\)};
\vertex[right=5cm of b3] 	(b4) 	{\(d\)};

\vertex[below=1em of b3] 	(b5) 	{\(d\)};
\vertex[right=2.5cm of b5] 	(b6);
\vertex[below=1em of b4] 	(b7) 	{\(u\)};

\vertex[below=7em of b5] 	(c1) 	{\(d\)};
\vertex[right =2.5cm of c1] (c2);
\vertex[below=7em of b7]	(c3) 	{\(u\)};

\vertex[below=1em of c1] 	(c4) 	{\(d\)};
\vertex[right=5cm of c4] 	(c5) 	{\(d\)};

\vertex[below=1em of c4] 	(c6) 	{\(u\)};
\vertex[right=5cm of c6] 	(c7) 	{\(u\)};

\vertex[below=2em of b7]	(e1)	{\(e^-\)};
\vertex[left=2cm of e1]		(e2);
\vertex[below=3em of e2]	(e3);
\vertex[below=3em of e1]	(e4)	{\(e^-\)};

\vertex[below= 1em of e1]	(n1)	{\(\bar{\nu_e}\)};
\vertex[above= 1em of e4]	(n2)	{\(\bar{\nu_e}\)};

\diagram
{
{[edges=fermion]
  (b1) -- (b2),
  (b3) -- (b4),
  (b5) -- (b6),
  (b6) -- (b7)
  (c1) -- (c2),
  (c2) -- (c3),
  (c4) -- (c5),
  (c6) -- (c7),
  (e2) -- (e1),
  (e3) -- (e4),
  (e3) -- (n2),
  (e2) -- (n1),
},
(b6) -- [boson, edge label'=\(W\)] (e2),
(c2) -- [boson, edge label=\(W\)] (e3),
};
\draw [decoration={brace}, decorate]  (b5.south west)--(b1.north west) node [pos=0.5, left] {\(n\)};
\draw [decoration={brace}, decorate]  (c6.south west)--(c1.north west) node [pos=0.5, left] {\(n\)};
\draw [decoration={brace}, decorate]  (c3.north east)--(c7.south east) node [pos=0.5, right] {\(p\)};
\draw [decoration={brace}, decorate]  (b2.north east)--(b7.south east) node [pos=0.5, right] {\(p\)};
\end{feynman}
\end{tikzpicture}
\subcaption{Feynman-Diagram for the $2\nu\beta\beta$-decay. Two neutrons transfer into two protons, two electrons and two electron anti-neutrinos. This decay has already been observed, but only 12 nuclei are known to make this transition.}
\label{fig:Feyn2nbb}
\end{subfigure}\hfill%
\begin{subfigure}{.475\textwidth}
	\centering
	\begin{tikzpicture}
	\begin{feynman}
	\vertex 					(b1) 	{\(u\)};
	\vertex[right =5cm of b1] 	(b2) 	{\(u\)};
	
	\vertex[below=1em of b1] 	(b3) 	{\(d\)};
	\vertex[right=5cm of b3] 	(b4) 	{\(d\)};
	
	\vertex[below=1em of b3] 	(b5) 	{\(d\)};
	\vertex[right=2.5cm of b5] 	(b6);
	\vertex[below=1em of b4] 	(b7) 	{\(u\)};
	
	\vertex[below=6em of b5] 	(c1) 	{\(d\)};
	\vertex[right =2.5cm of c1] (c2);
	\vertex[below=6em of b7]	(c3) 	{\(u\)};
	
	\vertex[below=1em of c1] 	(c4) 	{\(d\)};
	\vertex[right=5cm of c4] 	(c5) 	{\(d\)};
	
	\vertex[below=1em of c4] 	(c6) 	{\(u\)};
	\vertex[right=5cm of c6] 	(c7) 	{\(u\)};
	
	\vertex[below=2em of b7]	(e1)	{\(e^-\)};
	\vertex[left=2cm of e1]		(e2);
	\vertex[below=2em of e2]	(e3);
	\vertex[below=2em of e1]	(e4)	{\(e^-\)};
	
	\diagram
	{
		{[edges=fermion]
			(b1) -- (b2),
			(b3) -- (b4),
			(b5) -- (b6),
			(b6) -- (b7)
			(c1) -- (c2),
			(c2) -- (c3),
			(c4) -- (c5),
			(c6) -- (c7),
			(e2) -- (e1),
			(e3) -- (e4),
		},
		(b6) -- [boson, edge label'=\(W\)] (e2),
		(c2) -- [boson, edge label=\(W\)] (e3),
		(e2) -- [insertion=0.5] (e3),
	};
	\draw [decoration={brace}, decorate]  (b5.south west)--(b1.north west) node [pos=0.5, left] {\(n\)};
	\draw [decoration={brace}, decorate]  (c6.south west)--(c1.north west) node [pos=0.5, left] {\(n\)};
	\draw [decoration={brace}, decorate]  (c3.north east)--(c7.south east) node [pos=0.5, right] {\(p\)};
	\draw [decoration={brace}, decorate]  (b2.north east)--(b7.south east) node [pos=0.5, right] {\(p\)};
	\end{feynman}
	\end{tikzpicture}
	\subcaption{Feynman-Diagram of the $0\nu\beta\beta$-decay. Two neutrons transfer into two protons and two electrons. One probability is that it is mediated via the exchange of a massive Majorana neutrino. This transition has not been detected yet.}
	\label{fig:Feyn0nbb}
\end{subfigure}
\caption{Feynman-Diagrams of the $0\nu\beta\beta$- and the $2\nu\beta\beta$-decay}
\end{figure}
\WarningsOn


In the case of a \onbb\ decay, no neutrinos are released:
\begin{equation}
(A,Z)\rightarrow (A,Z+2) + 2e^- 
\end{equation}

From here, it will be assumed that the \onbb\ decay is caused by an exchange of massive Majorana neutrinos as seen in \ref{fig:Feyn0nbb}.
The non zero mass is due to the dependence of the \onbb\ decay half life \thalfzero\ on the effective neutrino mass $\left\langle m_{\beta\beta}\right\rangle^2$ ($1 / \thalfzero\ \propto \left\langle m_{\beta\beta}\right\rangle^2$) \cite{vergados_theory_2012}.
This effective neutrino mass is defined by 
\begin{equation}
\left\langle m_{\beta\beta}\right\rangle = \abs{\sum_i U_{ei}^2m_i }
\label{mBetaBeta}
\end{equation}
The PMNS mixing matrix elements $U_{\alpha i}$ originate from neutrino flavor mixing. 
Neutrino oscillation experiments have shown that neutrino flavor eigenstates $\nu_\alpha$ that couple to the W boson are a superposition of neutrino mass eigenstates $\nu_i$.
\begin{equation}
\ket{\nu_\alpha} = \sum_i U^*_{\alpha i} \ket{\nu_i} 
\end{equation}
The value of each $U_{\alpha i}$ is defined by three mixing angles and three CP-violating phases.
The Dirac phase is always present, regardless whether the neutrino is a Majorana or Dirac particle.
The two Majorana phases only have a physical meaning for Majorana neutrinos.
Neutrino oscillation is only sensitive to the mass square differences $\Delta m^2_{ij} = m^2_i - m^2_j$ of the mass eigenstates and so far only an absolute value of $\Delta m^2_{32}$ could be determined.
This leaves three different possible scenarios to be considered in the \onbb\ decay:
\begin{enumerate}
    \item normal order ($m_1 \ll m_2 \ll m_3$)
    \item inverted order ($m_3 \ll m_1 < m_2$)
    \item quasi-degenerate order ($m_1 \approx m_2 \approx m_3$) in which the mass eigenvalues are much larger than the mass differences
\end{enumerate}
The effective neutrino mass as a function of the lightest mass of each mass ordering can be seen in figure \ref{fig:MassOrder}.
A \onbb\ decay may  provide new knowledge about the mass ordering, the mass scale as well as the Majorana phases. 
\\

Other attempts to measure the mass of the electron neutrino are measuring the end point of the beta decay. 
A promising experiment right now is the KATRIN experiment.
The absolute mass these kinds of experiments can measure, is an incoherent mass sum as seen in equation \ref{massBeta}.

\begin{equation}
\left\langle m_{\beta}\right\rangle = \sqrt{\sum_i \abs{U_{ei}}^2m^2_i}
\label{massBeta}
\end{equation}

A third method involves cosmological observation in which the sum of the neutrino masses $\Sigma$ can be measured:
\begin{equation}
\Sigma = \sum_i m_i
\end{equation}
From these three masses it is possible to determine the Majorana phases, the individual masses of the mass eigentstates and therefore also the mass order of the neutrinos.
\\

\begin{figure}[t!]
	\centering
	\begin{minipage}[t]{.475\textwidth}
		\centering
		\includegraphics[width=.825\textwidth]{./Bilder/NeutrinoMassOrdering.png}
		\caption{Effective neutrino mass $\left\langle m_{\beta\beta}\right\rangle$ as a function of the smallest mass of the respective mass hierarchy. NS stands for the normal order and IS for the inverted order. Taken from \cite{bilenky_neutrinoless_2012}.}
		\label{fig:MassOrder}
	\end{minipage}\hfill%
	\begin{minipage}[t]{.475\textwidth}
		\centering
		\includegraphics[width=\textwidth]{./Bilder/TheoretischesSpektrmdes0nubbDecay.png}
		\caption{Effective spectrum of a \twonu\ decay when adding the energies of the two escaping electrons.  Taken from \cite{elliott_double_2002}.}
		\label{fig:TheoSpektrum}
	\end{minipage}
\end{figure}

\subsection{Experimental methods to detect a \onbb\ decay}

The experimental signature of the \onbb\ decay is a sharp peak at the $Q_{\beta\beta}$ value of the \twonu\ decay in the effective spectrum of the two electrons (see figure \ref{fig:TheoSpektrum}).
One experimental approach consist in using a detector made of material enriched in a \onbb\ decaying isotope.
This has the advantage that the detection efficiency is maximized.
\\

Since \onbb\ decay should have a very long half-life, any background has to be minimized just to measure the resulting influence.
There are three kind of background sources which have to be considered.
\begin{enumerate}
    \item Cosmic background.
Muons and other particles shower down to the earth from the atmosphere and create background in the detectors.
Most of their influence can be suppressed by placing the experiment deep underground.
By also applying a muon veto system the most of the residual muon flux can be suppressed.
\item Natural radioactivity.
This is typically the dominant background source originating from radioactive isotopes which are naturally present in all materials.
Its influence can be suppressed by passively shielding the detectors and by the selecting of low radioactive components in the setup. 
\item The \twonu\ decay material itself.
Its impact on the background cannot be reduced by external measures or shielding.
However, by using a detector with high energy resolution or a decay material of high Q-value, its influence can be suppressed.
\end{enumerate}
\\

\nuc{Ge}{76} is often used in \onbb\ decay search experiment. 
As it is a semiconductor, it can be used as detector material itself.
A disadvantage of \nuc{Ge}{76} is its low Q-value of $Q_{\beta\beta} = 2039\unit{keV}$, which is lower than \nuc {Tl}{208}'s and \nuc{Bi}{214}'s end point energy.
It is also hard to increase the target volume compared to e.g. \nuc{Xe}{136}.
Its advantages, however, are its ability to be made with great intrinsic radio-purity, its high energy resolution and its high detection efficiency.
These facts outweigh the disadvantages which is the reason why it already has a long history for being used as decay material and why it was chosen for the \gerda\ experiment.
\\

%\nuc{Ge}{76} has a long history of use as decay material in \onbb\ experiments, most notably in the Heidelberg-Moscow(HdM) and the IGEX experiments.
%Both of these experiments are the predecessor experiments of \gerda\.
%With detectors made of enriched germanium plus the background reducing precautions and active vetos described above they were able to set a similar limit of the half life of the \onbb\ to $\thalfzero(\nuc{Ge}{76}) > 1.9\times10^{25}\unit{yr}$ \cite{noauthor_phys._nodate}. 
%HdM experiment actually claimed to have measured the half life of about $\thalfzero(\nuc{Ge}{76}) = 1.19\times10^{25}\unit{yr}$ , but its legitimacy has been questioned by a part of the scientific community \cite{klapdor-kleingrothaus_search_2004}.
%\\

%This is where the  \gerda\ experiment comes in.
%Its first measuring phase (\PI) was planned to verify or falsify the results of the HdM experiment using the detectors used in the HdM and the IGEX experiment.
%\PI started in November 2011 and May 2013 with a total exposure  of  $21.6 \frac{\unit{kg}}{\unit{yr}}$ and no signal of a \onbb\ observed \cite{agostini_results_2013}.
%With its results and the results from HdM and IGEX a new lower limit for the half life was able to be set at $\thalfzero(\nuc{Ge}{76})>3.0\times10^{25}$  yr  (90$\%$ C.L.)
%The second phase (\PII) with 30 new custom-made enriched detectors together with the old detectors has started measuring in late 2015 and had its latest data published !!!! hier noch ob ich das NATURE paper quoten soll!!!!!.
%A more detailed description of the construction and functionality of the \gerda\  \PII~ is the topic of the next section.

% also a bit about standard double beta decay
% differences between the standard and neutrinoless beta decay
 


\section{\gerda\ \PII}
\label{sec:gerda}

% general Information, e.g. 
% sizes 
% Gran Sasso, 
% what other  neutrino less beta decay experiments are there,

\subsection{Experimental Setup}
\label{sec:ExSetup}
The \gerda\ experiment is located in the underground Laboratori Nazionali del Gran Sasso (LNGS) of INFN, Italy.
The laboratories are located approximately 1.4 km below ground, which corresponds to a water equivalent of 3.5 km.
\gerda\ uses \nuc{Ge}{76} as \onbb\ source as well as the detector material \cite{agostini_background_2017}.
\\

The detectors are operated bare in a liquid argon (LAr) tank of 64m$^3$ volume at an working temperature of about 90K.
Its main purpose is to cool the germanium detectors down to their working temperatures and to passively shield them against external radiation originating from the outside.
LAr can also scintillate.
This is why in \PII, extra instrumentation was positioned inside the LAr tank in order to measure any light signal around the detectors.
As any \nuc{Ge}{76} decays are unlikely to create scintillation light, their signal can be used as a veto - the so-called LAr veto.

Situated around the LAr cryostat is a 590m$^3$ water tank.
Its main purpose is to shield the setup from outside radiation not only passively by absorption but also actively as a muon veto.
Situated in the water tank are 66 photomultipliers.
They detect Cherenkov light created by Muons.

Above the water tank, a clean room is installed in which the detector strings are assembled.  
The general setup can be seen in figure \ref{fig:gerdaSetupPII}.
\\

\begin{figure}[t!]
	\centering
	\begin{minipage}[t!]{.45\textwidth}
		\centering
		\includegraphics[height=60mm]{./Bilder/GERDAsetupPhaseII.png}
		\caption{Sketch of the \gerda\ \PII's experimental setup. The germanium detector array is placed inside a liquid argon (LAr) cryostat which itself is surrounded by a water tank. Taken from \cite{collaboration_upgrade_2018}.}
		\label{fig:gerdaSetupPII}
	\end{minipage}\hfill%
	\begin{minipage}[t!]{.45\textwidth}
		\centering
		\includegraphics[height=60mm]{./Bilder/DetectorDesign.png}
		\caption{Sketch of the Semi-Coaxial (COAX) and Broad Energy Germanium (BEGe) detector designs. Taken from \cite{agostini_background_2014}.}
		\label{fig:DetcDes}
	\end{minipage}
\end{figure}

Regarding the detectors themselves, \gerda\ \PII\ uses seven semi-coaxial detectors (COAX) which have already been used in the predecessor experiments (Heidelberg-Moscow and IGEX) as well as 30 new Broad Energy Germanium detectors (BEGe).
Both detector types are made of germanium that has been enriched from 7.8$\%$ of \nuc{Ge}{76} to about 87$\%$ \cite{agostini_background_2017}.
They also share the same basic functionality.
They are made from p-type germanium material.
Both detectors have the majority of their surface covered by a 1-2mm thick n$^+$ doped electrode and only a small part with a p$^+$ electrode.
If an electron hole pair is created in the p doped area, the charged cariers are separated and guided to the n$^+$ or the p$^+$ layer respectively by a strong electrostatic field (3 to 4 kV) between the electrodes  \cite{spieler_semiconductor_2005}.
If the pair is created in the n$^+$ layer, the hole is most likely to recombine in the $n^+$ layer due to its low mobility and, thus, creates no measurable signal.
The n$^+$ layer is therefore not active and called a dead layer.
\\

The two detectors, however, differ in their design as seen in figure \ref{fig:DetcDes}.
Moreover the detector types also differ in their mass and energy resolution.
But their design also results in a worse energy resolution due to their higher capacity compared to the BEGes \cite{agostini_production_2015}.
BEGes also allow for better pulse shape discrimination (PSD) compared to the COAX detectors \cite{agostini_pulse_2013}.
PSD can make a statement about the creation process of the electron hole pair by looking at the shape of the signal.
It can therefore discriminate between different event topologies and acts as a veto.
\\

The enriched detectors are assembled into 6 strings forming a hexagonal array together with a seventh string.
This seventh string consists of three extra coaxial detectors made of natural isotopic germanium.
However, they are not used in \gerda's main analysis.

Custom-made amplifiers also located in the liquid argon above the detectors, the analog signals of the detectors are digitized at a sampling rate of 100MHz if a triggering signal was found \cite{riboldi_cryogenic_2015}.
Every 20 seconds a charge pulse, called the test pulse, is injected into the the front-end electronics.
Its purpose is to monitor the stability\cite{agostini_background_2017}.
The analysis of the signals is performed off-line.


\iffalse
\begin{figure}[t!]
	\centering
	\begin{subfigure}{.66\textwidth}
		\centering
		\includegraphics[width=.9\textwidth]{./Bilder/strings.png}
		\caption{}
		\label{fig:strings}
	\end{subfigure}\hfill%
	\begin{subfigure}{.30\textwidth}
		\centering
		\includegraphics[width=.9\textwidth]{./Bilder/strings-top.png}
		\caption{}
		\label{fig:stringsabove}
	\end{subfigure}
\end{figure}
\fi
% well, just dump everything here
% also a bit about Tier1-4 storage of data 

Surrounding the detectors so-called nylon mini shrouds (NMS) are attached to limit the amount of LAr volume around the detectors.
These NMS are placed there to passively suppress the background created from \nuc{K}{42}.

\subsection{Liquid Argon as Coolant, Shielding and Scintillator} 
\label{sec:LArcoolant}


LAr is also a good coolant due to its low boiling point.
In \gerda\, it is cooled down to the working temperature of the germanium detectors at about 90 K.
LAr can also be produces with a very high purity by air separation and further distillation.
It also has good shielding capabilities !!!!!
It can therefore be used as an ultra radio-pure passive shielding, as noble elements rarely make any chemical bonds due to their fully filled electron shells.
It is therefore possible to filter the majority of impurity just by distillation.
Compared to other noble gases, argon also has the advantage of a very low price being easily obtained by liquefaction of air from the atmosphere.
\\

LAr has the property that scintillation light is produced when radiation excites or ionizes atoms in the material \cite{olsen_improvements_nodate}.
The excited atoms in noble liquids form dimer pairs (Ar$^*_2$).
These dimer pairs are metastable and relax with the release of a vacuum ultraviolet photon (VUV)  which have a wavelength of 128 nm.
\\

\begin{figure}[t!]
	\centering
	\begin{minipage}{.475\textwidth}
		\centering
		\includegraphics[width=0.6\textwidth]{./Bilder/LArVetoSetup.png}
		\caption{Sketch of the liquid agron (LAr) veto setup. 
        In it 16 photomultiplier tudes (PMT) and 90 silicon photomultiplers (SiPM) are installed. 
        \iffalse
        16 photomultiplier tubes (PMT) are mounted in a cylindrical copper surface at the top and bottom. 
        At the level of the detectors, the cylinder consists of fibers instead of copper. 
        They are read out at their ends by silicon photomultipliers (SiPM). 
        \fi
        Taken from \cite{collaboration_upgrade_2018}.}
		\label{fig:LArVetoSetup}
	\end{minipage}\hfill%
	\begin{minipage}{.475\textwidth}
		\centering
		\includegraphics[width=\textwidth]{./Bilder/GerdaErgebnisse.png}
		\caption{
        Recent results from \gerda\ \PII. 
        The measured spectra of COAX and BEGe detectors are separately displayed.  
        \iffalse
        Only one decay 2 sigma from the \onbb\ decay was found. 
        This leads to the conclusion that no \onbb\ decay was measured and a new lower limit of \thalfzero\ = $0.8\times10^{26}\unit{yr}$ (90$\%$ CL) was determined. 
        \fi
        Taken from \cite{zsigmond_new_2018}}
		\label{fig:gerdaErgebnisse}
	\end{minipage}
\end{figure}

Background events often deposit energy in the argon while passing through it.
Their scintillation light around the detectors can therefore be used as a veto to reject those events.
The LAr veto system consists of a cylindrical copper shell around the germanium detectors is equipped with 16 photomultiplier tubes (PMT), situated at the bottom and at the top of this volume.
Also, at the level of the detectors the shell is not made of copper but of wavelength shifting fibers.
These are read out by 90 silicon photomultipliers (SiPM)\cite{csathy_optical_2016} (see figure \ref{fig:LArVetoSetup}).
VUV has a wavelength which is so small most material absorbs it in ionization.
This is why wavelength shifting material covers the surface of the LAr veto shifting from 128nm to 400nm.
\\


\subsection{Data processing and analysis}
\label{sec:DataProc}

\label{sec:Resultsofgerda}
\begin{figure}[t!]
	\centering
		\includegraphics[width=100mm]{./Bilder/TierStructure.png}
		\caption{The multi-tier structure used by GELATIO. Tier0 and Tier1 contain the entire raw data. However, the data in Tier1 has already been converted to root files. Tier2, Tier3 and Tier4 store  progressively more analyzed data. Taken from \cite{agostini_gelatio:_2011}}
		\label{fig:TierStructure}
\end{figure}

As already mentioned, in the case of an event the digitized signals of the germanium detectors and the photomultipliers are stored for further off-line analysis.
The software used for this purpose is \gerda\ LAyouT for Input/Output (GELATIO).
It is a data analysis framework suitable for off-line digital signal processing and analysis of data recorded by germanium detectors.
Its advantages lie in its multiple level data organization as seen in figure \ref{fig:TierStructure}.
Tier0 and Tier1 store raw information whereas all higher Tiers contain progressively more analyzed data.
Additional information about the analysis process can be found in \cite{agostini_gelatio:_2011} and \cite{agostini_off-line_2011}.
In this thesis only data from Tier3 and Tier4 are used.
\\
\iffalse
To ensure that the filter parameters of the analysis are set without bias, all events measured in a  50 keV interval around the $Q_{\beta\beta}$ were only stored without any analysis applied on them.
Only after all parameters in the rejection process were finally defined all events from this interval will be analyzed. 
Such a procedure is called a blinding process and the revealing of the blinded events an unblinding. 
\\
\fi
% Mui, Mountain, Pulse shape disc.
% especially LAr-Veto 

\subsection{Recent Results}

\begin{figure}	
		\centering
	\includegraphics[width=\textwidth]{./Bilder/Kr85Decay.png}
	
	\caption{
	The decay scheme of \Kr. 
    \iffalse
    It decays via two different channels. 
    The majority of transitions end in the ground state of \nuc{Rb}{85} while 0.434$\%$ of the time \Kr\ decays into an excited state being 514 keV over the ground state. 
    The excited state has a half life of 1.015$\mu$s. It
    then relaxes directly into the ground state. 
    \fi
    Taken from \cite{noauthor_livechart_nodate}.
	}
	\label{fig:Decay}
\end{figure}
Only recently, new data was published by \gerda\ \cite{zsigmond_new_2018}.
In it, only one event in the proximity of the Q-value was found as shown in figure \ref{fig:gerdaErgebnisse}.
However, more than 2 sigma from the expected peak position.
The conclusion was therefore that no evidence for a \onbb\ decay has been seen.
A new lower limit for \nuc{Ge}{76} was determined to be $\thalfzero\ = 0.9\times10^{26}\unit{yr}$ (90$\%$ CL).
Currently \gerda\ receives an upgrade and is planned to measure until 2020.
After that, the successor experiment LEGEND is planned to further investigate the lower limit of \nuc{Ge}{76} half life.

% what has happened so far

% short outline of approaches


\section{\Kr\ isotope in the atmosphere}
\label{sec:Kry85}

\Kr\ has a mass number of A = 85 and an atomic number of Z = 36.
This isotope is not stable and decays via a $\beta^-$-decay into \nuc{Rb}{85}.
The half life of this decay is 10.756 yr \cite{singh_nuclear_2014} and a Q-value of $Q_\beta = 687 \unit{keV}$.
\Kr's $\beta$-decay has two different possible transitions (see figure \ref{fig:Decay}).
With overwhelming 99.563$\%$, the \Kr decay has its final state in the ground state of \nuc{Rb}{85}.
In 0.434$\%$, however, it decays via an excited state of \nuc{Rb}{85} 514 keV above the ground state.
This excited state has a half life of 1.015 $\unit{\mu s}$ and relaxes directly into the ground state while emitting  a photon carrying the energy difference.
\\

\begin{figure}[t!]
	\centering
	\ifmakefigures%
	\includegraphics[width=80mm]{./Bilder/Kr85Aenderung.png}
	\fi%
	\caption{
	    \Kr's specific activity in the atmosphere between 1945 to 2009. It can be seen that over this period the average specific activity in the atmosphere has increased. 
		Taken from \cite{ahlswede_update_2013}.
	}
    \label{fig:Kr85Aenderung}
\end{figure}

The argon used in \gerda\ \PII\ was extracted from the atmosphere.
When argon is separated from the other components of the air, argon can easily be made radioactively pure.
Nevertheless, a very small portion of alien elements like krypton can still be present in the extracted argon and therefore also \Kr\.  
But this would also mean that any residual \Kr\ in the argon must also have already present in the atmosphere.
\\

When investigating on how  \Kr\ got into the atmosphere, two different sources on earth can be identified.
On the one hand side, it can be created naturally in the atmosphere by an interaction of \nuc{Kr}{84} with cosmic rays.
On the other hand, the production of \Kr\ from nuclear fission of \nuc{U}{235} and \nuc{Pu}{239} generates an atmospheric inventory about four orders of magnitude higher than that.
Only a small amount is naturally produced in earth's crust.
The majority is man-made from nuclear power plants \cite{winger_new_2005}.
Krypton is a noble gas and therefore easily diffuses through everything in its way.
It rises until it reaches the atmosphere where a \Kr\ reserve is build over time.  
Due to the higher amount of nuclear power plants build in the last half century the\Kr\ activity in the atmosphere has risen from about 1961 PBq in 1973 \cite{telegadas_atmospheric_1975} to about 5500 PBq in 2009 (see figure \ref{fig:Kr85Aenderung}) \cite{ahlswede_update_2013}.
Compared to other radioactive noble gas isotopes like \nuc{Ar}{39}, however, the\Kr\ still only has a small specific activity in the atmosphere which is why its change is not of great concern.
\\

\iffalse
The concentration of \Kr\ in the air at ground level varies depending on the proximity to a nuclear power plant \cite{weiss_mesoscale_1986} but at atmospheric altitudes it only varies on large length scales.
This change is so small, however, that one can assume \Kr\'s concentration is everywhere the same. 
\\
\fi

Two other experiments also using LAr are the WARP and the Darkside experiment.
In both of experiments the specific activity of their residual \Kr\ was determined.
The Darkside experiment, using underground argon (UAr), has measured a specific activity of \((2.86\pm0.18) \frac{\unit{mBq}} {\unit{l}}\)  \cite{agnes_results_2016}.
This UAr has been extracted from underground reservoirs and should only have come into contact with \Kr\ from natural processes.
In the LAr of the WARP experiment a specific \Kr\ activity of   \((160\pm130)\frac{\unit{mBq}}{\unit{l}}\) was measured \cite{benetti_measurement_2006}.
It uses atmospheric argon which is likely why it has a higher specific activity compared to  Darkside.
\gerdas\ value is therefor also expected to be comparably high but up until now this value was unknown.
The aim of this thesis is to determin the specific \Kr\ activity in the LAr of \gerda\ \PII.
\iffalse
Its concrete value would be of interest when researching in the lower energy spectrum of \gerda\.
For example for a further investigation on \nuc{Ar}{39} that has a similar Q-value or the search for dark matter in the keV-scale.
\fi
% where does it come from?
% what properties does it have?
% why is it important to calculate its influence on \gerda\

 
\chapter{Line Count Rate Analysis}
\label{sec:SAfrom514}

The first and more precise method uses the 514 keV line count rate of the \Kr\ decay.
As discussed in section \ref{sec:Kry85}, \Kr\ has a small probability of $p=0.434\%$ to decay into an excited state of \nuc{Rb}{85m}. 
When \nuc{Rb}{85m} relaxes into its ground state, it emits a photon of 514 keV energy.
The counts $N_{\mathrm{peak}}$ in the 514 keV line in the \gerda\ spectra would therefore allow to draw a conclusion concerning the amount of \Kr. 
\\

An important factor for the calculation is the efficiency of the used germanium detector to fully absorb these 514 keV gammas.
For this, a Monte Carlo simulation is necessary in which $N_{\mathrm{sim}}$ gammas with an energy of 514 keV  in a  volume $V_{\mathrm{sim}}$ are simulated.
The detector efficiency can then be calculated from dividing the simulated line count at 514 keV by the total number of  decays ($\epsilon = \Delta N/N_{\mathrm{sim}}$).
The value $1/p \epsilon V_{\mathrm{sim}}$, using the detector efficiency, the simulated volume and the probability $p$ of \Kr\ to decay via the excited \nuc{Rb}{85m}, is a conversion factor from a measured line count rate to the density of decays necessary to create this signal.
\\

The final value needed is the experimental time $\bar{t}$.
Not every detector was measuring over the course of Phase II.
This is why a mean measuring time for all detectors has to be calculated.
With these three values, a mean specific activity $\bar{a}$ can be determined:

\begin{equation}
    \bar{a} = N_{\mathrm{peak}}\times\frac{1}{p \epsilon V_{\mathrm{sim}}}\times\frac{1}{\bar{t}}
    \label{equ:activityDieErste}
\end{equation}
The line count rate analysis is expected to generate a relatively precise estimation of the specific activity.
This is due to the 514 keV line being a clear feature which can only be traced back to \Kr.
\\


\iffalse
However, a problem of this method lies with the proximity of the \Kr\ to the 511keV peak of the positron electron annihilation. 
Its peak in the energy spectrum is expected to partially dominate over \Kr\ and does not allow for a direct measurement of the 514keV peak. 
This is not necessarily a great setback because one can just adapt the fit function to a double Gaussian peak function.
It is of interest, however, whether it is possible to completely suppers the annihilation peak without changing the 514keV photon line count.
For this one could consider using the LAr veto.
Due to the low mean energy of the escaping electron (47.65keV) of this decay, it is very unlikely that it creates any scintillating light. 
On the other hand one can expect the light of the positron electron annihilation to create a great signal in the photomultipliers.
Therefore it should be possible with the LAr veto to single out the 514keV photon events from the annihilation events.
If possible its value can be used as a cross-check for the value determined from the not filtered spectrum.

The rest of the chapter will cover the concrete implementation of the individual steps in their own sections.
\\
\fi
\section{Preparing the spectrum}

To determine the amount of measured 514 keV gammas, a fit has to be applied onto the corresponding peak in the energy spectrum.
The data used in this analysis is the fully available \gerda\ \PII\ data (run 53 to 92).
The standard \gerda\ analysis cuts where applied.
This includes data quality cuts, the Muon veto cut and the anti-coincidence cut between germanium detectors.
\\

The data was also split for the two detector types in \gerda.
This is necessary due to the differences in detector efficiency and resolution already mentioned in section \ref{sec:ExSetup}.
BEGes have a lower efficiency to detect full 514 keV gammas but show a higher energy resolution ($\Delta E_{\mathrm{BEGe}} = 2.267\unit{keV}  @ 514 \unit{keV}$) and vice versa for the COAX detectors ($\Delta E_{\mathrm{COAX}} = 2.720\unit{keV} @ 514 \unit{keV}$).
These values were determined in the appendix chapter \ref{sec:ResDetermination}.
The exposure of all BEGe detectors in the entire \PII\ is 30.8 kg$\cdot$yr and 28.1 kg$\cdot$yr for the COAX detectors as calculated in section \ref{sec:CalcActiv}.
\\

\begin{figure}[t!]
\centering
\begin{subfigure}{.475\textwidth}
  \centering
	\includegraphics[width=75mm]{./Bilder/500525NoFilterBEGes.pdf}

  \caption{BEGe}
    \label{fig:NoFilterBEGes}
\end{subfigure}\hfill%
\begin{subfigure}{.475\textwidth}
  \centering
	\includegraphics[width=75mm]{./Bilder/500525NoFilterCOAX.pdf}
  \caption{COAX}
  \label{fig:NoFilterCOAX}
\end{subfigure}
	\caption{Energy spectra from 500 to 525 keV after standard \gerda\ analysis cuts, split by the respective detectors where the signal was measured in.}
\end{figure}

In figure \ref{fig:NoFilterBEGes} and \ref{fig:NoFilterCOAX} two resulting spectra from 500 to 525 keV of the respective detectors can be seen.
A structure deviating from the background level can be identified  at 511 keV and 514 keV .
The 511 keV line originates from gammas created in positron electron annihilation events whereas the 514 keV line leads to the presumption that gammas are being created in the \Kr\ decays via the excited \nuc{Rb}{85m} state.
Analyzing these figure one can already make the statement that a non negligible amount of \Kr\ must be present in the LAr.
\\

\iffalse
After the adjusting the spectra to a lower background level, one can now determine more precisely the number of measured events in the 514keV peaks of the corresponding detectors (see figure \ref{fig:NoFilterBEGes} for the BEGe and \ref{fig:NoFilterCOAX} for the spectra of the COAX detectors). 
In these two spectra one can see two peaks - one at 511keV that corresponds to the positron electron annihilation events and one at 514keV that corresponds to the photons from the relaxation of \nuc{Rb}{85m}.
From this we can already claim that there must be a non negligible amount of \Kr\ in the liquid argon.
Otherwise no peak should have been able to be measured. 
The difference in resolution as discussed above can be seen in the fact that in the BEGe diagram the two peaks have a smaller full width at half maximum (FWHM).
Compared to the COAX detectors their peaks can easily be distinguished.  
\fi




\iffalse
Another possible approach suppress the annihilation peak using an almost ideal filter and fit the resulting one peak spectrum with the original fit function.
As mentioned above the LAr Veto should be a good candidate for such a filter.

In this thesis both approaches will be applied separately and later their results compared.
Hopefully both will end up delivering the same result as no \Kr\ caused event should make a notable light signal.
But this probability is not zero which is why some events of the 514keV peak might also trigger the veto.
This would result in a smaller peak amplitude and with it a lower specific activity than the actual value. 
Whether or not a rejection process using the LAr veto would be useful in this analysis or not is the topic of the following section.
\\
\fi

\section{Annihilation Peak Suppression}
\label{sec:APS}

Due to the occurrence of the 511 keV peak, two different approaches are now possible.
One approach would directly apply a double Gaussian peak fit function to the spectra also fitting the 511 keV peak.
Another approach would try to suppress the 511 keV peak and fit a single peak function through the resulting spectra.
Both approaches will be applied in this thesis.
\\

A promising candidate in order to suppress the peak, is the LAr veto.
It is triggered in the case an event in the germanium detectors coincides with a scintillation signal of at least ~0.5 phe \cite{agostini_background_2017}.
Again, the \Kr\ decay into the excited \nuc{Rb}{85m} leaves the escaping electron with a maximum of 173 keV which only creates a very weak scintillation light signal.
Furthermore, this beta induced light signal could be measured before the standard veto time window.
Any process involving positron annihilation, however, should create a measurable scintillation light signal visible to the LAr veto setup.
Therefore, the LAr veto should be able to discriminate between \Kr\ gamma events and annihilating events.
\\

Figures \ref{fig:LArBEGes} and \ref{fig:LArCOAX} show the energy spectra from 500 to 525 keV after the LAr veto.
One can see that the annihilation peak is reduced compared to the spectra before the LAr veto.
\\

\begin{figure}[t!]
\centering
\begin{subfigure}{0.475\textwidth}
	\includegraphics[width=75mm]{./Bilder/500525LArVetoBEGes.pdf}
    \caption{BEGes}
  \label{fig:LArBEGes}
\end{subfigure}\hfill%
\begin{subfigure}{0.475\textwidth}
	\includegraphics[width=75mm]{./Bilder/500525LArVetoCOAX.pdf}
  \caption{COAX}
  \label{fig:LArCOAX}
\end{subfigure}
    \caption{Energy spectra from 500 to 525 keV after standard \gerda\ analysis cuts and LAr veto.}
\end{figure}

\begin{figure}[t!]
	\centering
	\begin{subfigure}{.5\textwidth}
		\includegraphics[width=75mm]{./Bilder/AntiLArBEGe.pdf}
		\caption{BEGes}
		\label{fig:AntiLArBEGes}
	\end{subfigure}\hfill%
	\begin{subfigure}{.5\textwidth}
		\includegraphics[width=75mm]{./Bilder/AntiLArCOAX.pdf}
		\caption{COAX}
		\label{fig:AntiLArCOAX}
	\end{subfigure}
	\\
	\vspace{0.5cm}
	\caption{Energy spectra from 500 to 525 keV }
\end{figure}

In order to have a better visualization which events have been rejected, all events that triggered the LAr veto are plotted in figures \ref{fig:AntiLArBEGes} and \ref{fig:AntiLArCOAX}.
You can see that the majority of the filtered events had an energy around the 511 keV mark.
\\

Another way to possibly single out the \Kr\ decay caused events would be to look for pre-coincidence events in the LAr setup.
As the excited \nuc{Rb}{85} nuclei has a half life of 1.015 $\mu$s, the scintillation light created by the released beta electron is very probable to be measured before the standard veto time frame looking for a coincidental light signal.
It should therefore also be possible to distinguish \Kr\ decays from other events if it is possible to find these pre-coincidences.
\\


\begin{figure}[t!]
	\centering
	\ifmakefigures%
	\includegraphics[width=100mm]{./Bilder/BeispielSignal.pdf}
	\fi%

	\caption{
    The recorded signal of photomultiplier tube P4 in event 1614036. 
    In it is the raw energy signal plotted over the recorded time frame. 
    The blue line indicates the moment in time in which an event in one of the germanium detectors was measured. 
    From it one can see that the photomultiplier signal occurred before the germanium detector signal.   
    }
    	\label{fig:BeispielSignal}
\end{figure}


An investigation to find out whether the pre-coincidence signals are usable was applied by plotting all raw measured LAr veto signals of events with an energy around the 514 keV peak and by manually search for any pre-coincidences.
The expected signatures consist of a single peak in the PMTs (for example as seen in figure \ref{fig:BeispielSignal}) or a rising edge in the SiPMs respectively.
However, this investigation showed that almost none of the events considered had any kind of indication for pre-coincidences.
This excludes the presented approach as being a useful \Kr\ indicator.

\iffalse

\fi

\section{Fitting}
\label{sec:Fitting}

The spectra from before and after the LAr veto can now be fitted and, from the fit parameters, the amount of measured \Kr\ decay events can be determined.
\\

The spectra before the LAr vetoes \ref{fig:NoFilterBEGes} and \ref{fig:NoFilterCOAX} show two peaks.
This requires the fit function to include two Gaussian functions from which only the parameters of the second peak will then be used for further analysis.
In addition to the two Gaussian peaks, a constant background parameter will be added.
Theoretically, the \nuc{Ge}{76} spectra creating the background changes with energy.
In the investigated area, however, its overall change very small that it can be approximated as constant. 

\iffalse
When looking at the not LAr veto filtered spectra we can see that over the course of the displayed energy interval no real change of the background can be seen (see figure \ref{fig:NoFilterBEGes} and \ref{fig:NoFilterCOAX}).
Theoretically however, one can expect the background count rate to behave like the phase-space function of the dominant \nuc{Ge}{76} and therefore change with energy.
So it is necessary to also consider this change in count rate over energy in the fit function.
The phase-space function of \nuc{Ge}{76} is very complex.
In this case however the energy interval is relatively small compared to the complete spectrum.
This allows the approximation that its phase-space function changes like an exponential decrease.
\fi

The resulting fit function is displayed in equation \ref{equ:FitNoFilters}.
\\

\begin{equation}
\mathrm{f}(x) = \mathrm{A}\frac{1}{\sqrt{2\pi}\mathrm{C}}\exp\left(-\frac{(x-\mathrm{B})^2}{2\mathrm{C}^2}\right) + \mathrm{D}\frac{1}{\sqrt{2\pi}\mathrm{C}}\exp\left(-\frac{(x-\mathrm{E})^2}{2\mathrm{C}^2}\right) + \mathrm{G}
\label{equ:FitNoFilters}
\end{equation}
\\

Figures \ref{fig:FitNoFilterBEGes} and \ref{fig:FitNoFilterCOAX} show the resulting fit plots.
The values of the fit parameter values can be seen in the table of the figures.
Fitting parameters B and E - the mean values of the peaks - where fixed to the values of 511 keV and 514 keV respectively.
All other values were left free under the condition of a positive value.
Also, Parameter C - the sigma value - was used in both peaks being resolution of the detectors.
In the investigated energy range, these two parameters should not vary much which is why this simplification can be applied.
Their expected values derived from the resolutions of the detectors at an energy of 514 keV are $\sigma_{\mathrm{BEGe}} = 0.96 \unit{keV}$ for the BEGe detectors and  $\sigma_{\mathrm{COAX}} = 1.16 \unit{keV}$ for the COAX detectors (see appendix chapter \ref{sec:ResDetermination}).
The values determined in the fitting process have the corresponding expected value in their range of uncertainty .
\\

\begin{figure}[t!]
	\centering
	\begin{subfigure}{.5\textwidth}
		\centering
		\includegraphics[width=\textwidth]{./Bilder/500525FitNoFilterBEGes.pdf}
		\caption{BEGes}
		\label{fig:FitNoFilterBEGes}
	\end{subfigure}\hfill%
	\begin{subfigure}{.5\textwidth}
		\centering
		\includegraphics[width=\textwidth]{./Bilder/500525FitNoFilterCOAX.pdf}
		\caption{COAX}
		\label{fig:FitNoFilterCOAX}
	\end{subfigure}
    \\
	\caption{
		Fitted energy spectra from 500 to 525 keV before the LAr veto cut. 
		Function \ref{equ:FitNoFilters} was used as fit function. 
		Its course is indicated in red. 
		The two green plots at the bottom of the figures indicate the two independent peaks. 
		}
\end{figure}
\\

\iffalse

\begin{table}[t!]
\centering
\begin{tabular}{|l|c|c|}
\hline
Name	& Value [BEGe] & Values [COAX]\\ 
\hline
A [counts $\times$ / 0.2 ] &	(15.98 \(\pm\)	4.54)&	(31.93\(\pm\)7.08)	\\	
\hline
B [keV] &	(510.87 \(\pm\)	0.22)&	(510.88 \(\pm\)	0.190)\\	
\hline
C [keV] &	(0.952 \(\pm\)	0.011)	&	(1.165 \(\pm\)	0.010)	\\
\hline
D [counts / 0.2 keV] &	(36.66 \(\pm\)	4.99)	&	(30.40 \(\pm\)	5.24)	\\
\hline
E [keV] &	(513.95 \(\pm\)	0.16)	&	(513.87 \(\pm\)	0.14)	\\
\hline
F [keV] &	(0.953 \(\pm\)	0.014)	&	(1.145 \(\pm\)	0.012)	\\
\hline
G [counts / 0.2 keV] &	(11.98 \(\pm\)	5.27)	&	(11.35 \(\pm\)	7.45)	\\
\hline
H [counts / 0.2 keV] &	(189.92 \(\pm\)	110.05)	&	(996.72 \(\pm\)	972.62)	\\
\hline
I [1/keV] &	(5.80 \(\pm\) 1.22)$\times10^{-3}$	&	(9.24 \(\pm\)1.83)$\times10^{-3}$ \\
\hline

\end{tabular}
\caption{
	Fit parameters of fit function \ref{equ:FitNoFilters} applied on the spectra of the respective detectors. 
	Variable A and D correspond to the amplitudes of the two Gaussian peaks, B and E to the position of their maxima, C and F their standard deviation. 
	On the other hand is G the constant background, H the amplitude of the exponential background and I its decrease parameter.
	}
\label{tab:FitParNoFilter}
\end{table}
\\
\fi

However, the only fit parameter of real interest in this case is variable D.
Its value is the amplitude of the second Gaussian peak.
As the Gaussian peak was already chosen in the normalized form, this value also represents the amount of \Kr\ decays.

Taking into account the binning of the histograms, a value of  
\begin{equation*}
N_{\mathrm{peak,BEGe}} = (183\pm25) \unit{cts}
\end{equation*} in the peak of the BEGe spectrum and
\begin{equation*}
N_{\mathrm{peak,COAX}} = (152\pm26) \unit{cts}
\end{equation*}in the COAX spectrum has been detected.
\\

When it comes to the spectrum after the LAr veto, fit function \ref{equ:FitNoFilters} will also be used.
This is an advantage because it will also fit around a possible remaining 511 keV peak which otherwise might have affected the outcome.

\iffalse
As mentioned above one can assumed that the positron electron annihilation peak is fully suppressed.
therefore only one Gaussian peak has to be fitted together with the background function.
Additionally in the ideal case, that the majority of all \Kr\ decay events got through the LAr veto, one can expect the amplitude of the Gaussian peak to be equal to the amplitude of the not LAr veto filtered case.
The fit results in the function displayed in equation \ref{equ:FitFilters}.
\\ 

\begin{equation}
\mathrm{f}(x) = \mathrm{A}\frac{1}{\sqrt{2\pi}\mathrm{C}}\exp\left(-\frac{(x-\mathrm{B})^2}{2\mathrm{C}^2}\right) + \mathrm{D}\exp\left(\mathrm{-E}x\right) + \mathrm{F}
\label{equ:FitFilters}
\end{equation}
\\
\fi
\\

Diagrams \ref{fig:FitLArVetoBEGes} and \ref{fig:FitLArVetoCOAX} show the fitted plots.
As explained above, the amplitude D of the second Gaussian peak corresponds to the number of events measured in the area of the peak per binning.
This results in an amount of $(120\pm19)$ counts for the BEGe and $(128\pm22)$ for the COAX spectrum.
\\

Compared to the values of the non LAr filtered spectra, however, it can be seen that the number of counts in the BEGe and in the COAX detectors have dropped considerably.
This means that the LAr veto also rejected some of the \Kr\ decay caused events.
\\



\begin{figure}[t!]
	\centering
	\begin{subfigure}{.5\textwidth}
		\centering
		\includegraphics[width=\textwidth]{./Bilder/500525FitLArVetoBEGes.pdf}
		\caption{BEGes}
		\label{fig:FitLArVetoBEGes}
	\end{subfigure}\hfill%
	\begin{subfigure}{.5\textwidth}
		\centering
		\includegraphics[width=\textwidth]{./Bilder/500525FitLArVetoCOAX.pdf}
		\caption{COAX}
		\label{fig:FitLArVetoCOAX}
	\end{subfigure}
	\caption{
	Fitted energy spectra from 500 to 525 keV when only considering those events that got through the Muon veto, the detector anti-coincidence veto and the LAr veto. 
	Function \ref{equ:FitNoFilters} was used as fit function. 
	Its course is indicated in red. 
	The green plots at the bottom of the figures indicate the peaks responsible for the deviation. 
	Their course was determined from the corresponding fit parameter.
    In the case of the BEGe spectrum, the first peak function vanishes due to no measurable deviation from the background level.
    In the COAX spectrum, however, a deviation is still visible which is why the second peak function has a finite amplitude.
	}
\end{figure}
\iffalse
\begin{table}[t!]
	\centering
	\begin{tabular}{|l|c|c|}
		\hline
		Name	& Value [BEGes] & Value [COAX]\\ 
		\hline
		A [counts  / 0.2 ] &	(24.094734$\pm$3.856334)&	(25.680418$\pm$4.424006)\\	
		\hline
		B [keV] &	(513.925781$\pm$0.129591)&	(513.884766$\pm$0.185506)\\	
		\hline
		C [keV] &	(0.952731$\pm$0.019180)	&	(1.145909$\pm$0.018075)\\
		\hline
		D [counts / 0.2 keV] &	(14.548210$\pm$0.349083)	&	(10.602730$\pm$8.090884)\\
		\hline
		E [counts / 0.2 keV] &	(0.607414$\pm$5.193491)	&	(10.000000$\pm$9.831094)\\
		\hline	
		F [1 / keV] &	(0.901209$\pm$2.409593)	&	(0.001950$\pm$0.004637)\\
		\hline
	\end{tabular}
	\caption{
		Fit parameters of fit function \ref{equ:FitNoFilters} applied on the spectra of the respective detectors. 
		Variable A correspond to the amplitudes of the Gaussian peak, B to the position of its maximum, C its standard deviation. 
		On the other hand is D the constant background, E the amplitude of the exponential background and F its decrease parameter.
		}
			\label{tab:FitParFilter}

\end{table}
\fi

\iffalse
Nevertheless, the line count rate $N_{\mathrm{peak,BEGe}}$ and $N_{\mathrm{peak,COAX}}$ used in the further analysis were successfully determined from the fit of the unfiltered spectra.
These values represent the amount of events that were with almost absolute certainty caused by a 514 keV photon of the \nuc{Rb}{85m} relaxation.  
To be able to calculate the values of actual amount of \Kr\ per liter necessary for this amount of counts to be measured one has to determine a conversion factor. 
How this conversion factor can be determined is the topic of the following section.
\\
\fi

\section{Monte Carlo Simulation}
\label{sec:MonteCarlo514}

It is possible to calculate the decay density of \Kr\ necessary to create the measured line count by using the conversion factor $\frac{1}{p \epsilon V_{\mathrm{sim}} }$ between these two values. 
Such a conversion factor can be determined with the help of a Monte Carlo simulation.
The tool used to perform this simulation is \mage (MAjorana-GErda), a GEANT4-based physics simulation software developed jointly by MAJORANA and \gerda\ \cite{boswell_mage_2010}.
\mage is specialized in the simulation of radioactive decays and their corresponding energy deposition in germanium detectors.
In this thesis, the full \gerda\ implementation is used.
\\

A number of $N_{\mathrm{sim}}$ = 50.000.000 gammas with 514 keV were simulated in cylindrical volume of 2.5 m height and a diameter of 3 m resulting in $V_{\mathrm{sim}} = 17.65 \mathrm{m}^3$.
Compared with the volume of 64 m\(^3\) of LAr used in the \gerda\ experiment, this volume is much smaller.
Nevertheless, as we will see later, this volume is by far big enough for this purpose.
\\

\begin{figure}[t!]
	\centering
	\begin{subfigure}{.5\textwidth}
		\centering
		\includegraphics[height=75mm]{./Bilder/MC-Querschnitt-BEGes.pdf}
		\caption{Cross section from above}
		\label{fig:CrossSecAb}
	\end{subfigure}\hfill%
	\begin{subfigure}{.5\textwidth}
		\centering
		\includegraphics[height=75mm]{./Bilder/MC-Radius-BEGes.png}
		\caption{Radial cross section}
		\label{fig:CrossSecRa}
	\end{subfigure}
    \\
	\vspace{0.5cm}
    \caption{
    	The cross section of the simulated volume from above and its radial projection.
    	The colored area shows the density of all simulated decays. 
    	The red points indicate all events that were measured by BEGe detectors.
    	It can be seen that gammas which have been emitted far away from the detectors created no measurable signal in the detectors.
    	}
\vspace{0.5cm}
\end{figure}
\\

From these 50 million simulated gammas with the anti-coincidence veto already applied only about 90 thousand have created a signal in a detector, only 30.465 of them in a BEGe and 24.902 in a COAX.
The spatial distribution of all measured events in the BEGe detectors can be seen as red dots in figures \ref{fig:CrossSecAb} and \ref{fig:CrossSecRa}.

\iffalse
From it one can see that the overwhelming majority of the detected events were positioned close to the detectors themselves.
At longer distances, the majority of all decays were no longer measured.
This means that, while holding the simulated decay density constant, the amount of measured events should not change when enlarging the volume.
But that would also mean that the detector efficiency $\epsilon$ would have a reciprocal proportionality to the simulated volume.
thereforee, one can expect the ratio $\frac{1}{\epsilon V_{\mathrm{sim}}}$ to be invariant with change of volume, provided that the volume is large enough.   
This is the reason why the usage of a smaller volume of LAr in the Monte Carlo simulation was justified.
This ratio easily be adapted to a conversion factor from the amount of measured counts in the 514keV line to the density of \Kr\ necessary to create the measured peak, just by dividing the ratio through the probability $p = 0.434\%$ of \Kr\ decaying into the excited state of \nuc{Rb}{85m}.
\\
\fi


\begin{figure}[t!]
	\centering
	\ifmakefigures%
	\includegraphics[width=100mm]{./Bilder/MC-514-Phasenraum.pdf}
	\fi%
	\caption{
    Energy spectrum of simulated events measured in BEGe detectors.
	The blue colored bin represents the counts used for determining the detector efficiency.
	The majority of measured signals has an energy below 514 keV.
	Among other things, the Compton peak at roughly 343keV can be seen.  
	}
	\label{fig:PhasenraumMC514}
\end{figure}

\\
The spectrum of all the events detected in BEGe detectors is shown in figure \ref{fig:PhasenraumMC514}.
One can see that only in a small number of cases the gamma deposited all of its initial 514 keV in the detectors.
The majority of photons measured must have scattered before they arrived in the detector or did not deposit all their energy in them.
To calculate the detector efficiency, however, only the measured events at the 514 keV peak have to be accounted for.
In the BEGe detector spectrum, the peak contains a total of \(\Delta N_{\mathrm{BEGe}} = (4511\pm67)\) counts while the COAX peak contains \(\Delta N_{\mathrm{COAX}} = (3706\pm60)\) .
With a total of 50 million initial decays, this results in an efficiency of 
\begin{equation*}
\epsilon_{\mathrm{BEGe}} = \frac{\Delta N_{\mathrm{BEGe}}}{N_{\mathrm{sim}}} = (9.02\pm0.13) \times 10^{-5}  \frac{\mathrm{cts}}{\mathrm{gamma}}
\end{equation*}
\begin{equation*}
\epsilon_{\mathrm{COAX}} = \frac{\Delta N_{\mathrm{COAX}}}{N_{\mathrm{sim}}} = (7.412\pm0.12) \times 10^{-5}  \frac{\mathrm{cts}}{\mathrm{gamma}}
\end{equation*}
for the volume of the simulated cylinder.
This means that, if a 514 keV photon is emitted at any location in the liquid argon container, it has a probability \(\epsilon_{\mathrm{BEGe}}\) of being measured by one of the BEGe detectors.
On the other hand, for every measured 514keV photon in one of the BEGe detectors an amount of about $1 / \epsilon_{\mathrm{BEGe}} = 10515$ \nuc{Rb}{85m} relaxations must occur.
In other words, the value $1 / \epsilon_{\mathrm{BEGe}}$ is a factor to convert from the measured entries to the mean amount of \nuc{Rb}{85m} relaxations necessary to create it.
\\

This value is direct proportional to the simulated volume $V_{\mathrm{sim}}$.
This dependency can be derived from figures \ref{fig:CrossSecAb} and \ref{fig:CrossSecRa} where the red dots represent the location the measured gamma was released.
It can be seen, that essentially all measured gamma originated in the immediate vicinity of the detectors.
This would mean that, with a big enough volume and a constant decay density $\rho_{\mathrm{dec}}$, the value $\Delta N$ of each detector remains constant under change of volume.
$N_{\mathrm{sim}}$ on the other hand is directly proportional to the volume through to definition of $N_{\mathrm{sim}} = \rho_{\mathrm{dec}} V_{\mathrm{sim}}$.
With the definition of the detector efficiency, this results in the proportionality  $\frac{1}{\epsilon} \propto V_{\mathrm{sim}}$.
\\

Taking this into account, a volume independent conversion factor can be generated by dividing $\frac{1}{\epsilon}$ through $V_{\mathrm{sim}}$.
This new value $\frac{1}{\epsilon V_{\mathrm{sim}}}$ is a conversion factor from the measured line count to the gamma emissions per liter and can also be applied onto the counts measured in the real LAr tank.
By also dividing this new value through the probability $p=0.434\% \frac{\mathrm{gamma}} {\mathrm{decay}}$, you finally get the desired conversion factor $\frac{1}{ p\epsilon V_ {\mathrm{sim}}}$ between the measured line numbers and the necessary \Kr\ decay density to generate this peak. 
 
\begin{equation*}
\frac{1}{ p \epsilon_{\mathrm{BEGe}} V_{\mathrm{sim}}} = (144.65\pm2.15) \frac{\mathrm{decay}}{\mathrm{cts}\times l}
\end{equation*}
\begin{equation*}
\frac{1}{p \epsilon_{\mathrm{COAX}} V_{\mathrm{sim}}} = (176.07\pm2.90) \frac{\mathrm{decay}}{\mathrm{cts} \times l}
\end{equation*}



\section{Exposure and mean measuring time}
\label{sec:CalcActiv}

The last remaining value to be determined is the mean measuring time $\bar{t}$ of all respective detectors.
The reason why the whole duration of \PII\ was not used as measuring time, is due to the fact that not all detectors were recording the entire time.
For determining this mean measuring time, the exposures of all individual detectors have to be known.
An easy way to determine those values involves the test pulse signal.
As mentioned in the introduction, every 20 s the test pulse sends an electric pulse through the front end electrons of the detectors.
These events are also recorded and specifically marked as test pulse event. 
\\

Due to its periodicity, the effective measurement times of the individual detectors can easily be determined by counting the number of measured test pulse signals in each detector and by multiplying it with the 20 seconds:
\begin{equation}
    t_\mathrm{i} = N_{\mathrm{TP}}(\mathrm{i}) \times 20\mathrm{s}
\end{equation}
The masses of the individual detectors are also known.
The individual exposures are therefore also easily determinable by applying 
\begin{equation}
    \varepsilon_\mathrm{i} = t_\mathrm{i} \times m_\mathrm{i}
\end{equation}
The combined exposure of all detectors of the same kind can then be calculated by adding up all of the exposures of the individual detectors.  
\begin{equation}
    \varepsilon_{\mathrm{comb}} = \sum_\mathrm{i} \varepsilon_\mathrm{i}
\end{equation}
With this approach, it was possible to determine a combined exposure of $\varepsilon_{\mathrm{BEGe}} = 30.8 \unit{kg}\cdot \unit{yr}$ for the BEGe and $\varepsilon_{\mathrm{COAX}} = 28.1 \unit{kg}\cdot \unit{yr}  $ for the COAX.
The mean measuring times of the two detector types were then calculated by dividing their combined exposure through their combined masses. 
\begin{equation}
    \bar{t} = \frac{\varepsilon_{\mathrm{comb}}}{M} = \frac{\sum_\mathrm{i} N_{\mathrm{TP}}(\mathrm{i}) \times 20\mathrm{s} \times m_\mathrm{i}}{\sum_\mathrm{i} m_\mathrm{i}}
\end{equation}
By following this procedure the mean measuring times for the BEGe and the COAX detectors were determined to be
\begin{equation*}
   \bar{t}_{\mathrm{BEGe}} = 1.540 \unit{yr}  
\end{equation*}
\begin{equation*}
    \bar{t}_{\mathrm{COAX}} = 1.803 \unit{yr} 
\end{equation*}

These mean measurement times are exactly the time that each detector of one kind must have measured to obtain the same amount of combined exposure as from the actual individual measuring times.
The simplification of only using the mean measuring time and neglecting the individual detector on/off times creates a systematical error that will to be discussed later.
\\


\iffalse
However, this is a simplification that generates a certain error in the detector efficiency.
Every detector has a own efficiency of detecting 514 keV gammas.
If some detectors now have a longer measurement time than others, the combined detector efficiency of one type of detector deviates from the value determined above, which assumed that all detectors had the same measurement time.

Theoretically it would also have been possible to use the exposures of all individual detectors in the Monte Carlo simulation and determine how much a detector actually contributes to the detector efficiency.
However, this would have been very complicated and unnecessary, since the error can actually be neglected.
This is due to the fact that the measurement times of the individual detectors are almost the same anyway resulting in a mean value from which the individual measuring times do not vary too much.
This means that the error in the detector efficiency resulting from only using a mean measuring time can be neglected.  
\\
\fi
\iffalse
The determination of the mean measuring time needs the exposures of the individual detectors.
Those values can be calculated by looking at how many test pule signals have been recorded by it ($N_{\mathrm{TP}}(\mathrm{i})$). 
Since the test pulse signals have been set to a frequency of $f_\mathrm{TP} = 0.05\unit{Hz} $ over the entirety \PII\, an effective measurement time can be calculated, by multiply the number of counted test pulse signals it by 20 seconds.
The individual effective measuring times are therefore given by
\begin{equation*}
    t_\mathrm{i} = N_{\mathrm{TP}}(\mathrm{i}) \times 20\mathrm{s}
\end{equation*}
where i is the index of the respective input channel of each detector.
\\

The second problem arises from the fact that the decay densities were  calculated with the assumption that one could merge all detectors of the same kind into one single detector.
This would mean that fro the remainder of the analysis one would be able to calculate as if the BEGe or the COAX detectors were only one detector respectively.
But now that the effective measurement times of each individual detector is supposed to be used, this assumption does not hold true anymore.
There are two different workarounds for this.
\\

In the first method another Monte Carlo simulation would be run, this time considering the amount of time each detector was actually measuring.
This would be very inefficient.
\\

The second method works around the merging problem by calculating an average measurement time for all detectors of one kind.
With this one could then calculate with the detector block as if they were such a single detectors.
This is very elegant solution because it is by far easier then running another simulation.
\\

For this average measuring time one has to consider the fact that a weigh on each individual detectors effective measuring time has to be applied.
This arises from the fact that every detector has an individual mass.
It is known, the detector efficiency of a single detector is directly dependent on its mass.
If one wants to combine all single detectors into one large detector, one has to consider that heavier detectors contribute more to the detector efficiency than lighter detectors.
It is therefore necessary to weight the measuring time of each detector with its individual mass.
But the multiplication of the individual measuring time of a detector with its mass are also the exposure the individual detectors.
This means that to calculate the mean measuring time, one just has to divide the combined exposure of all detectors of one kind through their combined mass.
\\

With this one can finally determine the mean measuring times using equation \ref{meanmeauringtime}.

\begin{equation*}
    \bar{t} = \frac{\sum_\mathrm{i} t_\mathrm{i} \times m_\mathrm{i}}{\sum_\mathrm{i} m_\mathrm{i}}
\label{meanmeauringtime}
\end{equation*}
From this, an average measurement time for the BEGes of $\bar{t}_{\mathrm{BEGe}} = (1.540\pm0.001)\unit{yr}$ and for the COAX of $\bar{t}_{\mathrm{COAX}} = (1.803\pm0.001)\unit{yr}$ can be calculated.
\\
\fi

\section{Results}
\label{sec:res}
Finally, with the mean measuring times, the line count rates and the conversion factors the mean specific activity $\bar{a}$ of \Kr\ over the curse of the entire \PII\ can  finally be calculate by applying equation \ref{equ:activityDieErste}.
The resulting mean specific \Kr\ activities are
\begin{equation*}
    \bar{a}_{\mathrm{BEGe}} = (0.546\pm0.083)	\frac{\unit{mBq}}{\unit{l}}
\end{equation*}
and
\begin{equation*}
    \bar{a}_{\mathrm{COAX}} = (0.470\pm0.089)	\frac{\unit{mBq}}{\unit{l}}
\end{equation*}
These two values differ only inside their range of uncertainty.
From these two values an average specific \Kr\ activity for the whole period of \PII\ could then be calculated to
\begin{equation*}
\bar{a} = (0.508\pm0.086)\frac{\unit{mBq}}{\unit{l}}
\end{equation*}
\\

The uncertainty of this value is only statistical.
Systematical errors originate from the Monte Carlo simulation and the earlier mentioned detector on/off time. 
In the Monte Carlo simulation the situation of \gerda\ \PII\ are only simulated as good as possible.
For example variations on the construction or the functionality of the detectors or the neglect of cables and constructions around the detectors create an error.
By disregarding the actually detector on/off time an error is imposed.
In actuality the detectors measuring longer then the mean measuring time contribute a higher count of events to the overall amount of measured events than assumed with the mean measuring time and vice versa.
To compensate for the different detector efficiency  of the individual detectors to measure 514 keV gammas the mean measuring time was calculated using the mass as weighting factor.
But as the detector efficiency not only depends on the mass but also the concrete construction of the detectors an error occurs.
The size of these two errors are hard to estimate which is why they have not been evaluated.

%5.07978E-04	8.57336E-05

\iffalse

One can now make some comparisons of this value with the WARP and the Darkside experiments.
In the case of the WARP experiment a specific activity of $(160\pm130)\frac{\unit{mBq}}{\unit{l}}$ \label{} was measured for the \Kr.
On the other hand, an specific activity of $(2.8577 \pm 0.18122) \frac{\unit{mBq}}{\unit{l}}$ was measured in the Darkside experiment.
The here determined value of $(0.508\pm0.085)\frac{\unit{mBq}}{\unit{l}}$ is about one order of magnitude smaller than the specific activity in the Darkside and whole three orders of magnitude smaller than the WARP experiment.
From these comparisons one can see, that the specific activity of \Kr\ in \gerda\ \PII\ seems to be much smaller than in other experiments using LAr.
\\

What one can also do is take the specific activities of the other two experiments and determine how many counts one would have been able to measure if the \Kr\ had their specific activity.
As simplification only the theroeticall values for the BEGe detectors was determined. 
How many counts the corresponds activity would induce in the BEGe detectors can be determined with the help of formula \ref{equ:correspondingEvents}.
\begin{equation}
\mathrm{N} = \bar{a} \times p \times \epsilon_\mathrm{BEGe} V_{\mathrm{sim}} \times \bar{t}
\label{equ:correspondingEvents}
\end{equation}
With the activity and the values determined from the analysis above one can calculate a corresponding amount of about 76152 events for the WARP experiment.
In the case of the WARP experiment one could expect the count rate to be much higher than the 183 events determined from the actual measurement.
For the Darkside experiment with an amount of 1360 counts is again about one order of magnitude higher than the here measured events.
From these comparisons one can see that for a much higher specific activity of \Kr\ to have actually been present a much bigger amount of events should have been counted.
As a graphical representation, the peaks that would have been able to be seen in the measured spectrum are displayed in figure \ref{fig:WARP} and \ref{fig:Darkside}.
\\

\begin{figure}[t!]
	\centering
	\begin{subfigure}{.5\textwidth}
		\centering
		\includegraphics[width=75mm]{./Bilder/WARP.pdf}
		\caption{WARP}
		\label{fig:WARP}
	\end{subfigure}\hfill%
	\begin{subfigure}{.5\textwidth}
		\centering
		\includegraphics[width=75mm]{./Bilder/Darkside.pdf}
		\caption{Darkside}
		\label{fig:Darkside}
	\end{subfigure}
    \\
    \caption{
    	Energy spectra of the BEGe detectors from 500 to 525 keV.
    	The green plots in the spectra represent the theoretical peaks one could expect to measure if a specific activity of the respective experiment was present.
    	In both cases it would have created a much greater signal than observed above.
    	}
\end{figure}
\\

But why is the specific activity of \Kr\ in \gerda\ \PII\ so much smaller than in any other experiment?
As it was elaborated above, due to the argon being extracted from the atmosphere, \Kr\ should be present.
And the majority of the \Kr\ there should originate from nuclear power plants.
WARP's LAr also originated from the atmosphere but its specific activity was measured to be almost three orders of magnitude higher than \gerda's.
A possible explanation would be that the air from which the argon of WARP was extracted from was taken from a place with much a lot of nuclear power plants surrounding it while the argon of the \gerda\ experiment originated from air far away from any reactor.
On the other hand Darkside's argon originated from an underground reservoir where only natural fission decays produce any \Kr\.
But its specific activity is still much higher than \gerda's.
Also no extra purification of the LAr was applied.
This would mean that in the case of the LAr in \gerda\ one was extremely lucky to have found argon that had such low \Kr\ concentration that its impact on the background can basically be neglected.
\\

Now that a concrete value for the specific activity of \Kr\ has been determine the second attempt to measure the specific activity will described in the next chapter.
\\  
\fi



%	4.70E-04	8,87E-05	Bq/l


% calculate Amplitude of Gauss peak at 514keV and use factor from Monte Carlo Simulation to estimate

% look at phase diagram at range of 500 to 525 keV, use different filters and fit remaining data with Gaussian function
% -> get amplitude
% make a Monte Carlo simulation to estimate actual Kr85 activity in LAr from measured activity in detectors
% -> with amplitude and factors from MC-Simulation one can calculate the specific activity

\chapter{Change in Count Rate over Time}
\label{sec:SAfromDecrease}

This chapter focuses on the second approach using the change in count rate over time.
In the previous chapter using the line count rate has already shown the specific \Kr\ activity in the LAr to be in the order of 10$^{-4} \unit{Bq}/\unit{l}$,
However, when compared to \nuc{Ar}{39} ($T_{\frac{1}{2}} = 269\unit{yr}$) \cite{singh_nuclear_2006} having a relatively constant specific activity in the order of 1$\mathrm{Bq}/\unit{l}$ hardly any change in count rate made from \Kr\ should be able to be measurable due to it having a count rate four orders of magnitude smaller than the constant count rate of \nuc{Ar}{39}.
But if one were to actually measure a remarkable change due made by \Kr\ decays, one could use this result to falsify the value calculated in the first method.
\\

This process, however, assumes that \Kr\ is the only radioactive isotope with a notable change in its count rate meaning that it has the smallest half life to show any change over the time frame of \PII.




To actually measure the change in the count rate, the count rate must be drawn over time and an exponential fit function applied through it.
From the amplitude of the exponential function the value $R_{\mathrm{count}}(t=0)$ of counts per second at the start of \PII\ can be determined.
\\

It is also necessary to use a Monte Carlo simulation to calculate another conversion factor between the measured counts and the number of decays required for them. 
A new Monte Carlo simulation must be used, since in this case all \Kr\ must be considered in the simulation and not only the 514 keV gamma.
With this new conversion factor $\frac{1}{\epsilon V_{\mathrm{sim}}}$ the specific activity of \Kr\ can finally be calculated.
\begin{equation}
a(t=0) = \frac{R_{\mathrm{count}}(t=0)}{\epsilon V_{\mathrm{sim}}}
\label{equ:ActivityDieZweite}
\end{equation}
\section{Motivation}
\label{sec:motivation}
The resulting specific activity of the approach described above is in actuality the combined specific activity of the whole \gerda\ \PII\ setup and not specifically just \Kr.
The reason why one can expect to determine the individual specific activity of \Kr\ with this approach is that \Kr\ has one of the smallest half-lives ($T_{\frac{1}{2}}$ = 10.739\unit{yr}) of all remaining radioactive isotopes in the LAr. 
In comparison, other radioactive isotopes that are of interest are \nuc{Po}{210} (138 d)\cite{kondev_nuclear_2008} and \nuc{Ar}{42} (32.9 yr) \cite{chen_nuclear_2016} .
\\

Even though \nuc{Po}{210} has a half-life much shorter than \Kr's, most of the events created by it will have an energy much higher than \Kr's Q-value.
It can therefore be assumed that it will only have a negligible contribution to the change in count rate if only event with an energy lower than 400 keV will be considered.
\Kr's Q-value actually lies at 687 keV but it can be expected that the majority of \Kr\ events will have an energy lower then this.
By setting the limit even lower than \Kr\ Q-value of 687 keV, it will also suppresses some of the constant background of events with a higher energy while leaving out only a few \Kr\ decays with higer energies.
\\

\nuc{Ar}{42} has a half life that is in the same order of magnitude as \Kr\ and most of its emitted electrons also have an energy in the same range as \Kr\.
Even though its specific activity has already been determined to be smaller than \Kr's at about $0.148\mathrm{mBq/l}$ \cite{becerici_schmidt_results_2014}, this isotope can still be problematic as it decays into the unstable isotope \nuc{K}{42}.





The motivation behind this approach lies in \Kr\ having a low half life of $T_{\frac{1}{2}} = 10.739\unit{yr}$.
As it will be shown in section \ref{sec:motivation} it is one of a few isotopes that should show a notable change over the time scale of \PII.









Right after its creation \nuc{K}{42} is positively ionized as the electron also created in the beta decay escapes due to its high energy.
These positively ionized \nuc{K}{42} atoms will then be attracted by the electrostatic fields of the germanium detectors.
This results in a higher \nuc{K}{42} density around the detectors.
%This, however, makes it impossible to calculate a detector efficiency for \nuc{K}{42}  as it is no longer homogeneously distributed in the LAr tank. 
\nuc{K}{42} also has a half life of 12.355 h \cite{chen_nuclear_2016}.
It is therefore to be expected that \nuc{K}{42} has the same activity in the LAr tank as \nuc{Ar}{42} because compared to to \nuc{Ar}{42} it decays instantaneously.
However, due to its higher density around the detectors a much higher specific activity can also be expected to be measured in the detectors.
Because of this one can assume that it will also show a much bigger change in the count rate than expected.
This could mean that it might even dominate over \Kr\ change in count rate.
How much of an impact it will actually make on the change in count rate is hard to approximate.
However, if a notable change in count rate was able to be measured it might not even have to be caused by the \Kr.
In this case the change in count rate will be fitted with two different fit functions - on with a fixed half life of \Kr\ and one with a fixed half life of \nuc{Ar}{42}.
Depending on which of the fits is better suited for the measured data it will be decided whether \Kr\ or \nuc{Ar}{42} is responsible. 

\\



\section{Count Rate}
\label{sec:EventAct}

\begin{figure}[t!]
	\centering
	\begin{subfigure}{.475\textwidth}
		\centering
		\includegraphics[width=\textwidth]{./Bilder/ZeitverlaufALLE.pdf}
		\caption{energies lower than 400 keV}
		\label{fig:ZeitAll}
	\end{subfigure}\hfill%
	\begin{subfigure}{.475\textwidth}
		\centering
		\includegraphics[width=\textwidth]{./Bilder/ZeitverlaufLimits.pdf}
		\caption{energies between 200 and 400 keV}
		\label{fig:ZeitLimits}
	\end{subfigure}
    \\
    \caption{
    	The change of count rate over time.
    	In both figures the amount of counts measured in one week are plotted over the whole time frame of \PII. 
        The binning of these hisograms was chosen to be one week per bin which results in about 113 bins over all of  \PII.
    	In figure (a) no filters were imposed onto the used events while in (b) only those events were used that have an energy between 200 and 400 keV. 
    	One can see that further precautions must be taken before an exponential decrease can be determined. 
    	}
\end{figure}

To determine the change of count rate over time it is necessary to plot the amount of measured counts over time in a histogram (see figure \ref{fig:ZeitAll}) and later divide those counts by the mean measuring times inside the respective bins of the histogram.
The data used in this approach is the \gerda\ \PII\ data from run 55 to 92.
The used data had also been cut again by the standard \gerda\ analysis cuts (quality, muon, anti-coincedence).
The anti-coincidence cut here also because not only the gammas basically always deposit all their energy in one detector but due to the low  $Q_\beta$ of the \Kr\ the released beta as well.
 \\

Figure \ref{fig:ZeitAll} shows the amount of counted events with an energy below 400 keV in each individual week.
From it a jump in the count of events in the second half of \PII\ can be seen.
However, this discontinuous change makes it impossible to determine the change in the counting rate, as no fit function can be applied through the resulting counting rate diagram. 
It is therefore necessary to find a way to suppress this jump, but to do so, its origin must be investigated.
\\

This jump occurs on the 12.10.2017.
On this day the lower energy threshold of all detectors was lowered.
This dramatically increased the number of events measured.
Due to every detector having different characteristics, the limit was set for each detector individually.
Before the lowering the highest limit was set at about 140keV for the BEGes and at 185 for the COAX.
After the lowering all event with an energy of at least about 15keV could be recorded by all detectors.
For a graphical representation, see the beginning of the spectra in figure \ref{fig:before} for all events measured before and figure \ref{fig:after} for all events after the lowering of the limit. 
To work around this continuity problem all one has to do is to apply a lower energy limit of 200 keV.
\\


\begin{figure}[t!]
	\centering
	\begin{subfigure}[t]{.475\textwidth}
		\centering
		\includegraphics[width=\textwidth]{./Bilder/beforeTheFall.pdf}
		\caption{before}
		\label{fig:before}
	\end{subfigure}\hfill%
	\begin{subfigure}[t]{.475\textwidth}
		\centering
		\includegraphics[width=\textwidth]{./Bilder/afterTheFall.pdf}
		\caption{after}
		\label{fig:after}
	\end{subfigure}
	\caption{
		Energy spectra from 0 to 200 keV. 
		Figure (a) shows the resulting spectrum of all events measured before the lowering of the lower energy threshold of recorded events on 12.10.2017 while (b) shows the spectrum after this date.
		}
		\vspace{5mm}
\end{figure}

What must also be taken into account is that the approach through a combined counting rate requires all detectors used for the evaluation to have been continuously measuring throughout all of \PII\.
If at any point one of the detectors stops taking measurements, it would result in a jump in this combined count rate creating discontinuities in its course.    
In the further course of this analysis, therefore, all events are rejected that were measured in a detector whose measurement process was interrupted at some point in \PII.
This retroactively also explains why the data of \gerda\ run 53 and 54 have been disregarded.
In these two runs, a smaller number of detectors were measuring than usual.
This would result in a small pool of detectors with an uninterrupted measurement process available and therefore less data being rejected.
By discarding the data of run 53 and 54 more detectors can be used in the analysis and therefore more data.
The resulting count histogram considering the upper and lower energy limit and only using data from continuously measuring detectors can be seen in figure \ref{fig:ZeitLimits}.
\\

The count rate of each individual week can now be determined by dividing histogram \ref{fig:ZeitLimits} by the mean measuring times of each week in histogram.
To determine these mean measuring times the test pulse signal can be used again.
Similar to how it was done in the line count rate analysis for each week the number of recorded test pulse signals can be determined and their individual amount multiplied by 20 s.
Using this procedure results in a histogram as seen in figure \ref{fig:effectiveMeasuringTimes}.
\\

\begin{figure}[t!]
	\centering
	\begin{minipage}[t]{.475\textwidth}
		\centering
		\includegraphics[width=\textwidth]{./Bilder/testpuler.pdf}
		\caption{
			Histogram showing the mean measuring time of each week corresponding to the individual bins.
			Those were determined by the amount of test pulse signals measured in the time frames and their amount multiplied by 20s.
			One can see, that the measuring time of each week varied on a broad level.
			}
		\label{fig:effectiveMeasuringTimes}
	\end{minipage}\hfill%
	\begin{minipage}[t]{.475\textwidth}
		\centering
		\includegraphics[width=\textwidth]{./Bilder/eventRate.pdf}
		\caption{
			Count rate of measured events between 200 and 400 keV over the course of \PII.
			A continuous change in rate can be seen. 
			One can now apply a fit using function \ref{equ:FitFilters2} and through it determine the count rate of \Kr\ at the start of \PII.
			}
		\label{fig:ChangeInEventRate}
	\end{minipage}
	\\
\end{figure}

Figure \ref{fig:ChangeInEventRate} finally shows the resulting count rate measured over the course of \PII.
This graph can now be fitted with fit function
\begin{equation}
\mathrm{f}(x) = \mathrm{A}\times\exp\left(-\frac{\log(2)}{\mathrm{B}} x \right) + \mathrm{C}
\label{equ:FitFilters2}
\end{equation}
Fit parameter A corresponds to the initial count rate caused by the decay of \Kr\, B to \Kr's half-life, and C to the constant background rate.
Fit parameter B was fixed to \Kr's half-life of \(10.739\unit{y}\) , but all other parameters remained free, only limited by the condition of them being positive.
The resulting fitting plot can be seen in figure  \ref{fig:ChangeInEventRateFit}.
The only parameter of interest for further analysis is the parameter \(\mathrm{A}\). 
It corresponds to a count rate  
\begin{equation*}
R_{\mathrm{count}}(t = 0) = (1.495\pm0.227) \times 10^{-3} \frac{\mathrm{cts}}{\unit{s}}
\end{equation*}of \Kr\ decay caused events at the start of \PII\.
\iffalse
\begin{table}[t!]
	\centering
	\begin{tabular}{|l|c|}
		\hline
		Name 	& Value  \\ 
		\hline
		A [1/s] &	(1.521 $\pm$ 0.224)$\times10^{-3}$\\	
		\hline
		B [yr] &	10.739\\	
		\hline
		C [1/s] &	(2.714 $\pm$ 0.210)$\times10^{-3}$\\
		\hline
	\end{tabular}
	\caption{
		Fit parameters of fit function \ref{equ:FitNoFilters} applied on the spectra of the respective detectors.
		Parameter A represents the amplitude of the exponential decay function and B the half life of the decaying isotope.
		fit parameter C is there to handle the constant background created by other radioactive isotopes with much higher half lives.
		}
    \label{tab:FitParZeit}
\end{table}
\fi
\begin{figure}[t!]
	\centering
	\ifmakefigures%
	\includegraphics[width=100mm]{./Bilder/eventRateFit.pdf}
	\fi%
	\caption{
	    Fitted count rate diagram over the time frame of \PII.
	    For the fitting process function \ref{equ:FitFilters2}.
	    The resulting fit parameter can be seen in table \ref{tab:FitParZeit}.
	}
	\label{fig:ChangeInEventRateFit}
\end{figure}%


\section{Monte Carlo simulation}
\label{sec:MonteCarlo2}

The second and last value needed to determine the specific activity of \Kr\ is the conversion factor $1/\epsilon V_{\mathrm{sim}}$.
To obtain this conversion factor, almost the same process has to be applied as for line count rate analysis, with the only difference that this time the \Kr\ decays themselves are simulated instead of 514 keV gamma.
\\

A problem is, however, that detector efficiency of \Kr\ can be expected to be very small.
This is due to the beta electrons of the \Kr\ decay having a low probability of actually being detected.

On the one hand, electrons emitted far away from the detectors have a longer distance to travel in which they lose a lot of energy by generating scintillation light.
This effectively reduces their range and therefore their probability of even been detected.

On the other hand, electrons only have a small transmission factor in germanium.
This causes a significant number to be absorbed in the dead zone of the detectors where they do not generate any measurable signal.
In comparison, the 514 keV gammas have a large transmission factor that allows them to deposit their energy deep in the active volume of the detectors.
\\
\begin{figure}[t!]
	\centering
	\begin{minipage}[t]{.475\textwidth}
		\centering
		\includegraphics[width=\textwidth]{./Bilder/Sim1Phasenraum.pdf}
		\caption{
			Energy spectrum computed by simulating 1 billion \Kr\ decays and plotting the counts of events by their corresponding energy.
			The blue colored area represents the amount of counts used for the calculation of the detector efficiency.
			From it can be seen, that the majority of events were created by the photons of the 514 keV peak and only about 20$\%$ from the electrons of every other \Kr\ decay.
			}
		\label{fig:Sim1Spektrum}
	\end{minipage}\hfill%
	\begin{minipage}[t]{.475\textwidth}
		\centering
		\includegraphics[width=\textwidth]{./Bilder/Aktivitaet.pdf}
		\caption{
		The calculated specific activity of \Kr\ over the time frame of \PII.
		This plot was created by subtracting the constant background of each point in figure \ref{fig:ChangeInEventRateFit} and multiplying each value with the conversion factor.
		}
		\label{fig:activity}
	\end{minipage}
\end{figure}


That is why a far bigger number of N = 1 billion \Kr\ decays in a volume of again $V_{sim} = 17.65 \mathrm{m}^3$ were simulated.
From the simulated events  a combined spectrum of the measured \Kr\ beta electrons plus the 514 keV gammas can be created which can be seen in figure \ref{fig:Sim1Spektrum}.
\\

One can now compare this spectrum with figure \ref{fig:PhasenraumMC514} of the first Monte Carlo simulation.
It is of interest to find out how large the share of events generated by the beta electrons of all measured events is in the range of interest.
If their contribution to the spectrum were too big some corrections have to be made on the resulting detector efficiency as the simulation only estimated the volume of the dead layer creating an error.
To evaluate the betas contribution a comparison value has to be found.
Such a value can be generated by using the 514 keV gamma peak.
This peak can be used as a normalization factor.
When tanking the sum of all events measured in the range from 200 to 400 keV and dividing it through the number of line counts of the 514 keV peak one results in value $N_{\mathrm{sum}}/N_{\mathrm{peak}} = 2.481$ for the first and $3.137$  for the second simulation.
The value of the first simulation represents the relative amount of gamma events in 200 to 400 keV range per 514 keV line count whereas the value from the second simulation represents the ratio of all \Kr\ caused events per line count.
This means that one can determine the relative contribution of the gamma events in the second simulation by dividing the first value through the second value.
This results that a majority of $79.1 \%$ of all events measured in the 200 to 400 keV range were actually caused by gammas whereas only a smaller amount of $20.9 \%$ were caused by the released beta electron.
Due to the fact that the correction on the detector efficiency would have only change
\\

When compare this spectrum with the one determined for the other Monte Carlo simulation you can see these two spectra look relatively similar.
To test this one can calculate a ratio from the two spectra and compare them.
An appropriate ratio for the range of values of this analysis would be to take the sum of all events with an energy between 200 and 400 keV $N_{\mathrm{sum}}$ and divide by the counts in the 514 keV peak $N_ {\mathrm{peak}}$.
This results in a value of $N_{\mathrm{sum}}/N_{\mathrm{peak}} = 2.481$ for the first simulation and $3.137$ for the second simulation.
Due to the fact that the first simulation only computed 514 keV photons the first value represents the ratio when only the 514 keV photons would contribute to the energy spectrum.
The second value, on the other hand, represents the ratio when the 514 keV photons together with the electrons generate the spectrum.
This means that the electrons only generate about 20$\%$ of all measured events in the range of 200 to 400 keV.
The 514 keV photons created the rest.
\\

This is actually an advantage.
If the majority of the measured events were created by electrons then one would have to apply a correction onto the detector efficiency.
This is because of a weakness of the simulated detectors.
The dead layer of the actual detectors isn't known that well which is why the simulated detectors can only work with an assumptions of the actual dead volume of the detectors.
Unfortunately, it's hard to correct the determined value for this weakness.
It is therefore lucky that the 514keV photons are responsible for the majority of counts in the spectrum and no such correction has to be applied.
\\

But this conclusion of photons being the main cause of the spectrum assumes that in both simulations the same environmental conditions and \Kr\ characteristics were simulated.
Whether this is true or not can easily be determined by looking at line count of the characteristic 514keV peak.
In the case of the first simulation an amount of 10179 events were measured by the BEGe detectors in the 514keV peak with a decay density of about 2832 $\unit{photon}/\unit{l}$.
On the other hand in the second simulation only 817 events were measured with a decay density of 56640$\unit{decay}/\unit{l}$.
But one has to consider that in the case of the first simulation only the photons of the \nuc{Rb}{85m} relaxation were emitted and \Kr\ only decays into this exited state with a probability of 0.434$\%$.
The effective decay density of the first simulation is therefore about 652511 $\unit{decay}/\unit{l}$.
To compare these two line counts, one now has to scale one of the two line counts so that the same decay density can be assumed.
By applying the scale of $\frac{56640}{652511}$ onto the amount of events in the 514keV peak from the first simulation you get a adjusted value of 884 events.
The fact, that this value is roughly the same size as the value 817 form the second simulation, proves, that a similar environmental situation and \Kr\ characteristics must have been simulated.
This conclusion also justifies retroactively the simplification in the first simulation where only photons of the \nuc{Rb}{85m} were emitted to replace the computation of the actual decays.
\\

Now to calculating the conversation factor from this simulation.
This requires determining the number of events with an energy in the range 200 to 400 keV and the untriggered detector anticoincidence veto.
In this case an amount of $\Delta N$ = (1438$\pm$38) events was determined.
With the absolute amount $N_{\mathrm{sim}}$ of decays simulated we determine a detector efficiency of
\begin{equation*}
    \epsilon = \frac{\Delta N}{N_{\mathrm{sim}}} = (1.438\pm0.038)\times10^{-6} \frac{\unit{event}}{\unit{decay}}
\end{equation*}
This values says that any decay in the liquid argon has a probability of $\epsilon$ to be measured by one of the detectors that was always on.
With this value one can again calculate the volume independent conversation factor
\begin{equation*}
    \frac{1}{\epsilon V_{\mathrm{sim}}} = (39.38\pm1.04) \frac{\unit{decay}}{\unit{event} \times \unit{l} }
\end{equation*}
With this conversation factor one can now finally calculate the specific activity of \Kr.
By applying equation \ref{equ:ActivityDieZweite} and feeding it with the values found we result in an specific activity of $a(t=0) = (61.4\pm41.9) \frac{\unit{mBq}}{\unit{l}}$ at the start of Phase II.

One can now get back to the graph \ref{fig:ChangeInEventRateFit}, subtract the constant background of each point and scale each value by the conversation factor.
The resulting graph can be seen in figure \ref
Now it is of interest to compare this value to the specific activity found in the line count rate analysis.
For this one has to calculate the mean specific activity over all of Phase II.
This can easily be done by integrating an exponential decay function over all of \PII\ and then dividing it by 2.17 yr.

\begin{equation*}
    \bar{a} = \frac{1}{2.17\unit{yr}} \int_0^{2.17\unit{yr}} a(t=0)\times\exp \left( \frac{\log(2)}{T_{\frac{1}{2}}} t \right) \mathrm{d}t
\end{equation*}

with $T_{\frac{1}{2}}$ being the half life of \Kr.
This results in an mean specific activity of $ \bar{a} = 57.294 \frac{\unit{mBq}}{\unit{l}}$.
\\

When comparing the here obtained result with the specific activity found in the line count rate analysis you will realize that there is a problem.
The mean specific activity of all investigated detectors found in the first method had a value of $\bar{a} = (0.508\pm0.086)\frac{\unit{mBq}}{\unit{l}}$.
Compared to this is the value determined here is two orders of magnitude bigger.
This is problematic.
The critical question are now: 
Which of the methods is faulty, which of the assumption used was wrong and can one still make a statement about \Kr\ specific activity from the problematic approach.
These question will be answers in the following section.
\\


\section{Discussion}
\label{Discussion}

The biggest advantage of the line count rate analysis compared to the second method is that in its case the investigated effect can only originate from \Kr\ and no other isotope.
Because of this the likelihood of its determined value being the actual specific activity is relatively high.
By comparing the results of the first method with those of the WARP and Darkside experiments, you have already seen that the specific activity of \Kr\ is lower than in all other experiments.
The second method however has the problem that theoretically every other isotope that is residual in the LAr contributes to the investigated change over time.
It can therefore be very likely that the mistakes made lies in the assumption that all contribution of any other isotope to this change can be ignored. 
In the beginning of this chapter all isotopes that are the most probable sources of error have already been discussed.
It was already concluded that any contributions from \nuc{Ar}{39} and \nuc{Po}{210} can be disregarded due to their either high half life or their high mean kinetic energy of the escaping particle.
But as it was already indicated there, it might be necessary to discuss the influence of \nuc{Ar}{42} further.
\nuc{Ar}{42} has a half life of 32.9 y which is in the same order of magnitude as the half life of \Kr.
thereforee, one can expect to see a change in its activity over the course of \PII~ of about 4.4$\%$ of its absolute specific activity.
With an specific activity big enough one could expect this isotope to be responsible for the measured decrease.
But due to \nuc{Ar}{42}'s specific activity being much lower than the measured amplitude of $57.294 \frac{\unit{mBq}}{\unit{l}}$  it cannot be the only cause of this .
\\

What is really interesting about \nuc{Ar}{42}, is not the decay of \nuc{Ar}{42} itself, but rather its daughter nuclei \nuc{K}{42}.
\nuc{K}{42} has a half life of 12.355 h \cite{chen_nuclear_2016}.
It can now be approximated, that \nuc{K}{42}'s specific activity is equal to \nuc{Ar}{42}'s specific activity.
This comes from the fact that in comparison to the half life of \nuc{Ar}{42} the resulting \nuc{K}{42} practically decays instantaneously.
This means that it is possible to directly assign one \nuc{K}{42} decay to each \nuc{Ar}{42} that decays.
This doubles the number of decays which can be attributed to \nuc{Ar}{42}.
But assuming that the individual \nuc{Ar}{42} and \nuc{K}{42} isotopes are homogeneously distributed in the LAr, their combined specific activity is still by far not big enough to explain the bigger amplitude measured.
\\

But here lies the critical point.
The \nuc{K}{42} isotopes are in fact not homogeneously distributed in the LAr.
To understand why they are not one has to look at what a \nuc{K}{42} does right after it is formed.
Right after a \nuc{Ar}{42} decays its resulting \nuc{K}{42} isotope positively ionized.
This results from the fact that the newly formed electron escapes from the decay position due to its high velocity.
Now that the \nuc{K}{42} isotopes are positively charged, they can be deflected by an electric field.
\\

Such fields in the LAr tank originated from the germanium detectors.
In a germanium detector it is necessary to create a great electrostatic potential difference between the doped electrodes.
The resulting electric field separates electron-hole pairs that might have been created by other particles depositing their energy in the semiconductor \cite{spieler_semiconductor_2005}.
The field then guids the particles to the respective electrodes and by that creating a current which can be measured externally.
For a high detection efficiency strong fields are necessary.
Otherwise the electron or the hole might recombine in the p-type volume without creating any signal.
\\

The positively ionized isotopes are now deflected by these electrostatic fields to the negatively charged surface of the detectors. 
In the LAr the mean time it takes the ionized \nuc{K}{42} to capture an electron as compensate for its positive charge is relatively long. 
Thats why even though the fields are not very strong far away of the detectors the \nuc{K}{42} can still travel a relatively long distance before it either captures an electron or decays itself. 
Now that the density of \nuc{K}{42} isotopes is higher near the germanium detectors it can be expected that the specific activity around the detector also increases.
\\

The effect explained here has already been measured in \gerda\ and it is actually one of the greatest background sources of the whole experiment.
\nuc{K}{42} has also the unfortunate characteristic for the \gerda\ experiment that its Q-value of about $Q_\beta =3525\unit{MeV}$ lies higher than the Q-value of the double beta decay of \nuc{Ge}{76} $Q_{\beta \beta} = 2039\unit{MeV}$.
This means that electrons of the \nuc{K}{42} decay can create background events in the investigated rage. 
Initially it was not expected to create much background.
Its influence on the measurement however was only recognized after the start of \PI, when its counting rate was higher by a factor of 20. \cite{becerici_schmidt_results_2014}.
Techniques to suppress the background of \nuc{K}{42} were introduced over time like the usage of nylon mini-shrouds (NMS) as a mechanically barrier.
The NMS together with the active filters of the pulse shape analysis and the LAr veto have proven to bee able to suppress \nuc{K}{42} events by more then three orders of magnitude.
\\

In this case however it is not possible for the analysis to use the active filters to suppress the \nuc{K}{42} further.
Otherwise one would also filter out the \Kr\ events that are supposed to be investigated here in the first place.
So it still can be expected that a not negligible proportion of the measured specific activity comes from the denser \nuc{K}{42} in the region of the detectors.
What effective specific activity can be expected from the \nuc{Ar}{42} is difficult to estimate, but you can assume that it should be far greater than that of \Kr.
To get a quantitative approximation, you can use the plot of the event rate change again (see \ref{fig:ChangeInEventRate}) and modify the fit function used so that two exponential decays plus a constant background are taken into account.
Both exponential decay fit functions have their half life fixed to the respective values of \Kr\ (10.739y) and \nuc{Ar}{42} (32.9y).
The used fit function can be seen in equation \ref{fig:doubleFitFun}, the resulting fit function in figure \ref{fig:double} and the fit parameter in table \ref{tab:doubleFitpara}.
\begin{equation}
    \mathrm{f}(x) = \mathrm{A}\times\exp\left(-\frac{\log(2)}{\mathrm{B}} x \right) + \mathrm{C}\times\exp\left(-\frac{\log(2)}{\mathrm{D}} x \right) + \mathrm{E}
    \label{equ:doubleFitFun}
\end{equation}

\begin{table}[t!]
	\centering
	\begin{minipage}[t!]{.475\textwidth}
	\centering
	\ifmakefigures%
	\includegraphics[width=75mm]{./Bilder/doppelt.pdf}
	\fi%
	\captionof{figure}{
		    Fitted count rate diagram over the time frame of \PII.
	    For the fitting process function \ref{equ:doubleFitFun}.
	    The resulting fit parameter can be seen in table \ref{tab:doubleFitpara}.
	}
	\label{fig:double}
	\end{minipage}\hfill%
    \begin{minipage}[t!]{.475\textwidth}
    \centering
    \begin{tabular}{|l|c|}
        \hline
        Name  & Value \\
        \hline
        A [1/s]    & (1.024$\pm$0.122)$\times 10^{-3}$ \\
        \hline
        B [yr] & 10.739 \\
        \hline
        C [1/s]  & (1.454$\pm$0.130)$\times 10^{-3}$ \\
        \hline
        D [yr] & 32.9 \\
        \hline
        E [1/s] & (1.757$\pm$0.120)$\times 10^{-3}$ \\
        \hline
    \end{tabular}
    \caption{
        Fit parameters of fit function \ref{equ:FitNoFilters} applied on the spectra of the respective detectors.
		Parameters A and C represent the amplitude of the exponential decay function and B and D the half lives of the decaying isotopes.
		Fit parameter E is there to handle the constant background created by other radioactive isotopes with much higher half lives.
    }
    \label{tab:doubleFitpara}
	\end{minipage}
\end{table}


In a direct comparison of figure \ref{fig:double} with graph \ref{fig:ChangeInEventRateFit} no real difference can be see.
When looking at the fit parameters however one can see that in this case the \nuc{Ar}{42} exponential fit function has a bigger amplitude than \Kr.
Because the \nuc{K}{42} density is not homogeneously distributed in the liquid argon it is impossible to run a Monte Carlo simulation to again determine a new detector efficiency.
But using the conversation factor determined from the second simulation to make an estimate the respective effective specific activities can be calculated to $(40.4\pm0.59)\frac{\unit{mBq}}{\unit{l}}$ for \Kr\ and $(57.3\pm0.66)\frac{\unit{mBq}}{\unit{l}}$ for all of the decays resulting from \nuc{Ar}{42}.
These values are both still way to high for what was expected of them.
The maximum effective specific activity of \nuc{Ar}{42} expected to be  only about one orders of magnitude higher than the actual specific activity of only the \nuc{Ar}{42}.
But due to the fit parameters still generating far too much specific activity for these isotopes shows that other sources of change are still not accounted for.
\\

Nevertheless it has been shown with this fit that \Kr\ does not contribute solemnly to the change of count rate over time and with this the assumption from before has been shown to be false.
It would still be possible to determine an specific activity of \Kr\ from plot \ref{fig:ChangeInEventRate} by considerein every individual source of change in count rate.
But due to the higher number of fit parameters that would have to be determined and the way to small time frame compared to the half lifes of the isotopes, one can be quite pessimistic whether this analysis can even make a satisfying statement about the specific activity of \Kr.
\\

At the end of this analysis it is of interest again to investigate how the change in the count rate would have looked over time if the specific activities of the WARP or the Darkside experiment had been present.
Again, in the case of the WARP experiment an activity of $(0.16\pm0.13)\frac{\unit{Bq}}{\unit{l}}$ was measured. 
If one would plot its equivalent amplitude in a diagram like Figure \ref{fig:ChangeInEventRateFit} a significant change would have been visible.
On the other hand is the activity in the Darkside experiment with its $(2.8577 \pm 0.18122) \frac{\unit{mBq}}{\unit{l}}$ too small to measure an apparent change over time.
\\



% use fit to calculate the decrease in rate of the signals in range from 200 to 500keV
% from fit and assumption that only Kr85's activity is decreasing one can calculate the specific activity from the amplitude of the fit

% use Volume of LAr-Tank to determine the number of Kr85 and from this the concentration in the argon

\chapter{Summary and Outlook}
\label{sec:ConcAndOutlook}
\section{Summary}
With the line count rate analysis carried out in this paper it was able to show that the specific activity of \nuc{Kr}{85} in the LAr in \gerda\ \PII\ has a value of $(0.508\pm0.086) \frac{\unit{mBq}}{\unit{l}}$. 
On the other hand, the second attempt to determine a value for the specific activity via the change of count rate over time was unable to find a value.
This was due to \nuc{Ar}{42} having a comparable long half life and an effective specific activity much higher than expected.

\\
\section{Comparison to WARP and Darkside}

Two other experiments also using LAr were mentioned at the end of the introduction chapter: the Darkside experiment using UAr and the WARP experiment also using argon from the atmosphere.
Tt was assumed there that Darkside's specific \Kr\ activity [$(2.8577 \pm 0.18122) \frac{\unit{mBq}}{\unit{l}}$] was measured two orders of magnitude smaller than in the WARP experiment [$(160\pm130)\frac{\unit{mBq}}{\unit{l}}$] because the UAr did not come into contact with the man-made \Kr.
It was therefore also assumed that \gerda's specific \Kr\ activity should also be much higher than Darkside's.
However, the value of specific \Kr\ activity in the LAr of \gerda\ determined in this thesis is now even smaller than that of Darkside.
\\

For future experiments with LAr it would now be of particular interest to find out why the specific \Kr\ activity in \gerda\ is so much smaller than in the WARP experiment with the same source for its argon, but also why it is even smaller than the UAr of Darkside, which should be the radio pure material.
But it is hard to find a reason why it should be.
The LAr used in \gerda\ was bought commercially and has not been further distilled.
Also the \Kr\ in the atmosphere should have been relatively homogeneous or at least in the same order of magnitude \cite{j._jacob_atmospheric_1987}.
Therefore, it only matters on a way to small range where the air for the argon extraction was taken from in the atmosphere.
\\

One possible causes might be that in the extraction process used in the case of the \gerda\ LAr was better at filtering out any residual alien isotopes than it was for the WARP LAr.
Another possible reason might be that the air in \gerda\ or WARP was not actually taken from the atmosphere but from a location closer to earth.
At the surface the \Kr\ actually varies greatly with the proximity of the location to nuclear fission reactors \cite{weiss_mesoscale_1986}.
If the air for \gerda\ was now taken closer to Earth but further away from any nuclear power plant, this might explain the discrepancy between \gerda\ and WARP. 
But to have a real answer to why the specific \Kr\ activity in \gerda\ is so low, a proper investigation must be carried out to find out exactly where the seller of the argon has caught the air and how they separate the argon from it.
\\

Besides the activities themselves, you could also go the other way and calculate how high the line count rates would have been in the \gerda\ setup if the specific activities of WARP and Darkside would have been present.
In section \ref{sec:Fitting} a line count of about 180 counts was found.
The corresponding line count of the other specific activities of the other experiments can then be determined by upscaling the line count measured in \gerda\ to the corresponding values with the factor ($\bar{a}_{\mathrm{Exp}}/\bar{a}_{\mathrm{GERDA}}$).
This results in a line count of about $\mathcal{O}(1000)$ in the Darkside experiment and about $\mathcal{O}(50000)$ in the WARP experiment.
If their specific \Kr\ activity had been present, a much sharper peak could have been measured.


\section{Outlook}
Also mentioned in the introduction, the \gerda\ experiment is planned to keep measuring until 2020.
Afterwards the \gerda\ setup will be converted into the LEGEND experiment that will investigate the lower limit of \nuc{Ge}{76}'s half life further.
In it LAr is used again as coolant.
However, new argon will be used.
It could therefore be that in LEGEND again a much higher concentration of \Kr\ can be present and thus the background generated by it can be larger.
However, LEGEND will also feature a new and more precise LAr veto setup.
With this new LAr veto it might  actually be possible to use the pre-coincedences of the \Kr\ decay properly to be able to use it as a veto against it. 
\\





%%%%%%%%%%%%%%%%%%%%%%%%%%%%%%%%%%%%%%%%%%%%%%%%%%%%%%%%%%%%%%%%%%%%%%%%%%%%%%%%
% -----------------------------------------------  ADD SOME ACKNOWLEDGMENTS  ---
%%%%%%%%%%%%%%%%%%%%%%%%%%%%%%%%%%%%%%%%%%%%%%%%%%%%%%%%%%%%%%%%%%%%%%%%%%%%%%%%
\newpage
\section*{Acknowledgments}

%%%%%%%%%%%%%%%%%%%%%%%%%%%%%%%%%%%%%%%%%%%%%%%%%%%%%%%%%%%%%%%%%%%%%%%%%%%%%%%%
%%%%%%%%%%%%%%%%%%%%%%%%%%%%%%%%%%%%%%%%%%%%%%%%%%%%%%%%%%%%%%%%%%%%%%%%%%%%%%%%

%----------------------------------------------------------------- appendix ----
%%%%%%%%%%%%%%%%%%%%%%%%%%%%%%%%%%%%%%%%%%%%%%%%%%%%%%%%%%%%%%%%%%%%%%%%%%%%%%%%
% --------------------------------------------------- ADD APPENDIX (IN CASE) ---
%%%%%%%%%%%%%%%%%%%%%%%%%%%%%%%%%%%%%%%%%%%%%%%%%%%%%%%%%%%%%%%%%%%%%%%%%%%%%%%%
% \appendix

% \section{}
% \label{}

%--------------------------------------------------------------- references ----
\backmatter

\bibliographystyle{plain}
\bibliography{references}
\listoffigures
\listoftables

\end{document}
